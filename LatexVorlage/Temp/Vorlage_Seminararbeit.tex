\documentclass[12pt, a4paper]{article} % Dokumentenklasse

\usepackage[utf8]{inputenc} % Zeichensatz und Schrift
\usepackage[T1]{fontenc}
\usepackage{lmodern} % nice alternative is {mathpazo}
\usepackage[english]{babel}

\usepackage[left=2.5cm, right=2.5cm, top=2.5cm, bottom=2.50cm]{geometry} % Seitenformat

\usepackage{amsmath} % Mathematikzeichen
\usepackage{amsfonts}
\usepackage{amssymb}
\usepackage{mathtools}

\usepackage{booktabs} % Tabellen
\usepackage{array}
\usepackage{dcolumn}
\usepackage{tabularx}
\usepackage{threeparttable}

\usepackage{graphicx} % Grafiken
\usepackage{subfigure}

\usepackage{caption} % Paket für Tabellen- und Bildbeschriftungen
\usepackage{lipsum} % Paket für Random-Text (nur für Anschauungszwecke benötigt, kann gelöscht werden)

\usepackage{setspace} % Zeilenabstand
\onehalfspacing


%% Literaturverzeichnis und Zitation % %

\usepackage[%
citestyle=authoryear-comp,%
bibstyle=JME,%
maxbibnames=99,% %maximum number of names printed in bibliography before truncation with ``et al.'' is used
maxcitenames=2,
datezeros=false,% no leading 0 if dates are printed
date=long,%
isbn=false,% show no ISBNs
natbib=true,% enable natbib-compatibility
url=false,% show no urls
backend=biber % use Biber here
]{biblatex}

% vergrößert Abstand zwischen den Quellen im Lit.verzeichnis

\setlength\bibitemsep{1.5\itemsep}

% setzt "et al." bei deutscher Sprache

\DefineBibliographyStrings{ngerman}{%
  andothers = {et\addabbrvspace al\adddot}
}


% vermeidet Witwen und Waisen

\clubpenalty10000 
\widowpenalty10000
\displaywidowpenalty=10000 



% Hier .bib-Datei einfügen

\bibliography{bspbib}

%---------------------------------------------------------------------------------------
%
%								HIER BEGINNT DAS DOKUMENT
%
%---------------------------------------------------------------------------------------


\begin{document}

% TITELSEITE

\begin{titlepage}
\begin{center}
\includegraphics[scale=0.65]{KITlogo}\\
%\Large
%Karlsruher Institut für Technologie\\
\large
Fakultät für Wirtschaftswissenschaften\\
Institut für Volkswirtschaftslehre (ECON)
\vfill{
%\\\vspace{2cm}
\Large
Seminar in Macroeconomics\\\vspace{0.5cm}
,,Financial Instability and Prudential Regulation``\\\vspace{0.5cm}
Winter semester 2018/19% \vspace{2cm}
}
\vfill{
\LARGE
% -------------------------------------------------------------------------------------
% Ihr Thema
Financial Contagion
% -------------------------------------------------------------------------------------
\\ 
\vspace{0.5cm}
\normalsize
% -------------------------------------------------------------------------------------
% Nummer Ihres Themas
(Topic 1)
% -------------------------------------------------------------------------------------
}
\end{center}

\vfill{
\normalsize
% -------------------------------------------------------------------------------------
% Ihre pers\"{o}nliche Daten
\noindent Max Mustermann \\
Matr.-Nr. 00000 \\
max.mustermann@kit.edu}
% -------------------------------------------------------------------------------------
\end{titlepage}
\newpage

% VERZEICHNISSE

\pagenumbering{Roman}
\tableofcontents
% \listoffigures
% \listoftables
\newpage
\pagenumbering{arabic}


% ================================================================================
% 									Hauptteil
% ================================================================================


\section{Introduction}

\lipsum
\citet{BenignoWoodford2005} argue that ..., while the analysis in \citet{GTT2013} shows the opposite result. 

\section{Main part}

\subsection{Part 1 of main part}

\subsection{Part 2 of main part}

\section{Conclusion}




% ================================================================================
% 							Literaturverzeichnis
% ================================================================================

\newpage
\printbibliography[
heading=bibintoc,
title={Bibliography}
]


% ================================================================================
% 								Appendix
% ================================================================================

%\appendix
%\section{Anhang}




\end{document}