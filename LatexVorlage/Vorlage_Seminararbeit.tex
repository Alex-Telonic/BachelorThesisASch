\documentclass[12pt, a4paper]{article} % Dokumentenklasse

\usepackage[utf8]{inputenc} % Zeichensatz und Schrift
\usepackage[T1]{fontenc}
\usepackage{lmodern} % nice alternative is {mathpazo}
\usepackage[english]{babel}

\usepackage[left=2.5cm, right=2.5cm, top=2.5cm, bottom=2.50cm]{geometry} % Seitenformat

\usepackage{amsmath} % Mathematikzeichen
\usepackage{amsfonts}
\usepackage{amssymb}
\usepackage{mathtools}

\usepackage{booktabs} % Tabellen
\usepackage{array}
\usepackage{dcolumn}
\usepackage{tabularx}
\usepackage{threeparttable}
\usepackage{multirow}

\usepackage{float}

\usepackage{graphicx} % Grafiken
\usepackage{subfigure}

\usepackage{threeparttable} % footnotes for captions

\usepackage[bottom]{footmisc} % fix footnote at bottom of page

\usepackage{caption}

\usepackage{setspace} % Zeilenabstand
\onehalfspacing

\usepackage{placeins}

%% Literaturverzeichnis und Zitation % %

\usepackage[%
citestyle=authoryear-comp,%
bibstyle=JME,%
maxbibnames=99,% %maximum number of names printed in bibliography before truncation with ``et al.'' is used
maxcitenames=2,
datezeros=false,% no leading 0 if dates are printed
date=long,%
isbn=false,% show no ISBNs
natbib=true,% enable natbib-compatibility
url=false,% show no urls
backend=biber % use Biber here
]{biblatex}

% vergrößert Abstand zwischen den Quellen im Lit.verzeichnis

\setlength\bibitemsep{1.5\itemsep}

% setzt "et al." bei deutscher Sprache

\DefineBibliographyStrings{ngerman}{%
  andothers = {et\addabbrvspace al\adddot}
}


% vermeidet Witwen und Waisen

\clubpenalty10000 
\widowpenalty10000
\displaywidowpenalty=10000 



% Hier .bib-Datei einfügen

 \bibliography{bspbib.bib}




\usepackage[font=scriptsize, labelfont=bf]{caption}



%---------------------------------------------------------------------------------------
%
%								HIER BEGINNT DAS DOKUMENT
%
%---------------------------------------------------------------------------------------


\begin{document}

% TITELSEITE

\begin{titlepage}
\begin{center}
\includegraphics[scale=0.65]{KITlogo}\\
%\Large
%Karlsruher Institut für Technologie\\
\large
Fakultät für Wirtschaftswissenschaften\\
Institut für Volkswirtschaftslehre (ECON)
\vfill{
%\\\vspace{2cm}
\Large
Bachelor Thesis in Macroeconomics\\\vspace{0.5cm}
}
\vfill{
\LARGE
% -------------------------------------------------------------------------------------
% Ihr Thema
U.S. Commercial Banks: Trends and Cycles
% -------------------------------------------------------------------------------------
\\ 
\vspace{0.5cm}
\normalsize
% -------------------------------------------------------------------------------------
% Nummer Ihres Themas
% -------------------------------------------------------------------------------------
}
\end{center}

\vfill{
\normalsize
% -------------------------------------------------------------------------------------
% Ihre pers\"{o}nliche Daten
\noindent Alexander Schlechter \\
Matr.-Nr. 2054108 \\
alexander.schlechter@student.kit.edu}
% -------------------------------------------------------------------------------------
\end{titlepage}
\newpage

% VERZEICHNISSE

\pagenumbering{Roman}
\tableofcontents
% \listoffigures
% \listoftables
\newpage
\pagenumbering{arabic}


% ================================================================================
% 									Hauptteil
% ================================================================================
\iffalse
\section{ChangeLog}
\label{cha:history}

\begin{tabularx}{1\textwidth} { 
  | >{\raggedright\arraybackslash}X 
  | >{\raggedright\arraybackslash}X 
  | >{\raggedright\arraybackslash}X | }
 \hline
 \textbf{Date} & \textbf{Section: Contribution} & \textbf{Comment} \\
 \hline
 04.05.2020  & Extended and Improved Introduction  & ready for review  \\
 \hline
  04.05.2020  & Leverage General section improved and additional figures added  & ready for review  \\
 \hline
 07.05.2020  & Introduction, Data, Methods section improved &   \\
 \hline
 12.05.2020  & Overview section completed & ready for review
 \hline
  23.07.2020  & Introduction completely restructured and partly rewritten
  \hline
  
\end{tabularx}
\fi


\section{Introduction}

This thesis is an explorative investigation through the historical balance sheet filings of U.S. commercial banks. Its objective is to shed light on the financial development of arguably the most important backbone of the U.S. economy - commercial banking. Not least, the severe financial crisis in 2008, which originated from the banking industry, proves the importance of regulating commercial banking (\citet{ostrup2009origins}). However, only with a deep empirical understanding of the behaviour of commercial banks, one can design regulations that are ultimately effective. Using a dataset of balance sheets originally provided as call reports by the Federal Financial Institutions Examination Council (FFIEC), we analyse the cross-section over time to unveil interesting stylized facts about financial trends (long-term) and cycles (short-term) among commercial banks. The data gives us the unique possibility to get a detailed view into every balance sheet account of both assets and liabilities. We study the dynamics of each account on an aggregated level (all commercial banks) and by different bank sizes. With careful consideration of contextual information such as crisis and regulatory efforts in the considered time-frame, spanning the years between 1976-2013, we are able to gather time-sensitive analytics.\\
Overall, the two banking crises around 1990 and 2008 impacted the stability of commercial banking the most.
In the financial crisis, commercial banks balance sheets expand from $2007$ to the first half of $2008$ and only begin to contract in the second half of $2008$. This behaviour was related to a variety of factors such as: Investment banks bore the first impacts of the crisis, loans prepared for syndication needed to stay on commercial banks balance sheet as demand stalled, monetary interventions by the FED and more.\\
%We facilitated correlation analysis to reveal specific dynamics between balance sheet accounts.\\
Our findings indicate that larger banks tend to suffer more in crisis.
As part of the thesis focuses on the problematic commonly referred to as \textit{"Too Big to Fail}", we show the rising unequal distribution of assets among commercial banks and find that economic downturns act as a way of redistributing assets among banks.\\
In general, balance sheets vary signficantly between banks of different sizes.
Large and small banks differ in their overall balance sheet composition and risk appetite. For instance, the larger the bank, the more alternative ways of financing are utilized. Banks of different size also tend to respond differently to major economic events.
Finally, we take a look at leverage. It plays a major role in economic literature. For instance, \citet{geanakoplos2010leverage} emphasize its importance in times of crisis, by showing that leverage has a large impact on asset prices, contributing to booms and busts. We analyse leverage among commercial banks over-time and discover that in the crisis 2007/8 there was a spill-over effect, with large banks falling into distress first and smaller banks following with a lag.
Furthermore, a common area of interest regarding leverage is its pro-cyclicality with assets. \citet{AdrianShin2011} found pro-cyclical leverage for all commercial banks.\footnote{\citet{AdrianShin2011} study the industry leverage by aggregating assets and equity.} To confirm that their findings are robust, we varied a number of factors in our computations such the use of a longer timeframe of the underlying data and slightly different variables. We also distinguish between different bank sizes. We discover that \citet{AdrianShin2011} results are only partially true. While large commercial banks do show pro-cyclical leverage, small banks actually show no clear cyclical leverage pattern in regards to changes in cyclical assets over the time frame from $1976-2013$. While the leverage of the small bank sector as a whole is pro-cyclic, the average small bank does not feature pro-cyclical leverage.\\
In addition, we find that pro-cyclicality is not consistent over the time frame from $1980-2010$. Although the industry leverage does seem to be pro-cyclical most of the time, the average commercial bank does have pro-cyclical leverage from $1980-1990$, but counter-cyclical leverage from $2000-2010$. 
Thus, although we can not observe consistent pro-cyclical leverage among commercial banks, we can confirm \citet{AdrianShin2011} notion that commercial banks tend to actively manage leverage in regards to cyclical asset variations.\\
The thesis is structured as follows. We begin by outlining the data used. Then, we give a general overview of commercial banks and elaborate on each balance sheet position. A section looking at the distribution of assets among banks follows. We then continue by analysing banks of different sizes. Lastly, we examine commercial banks' leverage.

\iffalse
 categories to find see their development and relationships with each other. 
We 

to analyse the cross-section of banks over time by considering distributions, standard deviations and other valuable indicators such as the Gini coefficient. Furthermore, careful consideration of contextual information such as crisis and regulatory efforts in our time-frame - years 1976-2013 - makes it possible to gather time-sensitive analytics.
We give an overall view of the dynamics of commercial banks on an aggregate level and find interesting dependencies between the different balance sheet accounts. (For instance, the co-movement between securities and loans is significantly positive.)
We show the significant unequal distribution of assets among banks and find that economic downturns act as way of redistributing assets among banks, indicating that larger banks tend to be more affected by crises.
To facilitate our data we split the cross-section into several bank size categories and see that bank size strongly impacts bank behaviour. They strongly differ in their asset cycles and balance sheet composition.  


- Gini coefficient 


Within the analysis we also careful consider contextual information such as crises and regulatory efforts that occur in our time-frame, the years 1976-2013. Some areas of interest we will elaborate on are the growth and distribution of assets, the relationship between different balance sheet accounts and leverage. We gather interesting insights such as that small banks asset cycles are independent to that of large banks.\\
This thesis should be seen as a complement and a way of clarification to the wide variety of existing literature exploring similar themes. Adrian and Shin 2011, for instance, investigate the pro-cyclicality of leverage. We use their approaches as a basis to apply them on our data and compared the results. We find that indeed commercial banks do have pro-cyclical leverage, but it differs between bank sizes. Similar to DeYoung and Yom 2008, we also do correlation analysis between different balance sheet accounts, but we do not go in such great depths.\\
We start by outlining the data and methods used. Then, we give a more general overview of commercial banks and elaborate on each balance sheet position. A section about the development of distribution of assets follows. We then continue by analysing banks by different asset sizes. Lastly, we examine the important economic indicator leverage for commercial banks.

This thesis is an explorative investigation through the historical balance sheet filings of U.S. commercial banks. Its objective is to shed light on the financial development of arguably the most important backbone of the U.S. economy - commercial banking. Not least, the severe financial crisis in 2008, which originated from the banking industry, proves the importance of regulating commercial banking. However, only with a deep empirical understanding of the behaviour of commercial banks, one can design regulations that are ultimately effective. We use different perspectives and a variety of approaches to unveil interesting stylized facts about balance sheet trends (long-term) and cycles (short-term) among banks. Banks are studied in both dimensions - time and cross-section. Within the analysis we also careful consider contextual information such as crises and regulatory efforts that occur in our time-frame, the years 1976-2013. Some areas of interest we will elaborate on are the growth and distribution of assets, the relationship between different balance sheet accounts and leverage. We gather interesting insights such as that small banks business cycles are independent to that of large banks.\\
This thesis should be seen as a complement and a way of clarification to the wide variety of existing literature exploring similar themes. Adrian and Shin 2011, for instance, investigate the pro-cyclicality of leverage. We use their approaches as a basis to apply them on our data and compared the results. We find that indeed commercial banks do have pro-cyclical leverage, but it differs between bank sizes. Similar to DeYoung and Yom 2008, we also do correlation analysis between different balance sheet accounts, but we do not go in such great depths.\\
We start by outlining the data and methods used. Then, we give a more general overview of commercial banks and elaborate on each balance sheet position. A section about the development of distribution of assets follows. We then continue by analysing banks by different asset sizes. Lastly, we examine the important economic indicator leverage for commercial banks.

\fi

\iffalse
   ,  We outline trends and business cycles. 

Findings:


Analysis under close consideration of economic business cycles.  

Different perspectives: 
\begin{itemize}
\item banking sector as whole (Aggregated)
\item Banks categories
\end{itemize}

Analysis timeseries and cross section

Long term trend analysis vs short term business cycles analysed

Some questions approached: 
\begin{itemize}
\item How did the balance sheets of commercial banks evolve over time?
\item To what extent are balance sheet positions pro-cyclical, with regards to crisis and trough definitions by the NBER ?
\item Are there relationships between different balance sheet positions on an aggregate level?
\item How did the commercial bank landscape change over time in regards to asset size?
\item How did the balance sheets of commercial banks of different sizes evolve over time?
\item To what extent is leverage pro-cyclical ?
\item Are there differences in leverage behaviour between different banks categorized by asset size?
\end{itemize}

Explain structure of thesis...

\fi
\section{Main part}

\subsection{Data}
\label{sec:data}

The analysis in this thesis is build upon a dataset of balance sheets originally provided by the FFIEC. Also named call reports, the FFIEC collects balance sheet information quarterly from every FDIC insured institution. \citet{DrechslerSchnabel2017} used these reports and formed a consistent time-series from year 1976 to 2013, accounting for variable and other changes over the years. They only included commercial banks (Charter Type 200).
% Most of the accounts in those call-reports were recorded at fair value. 
%To plot these time-series we create a horizontal axis with a tick for every quarter. We also add a year label for every first quarter. This axis is consistently used throughout the analysis. 
Bank filings with negative equity are removed from the dataset, since they indicate a bankrupt bank. To prevent skewing the data, the two big investment banks Goldman Sachs and Morgan Stanley, becoming commercial banks in the proceedings of the financial crisis 2008, are removed. When looking at leverage, we aggregate all commercial banks to their belonging bank holding companies. %For our use-case it is not necessary deflate the data.\footnote{We are only interested in the actual priced value of the banks assets and not the quantity of the assets. Meaning for instance the banks could hold ten assets in 2000 with a value of 100\$ and ten in 2013 with a value of 150\$ caused by a rise the overall price level. Although there was no welfare increase as the quantity did not rise, the value the banks held still increased.} 
In the analysis it was often a few key players that drive the measurements. This aligns with the interdependent bank system of today, where just one "too big" bank going bankrupt can lead to significant spillover effects. Hence, we took those key players into careful consideration and did not filter them out as outliers. 
In the proceedings of the analysis, we took recession definitions provided by the National Bureau of Economic Research into account. They define a recession not in terms of two consecutive quarters of decline in real GDP, but a significant decline in economic activity spread across the economy, lasting more than a few months, normally visible in real GDP, real income, employment, industrial production and wholesale-retail sales (\citet{NBERBusinessCycles}). In addition, we differentiate between so called "banking" (originated in the banking sector) versus "market" (originated from outside banking sector) crisis as in \citet{BergerBouwman2013}. The assumption is that banking crisis are more strongly reflected in bank data.
The banking crisis are the credit crunch of the early 1990s (1990:Q3-1991:Q2) and the 2007/8 financial crisis (2007:Q4-2009:Q3). The market crisis are the two 1980s recessions (1980:Q1-1980:Q3 and 1981:Q3-1982:Q4) and the dotcom bubble (2001:Q2-2001:Q4). Additional events that could be considered as crisis, but not mentioned by the NBER, are the 1987 stock market crash (1987:Q4), the Russian debt crisis and the Long-Term Capital Management (LCTM) bailout of 1998 (1998:Q3–1998:Q4) and the terrorist attacks in early 2000s. \citet{BergerBouwman2013}, for example, also included these events in their analysis.
Apart from those crisis, it is important to consider other structural events that affected the U.S. commercial banks landscape considerably. We describe the most important ones here. 
The Gramm-Leach-Bliley Act in 1999 repealed part of the Glass-Steagal Act of 1933, removing barriers that prevented banks from offering traditional commercial banking services and investment banking services or insurance company services at the same time.  
The Reigle-Neil law in 1994 removed several obstacles to banks opening branches in other states and provided a uniform set of rules regarding banking in each state.
The FDIC Improvement Act (FDICIA), passed in 1991, gave the FIDC the responsibility to rescue banks with the least-costly method. Aimed to relativize the evolving moral hazard. To improve banking sectors' stability, regulators started to implement capital and liquidity regulations with the Basel 1 framework in 1988. They released further improvements of this framework with Basel 2 in 2004 and Basel 3 in 2010. Lastly, during the considered time-frame the banking sector experienced a wide-spread adoption of financial innovations, the main ones being interest rate derivatives, asset securitization and adjustable rate mortgages. 



%\subsection{Methods}

%We use a number of methods to aid the analysis of banking data over time and in the cross-section.
%For most methods we transform the data with the natural logarithm. As a result all changes can be seen as percentages. Furthermore, we apply the recognized Hodrick-Prescott Filter with the recommended parameter of 1600 for quarterly time-series to the data.\footnote{Potential seasonal effects are not accounted for.} The resulting graphs show the relative cyclical variations of the underlying variable and can be interpreted as percentage changes. For correlations and autocorrelations, we use the linear Pearsons' correlation coefficient. To determine significance we compute the 2-tailed p-value. Significance is then determined according to standard levels.
\iffalse
\begin{enumerate}
\item ***: <0.01
\item **: <0.05
\item *: <0.1
\end{enumerate}
\fi

\iffalse
\begin{itemize}
\item No inflation adjustment: We are only interested in the actual priced value of the banks assets and not the quantity of the assets. Meaning for instance the banks could hold ten assets in 2000 with a value of 100\$ and ten assets in 2013 with a value of 150\$ caused by a rise the overall price level. Although there was not welfare increase as the quantity did not increase, the value the banks hold still increased.
\item Tranformations: Log, to account for relative changes and enable comparisons over time.
\item Timeseries analysis:
\item Detrending with HP Filter (Parameter: 1600)
\item (Deseasonalize with X11 procedure)
\item Correlation/Autocorrelation
\item 2-tailed p-value on correlation with significance levels: ***: <0.01
					**: <0.05
					*: <0.1
\end{itemize}
\fi


\subsection{U.S. commercial banks}
\label{sec:AssetLiabs}

This section provides an overview about the distribution of financial components held by the U.S. commercial banking sector as a whole. We will see what types and amounts of financial instruments banks hold and how these positions have evolved over time. 

\subsubsection{Stylized balance sheet}


Table \ref{tab:balancesheet} shows the balance sheet of a typical U.S. commercial bank.

\iffalse
\begin{itemize}
 \item Cash
 \item Fed funds sold and securities purchased under agreements to resell
 \item Securities
  	\begin{itemize}
 	\item Treasury
 	\item Mortage backed security
 	\item Other
 	\end{itemize}
 \item Loans net\footnote{Loans and leases net of unearned income and allowance for loan and lease losses}
 \item Trading assets
 	\begin{itemize}
 	\item Interest rate derivatives
 	\item Other fixed income
 	\item Other trading
 	\end{itemize}
 \item Other assets \footnote{composed of derivatives "not for trading" and other items}
 \end{itemize}
 
 \fi
 
\begin{figure}[H]
\begin{minipage}{\textwidth}
\centering
 \begin{tabular}{|l|l|}
 \hline
 \textbf{Assets} & \textbf{Liabilities}  \\ \hline \hline
 Cash & Equity \\ \hline
 \begin{tabular}{@{}l}Fed funds sold and securities purchased \\ under agreements to resell (fedfundsrepoassets)\end{tabular} & \begin{tabular}{@{}l} Fed funds bought and securities sold \\ under agreements to repurchase \end{tabular} \\ \hline
 Securities: & Deposits:  \\ 
 \begin{tabular}{l} - Treasury \\ - Mortgage-backed Security (MBS) \\ - Other \end{tabular} & \begin{tabular}{l} - short \\ - other \end{tabular} \\ \cline{1-2}
 Loans net & Other borrowed money\\ \hline
  Trading assets: & Trading liabilities:  \\ 
\multicolumn{2}{|c|}{\begin{tabular}{l} - net interest rate derivatives  \\  - net other fixed income \\ - net other trading \end{tabular}} \\ \hline
 Other assets & Other liabilities \\ \hline
 \end{tabular} \\ 
 \caption[1]{Stylized balance sheet of U.S. commercial bank}
 \label{tab:balancesheet}
 \end{minipage}
\end{figure}


We simplify the balance sheet of a typical U.S. commercial bank as in \citet{begenau2015banks}.
It is important to note that every position besides the trading assets are not held for trading purposes. For example, the securities position and the loans position are not held for trading.
Cash consists of noninterest-bearing balances, with currency and coin included, and of interest-bearing balances.
Federal funds sold and securities purchased under agreements to resell are both ways of lending excess cash to fellow commercial banks in return for interest. Fed funds bought and securities sold under agreements to repurchase in turn are ways of borrowing cash in the short-term. Securities can be divided into held-to-maturity and available-for-sale. These categories then include a large amount of different types of securities, with Treasury and MBS being the largest. Loans are netted by unearned income and allowance for loan and lease losses to gather their existent value. Trading assets are securities held with the intention to sell them with profit. They are intended to be held only for short-term. Trading asset can be in any type of form such as a derivative, Mortage-backed Security (MBS) or a loan. Trading liabilities tend to be in the form of short positions or derivatives. Deposits can be divided into transaction and non-transaction deposits. Time and savings deposits make up non-transaction deposits, while the major part of transaction deposits are demand deposits. Other assets are composed of derivatives "not for trading" and other items that have a small share and do not fit into the named categories.
 
\subsubsection{Overview}
Figure \ref{fig:assets} shows how the aggregated total assets held by all banks have evolved over time. 
 
\begin{figure}[H]
\begin{minipage}{\textwidth}
\includegraphics[scale=0.3]{graphs/DescriptiveStats/OtherAnalysis_clean_AssetDistribution_7613}
\caption[1]{Assets}
\label{fig:assets}
\end{minipage}
\end{figure} 
 
 
In particular, the first panel in Figure \ref{fig:assets} shows how the aggregate total assets split into its accounts evolved over time. The value of assets rose from below $2$ trillion to above $13$ trillion dollars. In comparison, the GDP of U.S. rose from $1.9$ trillion in $1976$ to $16.78$ trillion in $2013$. %The second panel in Figure \ref{fig:assets} compares the logged with the absolute value of total assets. The logged assets are plotted on the left vertical axis and grow linearly while the absolute assets are plotted on the right vertical axis and grow exponentially. 
The second panel shows the growth rate of assets per year. 
The two graphs already give a first insight into which periods of time had a particular impact on commercial banks assets. There are two periods with low growth - the credit crunch in $1990s$ and the $2007/8$ financial crisis. The period around $1990$ is marked by a period of consistent low growth over several years, while the financial crisis in $2007/8$ causes a short but significant year of negative growth in $2009$. We will dive into more detailed analysis of these periods when looking into the banks' business cycles.
  


\iffalse
Finally, the last graph of Figure \ref{fig:assets} shows the banks default rate per year. The graph aligns with the growth graph just above. In periods with a lot defaults we have a low growth rate. Periods that mark high default rates are from $1983-1992$ and $2009-2013$. These periods might be strongly interconnected with the two banking crisis - the credit crunch in 1990 and the financial crisis in 2007/8. The first high default period had much higher default rates and lasted much longer than the second.
The significance difference in numbers might be related to the fact that in the 1980s the amount of small banks in general and that of the defaults were considerably higher.  In the first period, $74\%$ of the banks that defaulted were small, while in 2010 the share of small banks defaults was only at $35\%$. We elaborate on the change in banking landscape in section \ref{sec:ToBigToFail}. The reason why the first period lasted much longer cannot be easily explained. According to \citet{federal1997history} there were various forces working together to produce this long period of defaults. Hence, the 1990 credit crunch might be related to the defaults, but is not seen as the major cause.
Another point regarding the crisis 2007/8 is the timing of the defaults - two years after the beginning of the crisis. This again, might be related to the argument \citet{antoniades2019commercial} make about the funding pressures being a main characteristic of the crisis 2007/8. Funding pressures caused investment banks to default in the crisis 2007/8, but not commercial banks. These defaulted later by the deterioration of assets in the real estate sector.

In \citet{antoniades2019commercial} it is argued that the crisis itself was marked by sudden aggregate funding pressures and that this was the cause of defaults among investment banks. However, the defaults of commercial banks were not caused by these funding pressures but by the deterioration of assets in the real estate sector, which was a longer process beginning in 2006 and lasting until 2013.
\fi




\subsubsection{Banks' balance sheet: Cycles}

To analyse commercial banks business cycles, we filter the time series for aggregate assets to extract the cyclical component, shown in Figure \ref{fig:cyclial_assets}. 

\begin{figure}[H]
\begin{minipage}{\textwidth}
\includegraphics[scale=0.3]{graphs/DescriptiveStats/OtherAnalysis_clean_CyclicalAssets_7613.png}
\caption[1]{Cyclical Assets}
\label{fig:cyclial_assets}
\end{minipage}
\end{figure}

Additionally, we plotted the cyclical component of each balance sheet account in Figure \ref{fig:positions}. In Figure \ref{fig:positions}, the left column represents the asset side and the right column the liabilities side of a balance sheet. 


\begin{figure}[H]
\begin{minipage}{\textwidth}
\includegraphics[scale=0.3]{graphs/DescriptiveStats/OtherAnalysis_clean_PositionsCyclical_7613.png}
\caption[1]{Cyclical asset accounts(left column) and cyclical liability accounts(right column). Trading assets and liabilities have missing data in the beginning of the time period.}
\label{fig:positions}
\end{minipage}
\end{figure}

The cyclical movements in all figures can be interpreted as percentage changes and the gray areas indicate crises, as defined in section \ref{sec:data}. Note that the aggregate balance sheet accounts mainly represent the large banks, because of their large market share. We address this point in section \ref{sec:ToBigToFail} and cover banks of different sizes in section \ref{sec:banksByAssetSize}.\\
In the following paragraphs, we focus the analysis on the most significant crisis periods. For this case we consider the definitions of crisis by NBER. Nevertheless, we are almost certain that the NBER crisis periods do not always align with the balance sheets of commercial banks. After all, the NBER does consider more factors than the business cycles of commercial banks. However, commercial banks' lending and transaction practises do play a key role in the overall economic welfare of a country. As a result we would expect that their business cycles match an economies boom and bust cycles to some extent.


\subparagraph{Financial crisis 2007/8}
%crisis schlägt sich erst ein bisschen später im asset cycle wieder. 
The 2007/8 financial crisis is reflected in the asset cycle of commercial banks with a lag. The assets experience a significant boom leading up the crisis, and only after $2008$ Quarter 3 we see a rapid decrease in value.

%For the financial crisis in 2008, the total assets cycle performs different than one might anticipate. There is a large positive variation in the middle of the crisis with a peak in $2008$ Quarter 3 followed by a rapid drop back to the trend. In particular the timing of the cyclical assets spike as well as that the rise above the trend was proportionally larger than the fall below were unexpected. 

To explain this observation, several factors need to be accounted for. First, especially in the beginning of the crisis the major and more direct effects were born by the investment banks. In \citet{antoniades2019commercial} it is argued that the crisis itself was marked by sudden aggregate funding pressures. These funding pressures had a much stronger impact on investment banks than on commercial banks. Hence, the assets of investment banks might have decreased immediately with the beginning of the $2007/8$ crisis, but not those of commercial banks. They provided a key source of liquidity at the beginning of the crisis for investment banks as financial markets liquidity dried up. Commercial banks were later affected by the general deterioration of assets in the real estate sector, which was a longer process beginning in 2006 and lasting until 2013 according to \citet{antoniades2019commercial}.
Second, the Federal Reserve Bank (FED) used a series of regulatory efforts to ease the impacts of the crisis.\footnote{See the Monetary Policy Report to Congress mentioned in the bibliography - \citep{FEDReport}} These had an effect on the valuation of commercial banks' assets and might be the reason why the maximum growth rate of over $6$\% was substantially larger than the negative growth rate of just over $2$\% that followed. 

 why the spike in assets in the crisis was substantially larger than the fall that followed. Next to other smaller interventions, a Troubled Asset Relief Program (TARP) was passed by congress to reduce the negative impact of the substantial amount of  illiquid  structured  securities  and  mortgages  still held  by  banks. 
Last, as outlined in \citet{bech2009profits}, major restructuring events occur over the crisis period, with acquisitions and mergers boosting aggregate assets by more than $580$ Billion dollar. Removing the most relevant restructuring event in $2008$, that is the acquisition of  Washington Mutual  Bank's  by JPMorgan Chase, we can see a reduced second spike in Figure \ref{fig:cyclial_assets_adjusted} in the crisis period. Table \ref{tab:cyclicalAssetsComparison_Crisis} also shows that in third quarter of $2008$ ($2008-09-30$) the adjusted cycle is $1.3$\% lower. 

\begin{figure}[H]
\begin{minipage}{\textwidth}
\includegraphics[scale=0.3]{graphs/DescriptiveStats/OtherAnalysis_clean_CyclicalAssetsAdjusted_7613.png}
\caption[1]{Cyclical Assets adjusted to 2008 merger activities.}
\label{fig:cyclial_assets_adjusted}
\end{minipage}
\end{figure}

\begin{table}[H]
\begin{minipage}{\textwidth}
\input{graphs/Tables/LatexTables/cyclicalAssetsAdjustedCompare_Crisis.tex}
\caption[1]{Comparison of asset cycles in 2007/08 crisis. The adjusted asset cycle has 2008 merger activities removed. The cycles are in percentage.}
\label{tab:cyclicalAssetsComparison_Crisis}
\end{minipage}
\end{table}

%the cyclical shares of balance sheet accounts in regards to total assets would be helpful in identifying the driver for asset spike. 

%loans and securities cycles
We are also interested in the main balance sheet accounts that drove the cyclical behaviour in $2008$. In general, loans and securities have the largest share in regards to total assets and were likely the driver of aggregate assets. All other accounts have a share below $10\%$ during the crisis.\footnote{Figure \ref{fig:share_both} shows the share of balance sheet accounts. We will address this figure in detail in section \ref{sec:shareBalanceSheetAccounts}} In Figure \ref{fig:positions} we see that securities have its lowest point at the beginning and then rise over the period of the crisis. Hence, securities might only have contributed for the second spike in mid-$2008$. Loans, however, match the behaviour of aggregate assets and could be the main contributor to cyclical aggregate asset movements in $2007/8$. \citet{bassett2008profits} mention that the reason for the strong loan growth in $2007$, results from loans that banks planned to move off the balance sheet by selling them to investors. However, investors suddenly lost interest in these loans towards the end of $2007$, because of concerns about their quality. This forced the banks to keep the loans on their balance sheet. 
In the second half of $2008$ loans then fell. This might have been partly caused by an increase in loan loss provision, the loans are netted with.
%Loan loss provision
Figure \ref{fig:loan_loss_provision} shows us the loan loss provision account cycle to underline this thought. An increase of more than $50\%$ in the 2007/08 crisis can be observed.

\begin{figure}[H]
\begin{minipage}{\textwidth}
\includegraphics[scale=0.3]{graphs/DescriptiveStats/OtherAnalysis_clean_LoanLossProvision_7613.png}
\caption[1]{Cyclical Loan Loss Provision}
\label{fig:loan_loss_provision}
\end{minipage}
\end{figure}

%other balance sheet accounts
Other balance sheet accounts also behave differently than normal in the crisis, as seen in Figure \ref{fig:positions}.
Cash, for instance, rose over $25\%$ in the crisis, outshining every cyclical movement of cash in other time periods. With the background of the crisis it seems reasonable that banks liquidated assets in $2008$ to be prepared for potential liquidity pressures. 
Other borrowed money and foreign deposits show significant growth in end of $2007$ and beginning of $2008$. This was followed by an immediate drop in the second half of $2008$, similar to loans. Here, part of the loan growth observed leading up to crisis, as mentioned by \citet{bassett2008profits}, was financed by other borrowed money and foreign deposits.\footnote{Section \ref{sec:shareBalanceSheetAccounts} deals with other borrowed money in more depth.} A positive correlation of both accounts with loans supports this thought.\footnote{Other borrowed money has a correlation of $0.45$ and foreign deposits a correlation of $0.59$ with loans (see Table \ref{tab:corr_comb}). Note, correlations are computed over the whole time-period $1976-2013$. We will address correlations in a following paragraph.
} 
Finally, trading assets cycle falls over the period of the crisis from $+25\%$ to just below $0\%$.
%The rise in cash comes along with a significant fall in securities. Figure \ref{fig:share_both} further supports our thesis that securities are sold to raise cash. Here we can see that while the share of securities fell until 2008, the share of cash rose.



\subparagraph{Other crisis}
The early $1980s$ recession does not have a stronger impact on the cyclical movements than periods without crisis. But the $1990s$ credit crunch and 2001 dotcom bubble triggered downward movements of the commercial bank asset cycles below the trend. These two crises lead the asset cycles to reach their lowest points - in $1993$ around $-4\%$ and $2002$ around $-4.2\%$ . The two minimum points occur after the crises, indicating a lag between the crises and its effect on the balance sheet size of commercial banks.

\subparagraph{Other anomalies}
Equity has its lowest downward variation in 2003. We will see more of equities behaviour in the leverage section, when we analyse it in relation to total assets.
Another interesting observation is that some positions show larger fluctuations in more recent times. The volatility of federal funds sold and securities purchased under agreements to resell (fedfundsrepoasset) increased from 1996 onwards and of foreign deposits from 1992 onwards as seen in Figure \ref{fig:positions}.


\subparagraph{Relationships between balance sheet accounts}
We now take a look at possible relationships between commercial bank balance sheet accounts.
Table \ref{tab:corr_assets}, \ref{tab:corr_liab} and \ref{tab:corr_comb} show the correlations between each of the balance sheet accounts.\footnote{Note, the mere assumption of a correlation between the two sides of a balance sheet contradicts the Modigliani-Miller-Theorem. The Theorem states the independence of assets by the financing capital structure. In addition, a key part of asset liability management for banks is maturity transformation. For correlation analysis, we should have differed between the different maturities of assets and liabilities. Correlations between positions of different maturity would have a more causal relationship. Furthermore, canonical correlation analysis could have been used to consider that balance sheet positions are jointly determined by the other positions.}
We find a strong inverse relationship of -0.73 between loans and securities as reported in Table \ref{tab:corr_assets}. When securities fall, loans rise and vice versa. The scatter plot in Figure \ref{fig:scatterplots} illustrates this negative relationship. This does not come as a surprise, as in the process of securitization usually part of the loans are packaged into securities, such as, mortgage backed securities and others.\\
Furthermore, there is a small positive relationship between fedfundsrepoassets and trading assets. This could indicate that banks lending out excess federal funds (fed funds) or purchasing repurchase agreements (repos) are in such a healthy position to be able to increase trading assets as well. The scatterplot of this relationship in Figure \ref{fig:scatterplots} confirms a possible positive linear correlation. A similar positive relationship can be seen between fedfundsrepoassets and loans. However, the scatterplot in Figure \ref{fig:scatterplots} does not support a clear relationship.\\
Domestic deposits are also negative correlated with foreign deposits (-0.34) and other borrowed money (-0.23). \citet{bassett2008profits} mention that in the financial crisis 2007/8, commercial banks turn to foreign deposits and other borrowed money for financing as domestic deposits fall. This behaviour could explain the inverse relationship in general. Foreign deposits and borrowed money can be seen as alternative ways of financing to compensate for fluctuations in deposits.
In addition, there is a positive correlation of $0.37$ between equity and trading assets, indicating that increases in equity lead to increases in trading.\\
%Another observation is the strong positive relationship between loans and foreign deposits of $r=0.59$. Foreign deposits are deposits made in foreign offices. 



\begin{table}[H]
\begin{minipage}{\textwidth}
\scriptsize
\input{graphs/Tables/LatexTables/corr_assets.tex}
\caption[1]{Pearson Correlation Coefficient for Assets. Significance levels are: $*** : p<0.01$, $** : p<0.05$, $* : p<0.1$ }
\label{tab:corr_assets}
\end{minipage}
\end{table}

\begin{table}[H]
\begin{minipage}{\textwidth}
\scriptsize
\hskip-2.0cm
\input{graphs/Tables/LatexTables/corr_liab.tex}
\caption[1]{Pearson Correlation Coefficient for Liabilities}
\label{tab:corr_liab}
\end{minipage}
\end{table}


\begin{table}[H]
\begin{minipage}{\textwidth}
\scriptsize
\hskip-2.0cm
\input{graphs/Tables/LatexTables/corr_liab_assets.tex}
\caption[1]{Pearson Correlation Coefficient between Assets and Liabilities}
\label{tab:corr_comb}
\end{minipage}
\end{table}

\iffalse
\begin{figure}[H]
\begin{minipage}{\textwidth}
\centering
\caption[1]{Correlation assets\footnote{Pearsons correlation coefficient based on the detrended data used in Figure \ref{fig:positions}.} }
\includegraphics[scale=0.8]{graphs/Tables/asset_cycle_corr_sig}
\label{fig:corr_assets}
\end{minipage}
\end{figure}
\hskip-2.0cm


\begin{figure}[H]
\begin{minipage}{\textwidth}
\centering
\caption[1]{Correlation liabilities\footnote{Pearsons correlation coefficient based on the detrended data used in Figure \ref{fig:positions}.} }
\includegraphics[scale=0.8]{graphs/Tables/liab_cycle_corr_sig}
\label{fig:corr_liab}
\end{minipage}
\end{figure}

\begin{figure}[H]
\begin{minipage}{\textwidth}
\centering
\caption[1]{Correlation assets with liabilities\footnote{Pearsons correlation coefficient based on the detrended data used in Figure \ref{fig:positions}.} }
\includegraphics[scale=0.8]{graphs/Tables/comb_cycle_corr_sig}
\label{fig:corr_comb}
\end{minipage}
\end{figure}
\fi

\begin{figure}[H]
\begin{minipage}{\textwidth}
\centering
\includegraphics[scale=0.45]{graphs/DescriptiveStats/OtherAnalysis_clean_scatterplots_7613}
\caption[1]{Scatterplot for selected positions and linear regression based on the detrended data used in Figure \ref{fig:positions}. Shaded area indicates a confidence interval.}
\label{fig:scatterplots}
\end{minipage}
\end{figure}

\subsubsection{Banks' balance sheet composition: Trends}
\label{sec:shareBalanceSheetAccounts}
Further elaborating on Figure \ref{fig:share_both}, we can see that loans are the main target of investment for commercial banks. Throughout the 37-year time frame the share of loans stays between $50$ and $60\%$. Banks start with a share at $55\%$ until it rises to just above $60\%$ from 1985 onwards. The credit crunch crisis in 1991 causes a fall of the share back to $55\%$. This fall continued until 1995. From then on, the share of loans rise back to $60\%$ until 2008, where it starts to fall again. It falls to an all-time low in 2013 with a share of just above $~50\%$. This came along with a rise in securities, confirming the observed negative relationship between securities and loans in the section before. 
The development of the cash share is also interesting. Cash continuously falls from a share of just below $20\%$ to a share of below $5\%$. Here, the 2007/08 crisis also marks a turning point with the cash share rising to above $10\%$.
On the liability side, deposits are the dominating source of funding for commercial banks. The share starts in 1976 with $70\%$ and falls until 2008 to an all time low of just above $50\%$. From there it goes back to roughly $65\%$. This decrease in deposits, especially until 2008, must obviously come along with the increases of other types of finance. In particular, other borrowed and foreign deposits rise with the decrease of domestic deposits. This again confirms the thought raised in the sections before that they are seen as an alternative source of financing when domestic deposits decrease.
Other borrowed money has a peak in 2008. Other borrowed money consists of Federal Home Loan Bank advances (FHLB) and other borrowings not clearly defined. After the crisis in 2008 there is a rapid decrease in other borrowings. FHLB advances are mainly used in funding mortgages for low income households, which explains the alignment with the housing crisis in 2008.\footnote{See the Affordable Mortgage Lending Guide by the FDIC mentioned in the bibliography \citet{FHLB}} Lastly, Figure \ref{fig:share_both} shows a general increase in the share of equity commercial banks hold from just above $5\%$ to above $10\%$. 

\begin{figure}[H]
\begin{minipage}{\textwidth}
\includegraphics[scale=0.3]{graphs/DescriptiveStats/OtherAnalysis_clean_sharePositions_7613.png}
\caption[1]{Share of balance sheet positions. The first two graphs show assets, where the second is a focus of the first. The third and fourth panel show liabilities, where the fourth is a focus of the third.}
\label{fig:share_both}
\end{minipage}
\end{figure}
 

\iffalse
- general more negative correlation as expected. Stealing shares
- foreign deposits and other borrowed money positive correlation. Makes sense, because other borrowed money is mainly foreign as well.
- positive relationship between equity and trading assets. More equity, more risk?
- positive relationship between trading assets and fedfundsrepoliab. Repo financing used to buy trading assets
- positive relationship between fedfundsrepoasset and trading liabilties 
\fi


\subsubsection{Defaults}

Bank failures are another way of examining the stability of the commercial banking sector over time. In Figure \ref{fig:banks_default} we show the banks default rate per year. For instance, in 1989 there were around $0.6\%$ defaults. It is based on the negative equity recorded by banks. Hence, it is not exact and some banks might continue to exist in case of mergers or bailouts. Also, sometimes banks are counted several times if despite negative equity, they continue to publish their reports more than one last time.

\begin{figure}[H]
\includegraphics[scale=0.3]{graphs/DescriptiveStats/NewLeverage_BanksDefaultPercentage_7613.png}
\centering
\caption{Bank Failures}
\label{fig:banks_default}
\end{figure}


Bank failures align with the asset growth graph shown in Figure \ref{fig:assets}. During periods with a lot of defaults we have a low growth rate. Periods that mark high default rates are from $1983-1992$ and $2009-2013$. These periods might be strongly interconnected with the two banking crisis - the credit crunch in 1990 and the financial crisis in 2007/8. The first high default period has much higher default rates and lasts much longer than the second.
The significance difference in numbers might be related to the fact that in the 1980s the amount of small banks in general and that of the defaults were considerably higher.  In the first period, $74\%$ of the banks that defaulted is small, while in 2010 the share of small banks defaults was only at $35\%$. We elaborate on the change in banking landscape in section \ref{sec:ToBigToFail}. The reason why the first period lasts much longer cannot be easily explained. According to \citet{federal1997history} there were various forces working together to produce this long period of defaults. Hence, the 1990 credit crunch might be related to the defaults, but is not seen as the major cause.
Another point regarding the crisis 2007/8 is the timing of the defaults - two years after the beginning of the crisis. This again, might be related to the argument \citet{antoniades2019commercial} makes about the funding pressures being a main characteristic of the crisis 2007/8. Funding pressures caused investment banks to default in the crisis 2007/8, but not commercial banks. These default later by the deterioration of assets in the real estate sector.



\iffalse
\begin{figure}[hbtp]
\centering
\caption{Share of asset positions - unstacked}
\includegraphics[scale=0.3]{graphs/DescriptiveStats/OtherAnalysis_ShareofAssetsPlot_7613.png}
\label{fig:sharepositions}
\end{figure}


\noindent \textit{Graph description}: Figure 1 shows the aggregates of the main variables from the asset side of the balance sheet over time. Figure 2 shows the share of each aggregated balance sheet position of all commercial banks over time. Figure 3 plots the share of each balance sheet position unstacked.\\


\noindent \textit{Key Observations:}
\begin{itemize}
\item loans make up the largest share of assets
\item share of trading assets have risen as well as interest rate derivatives
\item loans and trading assets have risen more than securities in timeframe year 2000-2009
\item Share of trading assets peaked in 2008 while securities fell.
\item There is a noticeable anomaly in year 2002. Significant amounts of repo lending is transferred into other assets. Other assets are derivatives not available for sale. 
\item drop in assets in 2002 and 2008
\item share in cash has fallen until 2008 and then increased again
\item Share of cash continously fell until 2008, and then it increased significantly again
\end{itemize}
\fi
\newpage

\iffalse

\begin{figure}[hbtp]
\centering
\caption{Banks default}
\includegraphics[scale=0.3]{graphs/DescriptiveStats/NewLeverage_BanksDefaultPercentage_7613.png}
\end{figure}

\noindent \textit{Graph description}: The graph shows an estimation of how many banks have defaulted at a certain time (year,quarter). It is based on the negative equity recorded by banks. Hence, it is not exact and some banks might continue to exist in case of mergers or bailouts. Also sometimes banks are double counted, if a negative equity does not immediately result in bankruptcy. \\
\noindent \textit{Key Observations:}
\begin{itemize}
\item main defaults in years 1986-1991 and 2009-2011
\item long stable period from 1991-2008
\item In 1990 there were many more smaller banks. Smaller banks might have a higher likelihood to fail. In 1990:  $~74\%$ small banks, 2010: $~35\%$ small banks 
\end{itemize}

\textbf{To-Do: Bank Defaults within bank categories}

\fi

\iffalse


\textbf{Not finished, more coming}

\begin{figure}[hbtp]
\centering
\caption{Loans}
\includegraphics[scale=0.3]{graphs/DescriptiveStats/OtherAnalysis_LoanDistribution_7613}
\end{figure}

\noindent \textit{Graph description}: It shows the share of loan types of total loans over time.\\

\noindent \textit{Key Observations:}
\begin{itemize}
\item real estate loans has largest share

\end{itemize}

\begin{figure}[hbtp]
\centering
\caption{Loans by repricing maturity}
\includegraphics[scale=0.3]{graphs/DescriptiveStats/OtherAnalysis_ShareofMaturityLoans_9713}
\end{figure}


\begin{figure}[hbtp]
\centering
\caption{Residential Loans by repricing maturity}
\includegraphics[scale=0.3]{graphs/DescriptiveStats/OtherAnalysis_ShareofMaturityResLoans_9713}
\end{figure}

\fi

\newpage

\subsection{Too Big to Fail: Distribution of assets among banks}
\label{sec:ToBigToFail}

This section empirically illustrates a problem commonly referred to as \textit{"Too Big to Fail}". 
Banks are considered as "\textit{too big to fail}", when their size and interconnections with other banks are so high that its individual risk impacts the systemic risk of a whole economy. Regulators tend to be reluctant to close those banks when they default allowing a moral hazard to emerge. The term first came into play with the failure and bailout of Continental Illinois National Bank and Trust Company in 1984 (\citet{nurisso20171970s}). From that point onwards, it developed into a world-wide phenomenon with its severe consequences unveiled in the financial crisis of 2008.\\
Indeed, over the last few centuries the number of banks on the U.S. landscape has fallen significantly from 14419 banks in 1976 to 6035 banks in 2013. While the mere reduction would not impose such a problem, the distribution of total assets developed more and more unequal. In 1976, the top $0.1\%$ a total of 14 banks held $32.4\%$ of all assets. In comparison, in 2013 the top $0.1\%$ - a total of 6 banks - held $50\%$ of all assets. Table \ref{table:assetsByPercentiles} and Figure \ref{fig:assetsByPercentiles} show these numbers by looking at the assets distribution by banks percentiles.

\begin{table}[H]
\begin{minipage}{\textwidth}
\scriptsize
\input{graphs/Tables/LatexTables/OtherDescriptiveAnalysis_BankCounts.tex}
\caption[1]{Count of banks by percentiles}
\label{table:assetsByPercentiles}
\end{minipage}
\end{table}

\begin{figure}[H]
\includegraphics[scale=0.3]{graphs/DescriptiveStats/OtherAnalysis_ShareofAssetsByBanksPercentiles_7613.png}
\centering
\caption{Aggregate assets by percentiles}
\label{fig:assetsByPercentiles}
\end{figure}


The unequal distribution of assets can also be seen in Figure \ref{fig:lorenz}. The curved lines show the Lorenz curve per year. The more curved the lines become, the more unequal is the distribution. The horizontal line represents perfect equality. Although in 1980 unequal distribution was high already, it increased even more. In year 2013, the top $5\%$ held almost $90\%$ of all assets. Figure \ref{fig:gini} shows us the Gini coefficient over time. Its range is from zero to one. A value of one means one bank owns everything, while a value of zero indicates perfect equality.\footnote{$10\%$ of banks own $10\%$ of assets, $50\%$ of banks own $50\%$ of assets and so on...} The higher the value, the higher the inequality in asset distribution. The trend of the coefficient supports our observation of rising inequality. An interesting observation here is the impact of crises on the asset distribution. Crisis tend to reduce the inequality and act as a way of redistributing assets. Assuming that assets values fall in times of crisis, the impact of crises must be higher on larger banks. We will look into how different bank size categories are impacted differently by crises in section \ref{sec:banksByAssetSize}.
Reasons for the trends we have just documented are not absolute clear. However, geographic deregulation and other regulation reforms such as the repeal of the Glass-Steagall act in 1999 did support the increasing inequality. In addition, larger banks are more likely to be bailed out. This puts them in an easier position to finance themselves and creates the perverse consequence of a moral hazard. A bank with a high likelihood to be bailed out takes on too much risk (\citet{FarhiTirole2012}). The severe consequences of this problem are clear since the financial crisis in 2008. Authorities responded to this issue by setting additional capital requirements on larger banks with frameworks such as Basel 1,2 and 3. 


\begin{figure}[H]
\begin{minipage}{\textwidth}
\includegraphics[scale=0.3]{graphs/DescriptiveStats/OtherAnalysis_clean_LorenzCurve_7613.png}
\caption[1]{Lorenz Curve. Values taken are always from quarter 1}
\label{fig:lorenz}
\end{minipage}
\end{figure}


\begin{figure}[H]
\begin{minipage}{\textwidth}
\includegraphics[scale=0.3]{graphs/DescriptiveStats/OtherAnalysis_clean_gini_7613.png}
\centering
\caption[1]{Gini coefficient}
\label{fig:gini}
\end{minipage}
\end{figure}




\subsection{U.S. commercial banks by size}
\label{sec:banksByAssetSize}

\subsubsection{General}

In this section we allocate banks into different categories ranked by asset size to see differences in their balance sheet behaviour. It is common approach by regulators and academics to categorize banks by their total assets. It measures the gross nominal volume of a bank's activity, but suffers from significant valuation problems, not only for derivatives, and it does not account for differences in individual bank business models. There are alternative ways of categorizing banks, such as, using capital or employees as a measure of size. \\
Following the convention of the Federal Reserve Bulletin, we divide commercial banks by assets into four categories.\footnote{Our choice of categorization could have been different. The asset size ranges they cover, differ over the years. This can be seen as an advantage or disadvantage. On the one side they evolve over the years and possibly match changing asset size levels. On the other side, there is a risk of distributional changes among the asset sizes of banks, making our chosen categorization unsuitable.} The first category are the ten largest banks. Second category covers the large banks - banks ranked from 11 through 100. The third category represent medium banks - banks ranked from 101 through 1000. Lastly, the last category are the small banks - banks ranked from 1000 and higher.
To get an overview of what asset sizes each category covers, Figure \ref{fig:cat_boxplot} contains boxplots for each category and year. Within all categories we can see a consistent rise of overall asset sizes.
In 1976, every top 10 bank has an asset size lower than a quarter of a trillion.
In 2013, the median asset size of the top 10 banks was $0.32$ trillion ($10^{12}$) with banks going up to an asset size of just under two trillion.\footnote{Note, we have not combined commercial banks with their matching bank holding company. Bank Holding Companies have asset sizes beyond two trillion.} We can also see a clear rise in heterogeneity over time regarding the asset sizes of the top 10 banks. The Interquartile Range (IQR) get to its maximum size by the end of the time-frame.
Large banks began with an asset size far below $0.25*10^{11}$ in 1976 and work their way up to asset sizes of up to $1.75*10^{11}$ dollar in 2013. The heterogeneity of large banks regarding asset size also increased over time.
Medium banks range between $~0.25-8$ billion ($10^{9}$) dollar assets per bank and small banks between $~0.25-5$ hundred million dollar assets over the given time-frame. Similar to what we will see later, the top two categories benefit more from the asset size increases. Compared to the asset increases within the top 10, the typical small bank did not show any significant gains over time. \\
Overall, the fact that the chosen categories do not have many outliers strengthens our choice of categories. Only the small banks category has a decent amount of outliers with banks that have asset sizes significantly lower than the median small bank. 

\begin{figure}[H]
\begin{minipage}{\textwidth}
\includegraphics[scale=0.3]{graphs/DescriptiveStats/BanksByAssetCategory_boxplots_eachCategory_7613.png}
\caption[1]{Boxplots for each category. Asset data is logged. Coloured boxes cover the mid $50\%$ of asset sizes - IQR:25th Percentile to 75th Percentile. For the top 10, all individual data-points are marked as dots. For the rest, only outliers are marked as dots. Outliers are data-points above 1.5 times IQR.}
\label{fig:cat_boxplot}
\end{minipage}
\end{figure}


\subsubsection{Banks' balance sheets by size: Trend and Cycles}
Figure \ref{fig:assetsbycat} shows us the development of aggregate assets by the defined banks categories over time.

\begin{figure}[H]
\begin{minipage}{\textwidth}
\includegraphics[scale=0.3]{graphs/DescriptiveStats/BanksByAssetCategory_assetsbycat_7613.png}
\centering
\caption[1]{Total assets by bank category}
\label{fig:assetsbycat}
\end{minipage}
\end{figure}

There are key points in time for each category that mark changes in their asset growth. From the start of the given time frame $1976$ until $1985$ all the categories show similar growth behaviour. Then, in $1984$, the growth of the top $10$ assets starts to slow. Shortly after that, $1985$ marks a starting point of flat, low growth for the small banks. The small banks do not recover from this low growth period until the end of our chosen time frame. An obvious reason for this is the fact that the  total number of banks also falls. Table \ref{table:assetsByPercentiles} shows 1984 marks a starting point for a continuous decrease in the number of banks. Category two and three asset growth, covering the banks ranked from $11-1000$, are alike until $1992$. From this point in time, medium banks enter a period of low and negative growth, while the large banks go on a period of high growth, together with the top 10 banks. In the 1990s, a lot of regulation reforms occur, aiding the growth of larger banks. These reforms are mentioned in section \ref{sec:data} and could be the key drivers for the growth of larger banks in the 1990s. In $2001$, the growth rate of banks ranked from $11-100$ also declined. The assets of the top $10$ banks, however, keep growing until the financial crisis in year 2008.\\
In Figure \ref{fig:catassetscycle} we can see asset cycles by each category over time. The figure also shows the standard deviation of aggregate assets by category. It seems that the larger the banks the more volatile its balance sheet size is.

\begin{figure}[H]
\begin{minipage}{\textwidth}
\includegraphics[scale=0.5]{graphs/DescriptiveStats/BanksByAssetCategory_catAssetsCycle_7613.png}
\centering
\caption[1]{Asset cycles by bank size category}
\label{fig:catassetscycle}
\end{minipage}
\end{figure}

\subparagraph{Financial crisis 2007/8}
All bank categories are affected by the crisis in 2008. For each category we see a spike, followed by a fall in assets. Note, the top two categories cover almost $90\%$ of all industries assets, hence this discussion connects to the one already held in section \ref{sec:AssetLiabs} about the whole industry asset cycle.
For a more detailed view, table \ref{tab:cyclicalAssetsByCat_Crisis} shows the raw cyclical percentage changes.
The top 10 banks are affected the most, but this does not come as a surprise as their asset cycle also has a higher volatility. Similar to behaviour of all aggregated banks in section \ref{sec:AssetLiabs}, all bank sizes have spikes in 2008, which is after the begin of the crisis defined by the NBER. The downturn that follows is also smaller than expected. In section \ref{sec:AssetLiabs}, we have outlined a variety of reasons why this might be the case. While for the top 10 and large banks the boom of assets thereupon is much stronger than the downturn that follows, medium and small banks asset boom is around the same size to the bust that follows. It seems like the larger banks show much more abnormal behaviour in the crisis than the medium and small banks. In other words, the medium and small banks experience a more standard boom and bust cycle.
The small banks downturn does not start before the second half of 2010, underlining their passive role in the crisis.


\begin{table}[H]
\begin{minipage}{\textwidth}
\input{graphs/Tables/LatexTables/cyclicalAssetsByCat_Crisis.tex}
\caption[1]{Asset cycles by bank size in crisis 2007/2008. The cycles are in percentage. Top 10 are cat1, large banks are cat2, medium banks are cat3 and small banks are cat4}
\label{tab:cyclicalAssetsByCat_Crisis}
\end{minipage}
\end{table}

\subparagraph{Other crises}
While among the large banks we can see an impact of the dotcom bubble, it passes the medium and small banks with almost no effect on their asset cycle. The dotcom bubble is marked by a significant fall in value among stocks. 
This underlines the more conservative approach by medium and small commercial banks. They do not, for instance, have a notable amount of trading assets. This might be the reason for reduced effect of this crisis.\footnote{We address this in the upcoming section \ref{sec:CompositionByBankSize}, when discussing the balance sheet composition.}
The credit crunch in 1990 also leaves different footprints on the banks of different sizes. All sizes beside the small banks experience a downturn after the credit crunch. The small banks, however, experience their downturn years before the credit crunch in 1989. 


\iffalse
From the crisis onwards, the growth behaviour became similar again.  
The clear trend of high growth by the top 10 banks and top $11-100$ are confirming observations made in the section \ref{sec:ToBigToFail}. While the top $11-100$ show a general trend, there is a period of flat almost falling growth for the top 10 banks. The period in question is between $1982-1992$, around $10$ years long. Actually, there are two clear points in time which marked significant changes in the development of the assets per category. From the start of our chosen timeframe $(1976)$ until $~1985$ all the categories showed similar growth behaviour. Just before $1985$ growth of top $10$ assets started to slow and $1985$ marked a period of flat low growth for the banks ranked $1000-\infty$. There could be several reasons for this. We need to take into account that the total amount of banks also fell. Table \ref{table:assetsByPercentiles} shows the year 1984 marked a starting point for a falling number in banks. This falling number of banks could have impacted the bank categories assets growth strongly. 
The second point in time is in year $1992$, assets of the top $101-1000$ stopped growing, while the top $11-100$ and top $10$ started to grow at a high rate again. In the 1990s a lot of regulation reforms occurred, aiding the growth of larger banks. These reforms are mentioned in section \ref{sec:data} and could have been key drivers for the growth of the top 10 in the 1990s. Interestingly, in 2013 they all came back to a low growth rate near zero.
\fi




\subsubsection{Asset cycle similarity between banks of different sizes}
This section describes the similarity between the categories asset cycles. It might convey different balance sheet behaviours by bank size. Table \ref{table:cat_assets_corr} shows the linear correlation between asset cycles over time for each category. As one might expect, all categories positively correlate with the category just below themselves. However, there are significant differences when going beyond that. Category $1$ (Top 10 banks) has a negative correlation with category $3$ (Top $101-1000$) of $-0.27$ and no correlation with category $4$ (Top 1000-Rest). Although the negative correlation of $r=-0.27$ is not strong, this difference in asset cycle timings would probably not be expected. It means that while the top 10 banks might go through a period of decreasing assets, the Top $101-1000$ might go through a period of increasing assets. However, a closer look at the graph indicates that the main driver for this negative correlation could be the period from 1996 to 1999. Indeed, the exclusion of this time-period from the correlation computation reveals an $r=0.05$. Similar to the relationship between top 10 and small banks (cat4), this complete lack of cyclical relationship between large banks and small banks underlines their independence of balance sheet decisions in regards to the other category.\\
We also compute the autocorrelations to take into account different timings. Significant asset changes of the top 10 might not have an immediate effect on the other categories in the same period, but perhaps one quarter later. We go up to ten quarters back to see possible impacts. The associated tables can be found in the Appendix (Figure \ref{autocorrelationsCat14}). An interesting observation can be found for the correlation between category 1 and category 2 (lag 1) one period later. The correlation rises from 0.4 to 0.43 with one quarter lag. Indicating that large banks (cat 2) react slightly delayed to the decisions of the top 10 banks. The rest autocorrelations show no sign of anomalies.



\begin{table}[H]
\begin{minipage}{\textwidth}
\centering
\input{graphs/Tables/LatexTables/cat_assets_cycle_corr.tex}
\caption[1]{Correlation between cyclical assets of each category. The pearson correlation coefficient is used. The stars after the values indicate significance according to standard levels (***: p<0.01; **: p<0.05; *: p<0.1) }
\label{table:cat_assets_corr}
\end{minipage}
\end{table} 

\subsubsection{Balance sheet composition by bank size: Trends}
\label{sec:CompositionByBankSize}

To get an understanding on the balance sheet composition by category and how they differ, Figure \ref{fig:catAssetsShare} and \ref{fig:catLiabShare} show the share of each account for both sides of the balance sheet. Loans continue to be the highest share on the asset side for all categories. Interestingly, all other categories beside the top 10 banks, show an increasing trend for share of loans. But the top 10 banks share of loans fall over our time-frame. Furthermore, only the top 10 are engaging in proper trading with a share of trading assets beginning to rise significantly in 1994.
For the liabilities, deposits are a main source of funding for all categories. However, the share of deposits varies between the categories. Larger banks tend to have a lower deposit share than medium and small banks. The share for the top 10 is between $~60\%-80\%$ and for the top 11-100 at $~60\%$ most of the time. Medium banks have a share that is consistently at $80\%$ and small banks a share of almost $~90\%$. Hence, other forms of finance are relatively low for smaller banks. Our findings confirm the pattern: The larger the bank is, the more alternative ways of financing beside deposits are facilitated.


\begin{figure}[H]
\begin{minipage}{\textwidth}
\includegraphics[scale=0.3]{graphs/DescriptiveStats/BanksByAssetCategory_cat_assets_share_7613.png}
\caption[1]{Share of total assets for each balance sheet account}
\label{fig:catAssetsShare}
\end{minipage}
\end{figure}


\begin{figure}[H]
\begin{minipage}{\textwidth}
\includegraphics[scale=0.3]{graphs/DescriptiveStats/BanksByAssetCategory_cat_liabs_share_7613.png}
\caption[1]{Share of total liabilities for each balance sheet account}
\label{fig:catLiabShare}
\end{minipage}
\end{figure}

\newpage


\subsection{Leverage}

\subsubsection{General}

In this section take a look at the leverage of commercial banks. Leverage is a well known and often used concept for monitoring risk and health of financial institutions. While there are a few definitions of leverage, given the dataset we are working with, the focus is on accounting leverage: $\dfrac{total \: assets}{total \: equity \: capital}$\footnote{Tier 1 capital, as defined in Basel III.}.
 Banks use leverage to improve their return on equity. As long as the interest on external capital does not exceed the total capital ratio, raising external capital, thus increasing leverage, is beneficial for a bank. With this incentive in mind, it might not come as a surprise that when a shareholder asks for a high return, increases in leverage follow. As a result the buffer to cover losses in case investments turn bad is reduced. As a result, increases in leverage can be seen as increases in risk.\\
We took into account that policy makers set capital requirements on banks on their highest organizational level and aggregated all commercial banks to their belonging bank holding company. 
We are also removing all banks with negative equity from the dataset as they can be considered bankrupt.\footnote{Banks with negative equity, do not report financial information in the following periods. Only in the rare case of bailouts, they survive.} We only want to assess leverage behaviour of still operating banks. For more information about bankrupted banks in the dataset see section \ref{sec:AssetLiabs}. 

\subsubsection{Is Leverage pro-cyclical?}
\label{LeverageProcyclical}
When looking at balance sheet leverage, it is important to realize its dynamics in regards to asset cycle movements.
Let us assume, we have a negative asset cycle and the asset values are falling together with the bank experiencing losses. This reduces banks equity. As balance sheet leverage can be written as $\dfrac{equity + liabilities}{equity}$, this would result in increased leverage, assuming liabilities do not change. Hence, when banks do not actively adjust their balance sheet towards asset cycle changes, leverage behaves countercyclical.\\
However, literature agrees that commercial banks tend to behave pro-cyclical in regards to asset changes. \citet{AdrianShin2011} as well as \citet{greenlaw2008leveraged} support this notion. We want to confirm this with our data. But while \citet{AdrianShin2011} use the growth rates of both asset and leverage and compute leverage by aggregating assets and equity of all commercial banks first - essentially computing leverage of the commercial banking market as a whole -, we compute leverage in a different way and use the cycles of both assets and leverage instead of the growth rate.\footnote{We compute the leverage of each individual banks first and then take the average of all individual leverage ratios. We also take a look at a larger time-frame than \citet{AdrianShin2011}. Ours is from $1976-2013$ and \citet{AdrianShin2011} is from $1983-2010$} The major difference is that within the \citet{AdrianShin2011} method the leverage has a weighted impact on the computations depending on a banks size, while within the average method, banks leverage is all weighted the same, independent from bank size. Figure \ref{fig:ScatterCyclicalLeverageAssets} shows the result of this approach for all commercial banks together. Although we can derive that commercial banks actively manage their leverage, no pro-cyclicality can be identified. 
Applying the same leverage computation as \citet{AdrianShin2011} gives use Figure \ref{fig:ScatterCyclicalAggregateLeverageAssets}, showing a positive relationship. 
These distinct results between the two Figures, indicate that within Figure \ref{fig:ScatterCyclicalAggregateLeverageAssets} the large banks drive the pro-cyclicality. While within Figure \ref{fig:ScatterCyclicalLeverageAssets}, where every banks leverage is weighted the same, the much higher quantity of small banks reduces the pro-cyclical impact of large banks on the average.
To reveal more information about leverage pro-cyclicality among commercial banks, we apply the two approaches to different banks sizes. Against the background of our given analysis one would expect to make two observations:
\begin{enumerate}
\item Large banks have pro-cyclic leverage behaviour but small banks not
\item The more homogeneous the banks in one bank size category are, the more similar the results of the two computational methods would be. Our upper two categories (top 10 and large banks), always include a smaller amount of banks. In addition, when considering the standard deviation, as seen in Figure \ref{fig:averageLeverage_Std_Categories} in section \ref{sec:LeverageDevelopment}, the upper categories are much more similar in regards to their leverage than the lower two categories (small and medium). As a result, we might see more similarity between the two computational methods for larger banks. 
\end{enumerate}
Figure \ref{fig:ScatterAggregateCyclicalLeverageAssetsbyCat} and \ref{fig:ScatterAverageCyclicalLeverageAssetsbyCat} give us the scatterplots by bank size. 
Indeed, we can confirm hypothesis two. The top 10 banks (cat 1) and large banks (cat 2) show similar relationships within both Figures, but the relationships identified for medium and small banks differ significantly between the two Figures. While Figure \ref{fig:ScatterAverageCyclicalLeverageAssetsbyCat} finds a correlation coefficient near zero for medium banks and a correlation for small banks of $0.15$, Figure \ref{fig:ScatterAggregateCyclicalLeverageAssetsbyCat} finds stronger positive correlation for small and medium banks ($0.35$ and $0.34$). Hence, we can deduce that the cyclicality in regards to leverage for large banks is robust to both treatments of the data. But not for medium and small banks. 
However, hypothesis one cannot be clearly confirmed. The top 10, with a coefficient of $0.24$, show less pro-cyclicality than large banks, with a coefficient of $0.44$. In addition, because of the different results received for the smaller banks, it is difficult to arrive at a conclusion about their pro-cyclicality. The larger banks within the medium and smaller bank size category might drive the pro-cyclicality seen in Figure \ref{fig:ScatterAggregateCyclicalLeverageAssetsbyCat}. 
One would need to split the smaller banks into additional categories to get a better understanding of their behaviour. 

A conclusion that can be drawn about among all bank sizes are that they actively manage leverage, otherwise we would need to observe a negative relationship as we explain above.
Finally, it is important to note the consequences of the observed pro-cyclicality among certain bank sizes.\footnote{Medium-sized banks and on a smaller level of the top 10 banks show clear pro-cyclicality}. It means that they do not only actively adjust their balance sheets, but they are increasing leverage in good times and decreasing leverage in bad times. They are taking on additional debt to not only balance the usual negative relationship, but to lever their assets even further. This has severe impacts on a countries business cycle. In good times, banks might lever to much and take on to much risk. On a contrary, in bad times, banks play to conservative hindering recovery. 


%This information helps us to also explain the difference between \ref{fig:ScatterCyclicalLeverageAssets} and \ref{fig:ScatterCyclicalAggregateLeverageAssets}. When computing the aggregate leverage, it is the large banks that drive the pro-cyclicality, while within the average leverage, every banks is the weighted the same, so the much higher quantity of small banks reduces the impact of large banks on the average. 

%In addition, we will see in section \ref{sec:LeverageDevelopment}, when considering the standard deviation, larger banks are much more similar in regards to their leverage than small banks. Because of the different results received for the smaller banks, it is difficult to arrive at a conclusion about their pro-cyclicality. One would need to split them into additional categories to get a better understanding of their behaviour.

\iffalse
However, \citet{AdrianShin2010} showed that commercial banks tend to actively manage their balance sheet by trying to keep their leverage constant. \citet{AdrianShin2010} used Flow of Funds data as their source.  Although we use a different way of computing cyclical leverage, Figure \ref{fig:ScatterCyclicalLeverageAssets}, which shows the cyclical average leverage for all commercial banks in regards to cyclical aggregate assets, supports this theory. Many observations show no changes in leverage, when assets change. \citet{AdrianShin2011} on the other hand used a different data source and come to the conclusion that leverage is pro-cyclical. Here they used the same source as our - FDIC call reports. But both times, instead of computing the average leverage, they computed leverage by aggregating assets and equity of all commercial banks first. Essentially computing leverage of the commercial banking market as a whole. Following this approach we created Figure \ref{fig:ScatterCyclicalAggregateLeverageAssets}. We arrive at the same results as \citet{AdrianShin2011}. 
However, as we wondered why we do not have the same results when taking the average leverage, we also consider the relationship by different bank sizes. We know from the previous section \ref{sec:banksByAssetSize}, that different bank sizes can differ in their behaviour. Figure \ref{fig:ScatterAggregateCyclicalLeverageAssetsbyCat} and \ref{fig:ScatterAverageCyclicalLeverageAssetsbyCat} give us scatterplots between cyclical leverage and assets by category, using the aggregate and average leverage decision. We would expect that the more similar the banks in one category are the more the aggregate and average behave similar. We will also see in \ref{sec:LeverageDevelopment} that the standard deviation of leverage is much higher within the smaller categories.
\fi 




\begin{figure}[H]
\begin{minipage}{\textwidth}
\includegraphics[scale=0.6]{graphs/DescriptiveStats/LeverageCylicality_ScatterAverageLeveragevsAggregateAssets_7613.png}
\caption[1]{Cyclical Assets versus Cyclical Leverage (All commercial banks). We compute the leverage for each bank individually and then take the average. With that average, we then compute cyclical growth by applying the HP-Filter. For the assets we also take the average and the compute the cyclical growth.}
\label{fig:ScatterCyclicalLeverageAssets}
\end{minipage}
\end{figure}

\begin{figure}[H]
\begin{minipage}{\textwidth}
\includegraphics[scale=0.6]{graphs/DescriptiveStats/LeverageCylicality_ScatterAggregateLeveragevsAggregateAssets_7613.png}
\caption[1]{Cylical Assets versus Cyclical Leverage (All commercial banks). We compute the leverage of all banks by (aggregate assets / aggregate equity). We then compute cyclical growth of leverage as well as assets with HP-Filter}
\label{fig:ScatterCyclicalAggregateLeverageAssets}
\end{minipage}
\end{figure}


\begin{figure}[H]
\begin{minipage}{\textwidth}
\includegraphics[scale=0.5]{graphs/DescriptiveStats/BanksbyAssetsCategory_ScatterAverageLeveragevsAverageAssetsCat_7613.png}
\caption[1]{Cylical Assets versus Cyclical Leverage by Category.We compute the leverage for each bank individually and then take the average. With that average, we then compute cyclical growth by applying the HP-Filter. For the assets we also take the average and the compute the cyclical growth. (corr: pearson correlation, sig: 2-tailed p-value rounded to third place after comma, Cat1: top 10 banks, cat2: large banks, cat3: medium banks, cat4: small banks}
\label{fig:ScatterAverageCyclicalLeverageAssetsbyCat}
\end{minipage}
\end{figure}

\begin{figure}[H]
\begin{minipage}{\textwidth}
\includegraphics[scale=0.5]{graphs/DescriptiveStats/BanksbyAssetsCategory_ScatterAggregateLeveragevsAggregateAssetsCat_7613.png}
\caption[1]{Scatterplot: Cylical Assets vs Cyclical Leverage by Category. We compute the leverage of bank by category with (aggregate assets / aggregate equity). We then compute cyclical growth of leverage as well as assets with HP-Filter.}
\label{fig:ScatterAggregateCyclicalLeverageAssetsbyCat}
\end{minipage}
\end{figure}

\subsubsection{Did pro-cyclicality change over time?}

While the approach of previous paragraph analyses pro-cyclicality of commercial banks over the whole time-frame, this paragraph splits the time-frame into $10$-year time ranges to observe if their cyclic leverage behaviour might vary over time. Table \ref{tab:corr_agg_range} and \ref{tab:corr_average_range} below show the correlations of cyclical assets with cyclical leverage depending on the two types of computation used for for all commercial banks.\footnote{See the captions within the tables for more detail.} 
In Table \ref{tab:corr_agg_range} we can observe positive correlations for the two more recent $10$-year time periods. The first time period from $1980-1990$ has a positive correlation of $0.24$ but its p-value above $0.05$ is to high to be significant.\footnote{The lower sample size because of only ten years of data, might have an impact on the p-value} These partly positive correlations align with the positive correlation identified for the whole time frame in Figure \ref{fig:ScatterCyclicalLeverageAssets}.
On the other hand, Table \ref{tab:corr_average_range} shows us completely different results. Here we find a significant (p<0.05) positive correlation for the first time-period, but an insignificant positive correlation for the period from $1990-2000$ and a negative correlation for the most recent time period. These opposing correlations are probably the reason for a correlation of almost zero in Figure \ref{fig:ScatterCyclicalLeverageAssets}. It seems like in $2000-2010$ the average commercial bank did not actively manage its leverage, resulting in leverage behaving counter-cyclical.


\begin{table}[H]
\begin{minipage}{\textwidth}
\input{graphs/Tables/LatexTables/corr_agg_range.tex}
\caption[1]{Correlation of cyclical aggregate leverage with cyclical aggregate assets over time. We compute the leverage of all banks by (aggregate assets / aggregate equity). We then compute cyclical growth of leverage as well as assets with HP-Filter. There are N=$36$ observations per time-range. Significance is rounded to two decimal places.}
\label{tab:corr_agg_range}
\end{minipage}
\end{table}


\begin{table}[H]
\begin{minipage}{\textwidth}
\input{graphs/Tables/LatexTables/corr_average_range.tex}
\caption[1]{Correlation of cyclical average leverage with cyclical average assets over time. We compute the leverage for each bank individually and then take the average. With that average, we then compute cyclical growth by applying the HP-Filter. For the assets we also take the average and the compute the cyclical growth. Significance is rounded to two decimal places.}
\label{tab:corr_average_range}
\end{minipage}
\end{table}

To find out which bank sizes drive the above observed behaviour we apply the same methods to each bank size category also. The results can be seen in Table \ref{tab:corr_agg_range_cat} and \ref{tab:corr_average_range_cat}.\footnote{Note, the high number of insignificant values observed makes it difficult to derive conclusions here.}
We divide the discussion of the results by the time periods in question.

\subparagraph{1980-1990}
Both methods give insignificant correlation results for the top 10 and large banks.\footnote{Again, this might be caused by the lower sample size}. The only significant values with $p<0.01$ are in Table \ref{tab:corr_agg_range_cat} for medium and small banks. Hence, the only conclusion that can be drawn, is that aggregate leverage of small and medium banks tends to move pro-cyclical for the time period in question. This is identical to the results from Figure \ref{fig:ScatterAggregateCyclicalLeverageAssetsbyCat} over the time-frame $1976-2013$ in section \ref{LeverageProcyclical} before. 

\subparagraph{1990-2000}
For this time-period, we find that large and small banks behave pro-cyclical. This behaviour is robust towards both computational methods applied. In addition, we find pro-cyclic behaviour for medium banks in Table \ref{tab:corr_agg_range_cat}. This pro-cyclic behaviour of the medium and large banks might have driven the positive correlation for all banks found in Table \ref{tab:corr_agg_range} for $1990-2000$. All other results for this time-frame are insignificant.

\subparagraph{2000-2010}
The most recent time-period shows small pro-cyclical behaviour with correlations of $0.37$ and $0.3$ from the top $10$ banks. The similarity of both findings means its robust to both data treatments. Although other observations for different bank sizes are insignificant. The negative correlation observed in Table \ref{tab:corr_average_range} combined with the insignificant negative observation seen within the medium banks in Table \ref{tab:corr_average_range_cat}, suggests that the medium banks might be driving the negative correlation in Table \ref{tab:corr_agg_range}. 

Overall, we see that pro-cylicality among commercial banks is not always consistent over time. The average bank for instance did behave pro-cyclic from $1980-1990$, but counter-cyclic in recent times from $2000-2010$. 
However, the fact that leverage of the commercial banking sector as whole is pro-cylic, is to some extent consistent over time as seen in Table \ref{tab:corr_agg_range}. 



\begin{table}[H]
\begin{minipage}{\textwidth}
\input{graphs/Tables/LatexTables/corr_agg_range_cat.tex}
\caption[1]{Correlation of cyclical aggregate leverage with cyclical aggregate assets over time. We compute the leverage of all banks by (aggregate assets / aggregate equity). We then compute cyclical growth of leverage as well as assets with HP-Filter. Significance is rounded to three decimal places.}
\label{tab:corr_agg_range_cat}
\end{minipage}
\end{table}


\begin{table}[H]
\begin{minipage}{\textwidth}
\input{graphs/Tables/LatexTables/corr_average_range_cat.tex}
\caption[1]{Correlation of cyclical average leverage with cyclical average assets over time. We compute the leverage for each bank individually and then take the average. With that average, we then compute cyclical growth by applying the HP-Filter. For the assets we also take the average and the compute the cyclical growth.}
\label{tab:corr_average_range_cat}
\end{minipage}
\end{table}




\iffalse
In the following discussion, we will approach the time-series analysis of leverage among commercial banks from a long- and short-term perspective. The long-term discussion analyses trends and cyclical changes  over the whole time-frame, while short-term discussion focuses on the cyclical behaviour in specific periods such as crises.
\fi
\subsubsection{Leverage development}
\label{sec:LeverageDevelopment}

\subparagraph{Long-term discussion} 

Figure \ref{fig:averageLeverage} shows the mean, median and weighted-average leverage for each point in time. We can see a clear impact of high levered banks on the average. Especially, in the periods around 1990 and 2008 when bankruptcy levels are high, there are major mean leverage increases as seen in Figure \ref{fig:averageLeverage}. The median gives us a clearer indication how leverage among healthy banks look. Hence, depending of what type of measure we choose (average or median), we arrive at different observations. The only consistent information conveyed by all measures is a falling trend in leverage from year 1976 to 2013. The median, representing the typical bank, starts with a leverage of $12.5$ in 1976 and falls continuously over the years to $10$ in 2013. The mean also falls from $12.5$ to $10$. It has some short-term fluctuations in 1990 and 2008, which we will elaborate on later. The weighted-average leverage, taking into account the total assets of a bank, starts with a significant higher level of leverage at $18$, but then also falls to a leverage ratio of $10$ in 2013. The idea behind the weighted leverage is that larger banks with more assets have a stronger impact on the overall systemic risk than smaller banks. The significant measurement differences between the common average and weighted leverage, marks the importance of differentiating between asset sizes in leverage analysis. Also, as seen in Figure \ref{fig:assetsByPercentiles}, the small banks (banks ranked from 1000-Rest) dominate the bank landscape in quantity. As a result, the overall leverage average and small banks leverage average (cat 4) are almost identical. 
The first graph in Figure \ref{fig:averageLeverage_Categories} shows the average leverage for each defined bank size category. Here we can also see an overall falling trend in leverage along all categories. This can be attributed to regulatory efforts such as Basel $1,2$ and $3$. In addition, the graph shows an interesting pattern until 1993 - the larger the bank the more it levers. However, after 1993, the pattern seems to disappear. In 2013, the pattern even reverses - the larger the bank the lower the leverage. These observations are closely linked to information gathered in section \ref{sec:ToBigToFail}. If the top 10 banks would have kept their higher leverage, their significant rise in total asset share from 1993 and onwards would have resulted in major leverage increases for the whole banking sector. Hence, regulators adjusted their regulations to target systemically important banks with stronger capital requirements (G-SIB Framework). The top 10 banks, category 1, are affected by these additional capital requirements.

\begin{figure}[H]
\begin{minipage}{\textwidth}
\centering
\includegraphics[scale=0.3]{graphs/NewLeverage_LeverageRatioAllBanks_7613}
\caption[1]{Median and Average leverage for all banks. The weighted-average leverage ratio is calculated by taking into account the asset size for each bank every point in time. Every leverage ratio for each individual bank is only accounted in the weighted-average by its share of assets compared to the total assets of all banks at that point of time.}
\label{fig:averageLeverage}
\end{minipage}
\end{figure}

\iffalse
\begin{figure}[hbtp]
\begin{minipage}{\textwidth}
\centering
\includegraphics[scale=0.3]{graphs/NewLeverage_LeverageRatioAllBanks_noOutlier_7613}
\caption[1]{Median and Average leverage for all banks. Banks with leverage ratios beyond 100 are considered as outlier and removed.}
\label{fig:averageLeverage_noOutlier}
\end{minipage}
\end{figure}
\fi

\begin{figure}[H]
\begin{minipage}{\textwidth}
\centering
\includegraphics[scale=0.3]{graphs/DescriptiveStats/NewLeverage_LeverageCategories_7613}
\caption[1]{Average leverage by category}
\label{fig:averageLeverage_Categories}
\end{minipage}
\end{figure}

\subparagraph{Short-term discussion} 
For short term analysis, we consider the cyclical component and standard deviation of leverage. The standard deviation indicates previously identified two critical points in time - 1990 and 2008. Looking at the first graph in Figure \ref{fig:averageLeverage_cyclical_Categories}, the cyclical graph also highlights those same periods. However, similar to the standard deviation, the spike in average leverage for all banks occurs right after the NBER crisis definition. We know that small banks drive the average leverage with their quantity. Hence, their cyclical components - graph 1 and graph 5 - in Figure \ref{fig:averageLeverage_cyclical_Categories} are almost identical. In comparison to the small banks, the cyclical leverage of the top 10 banks actually spikes during the crisis in 2007-8. The mid-categories two and three show behaviour right between the two extreme behaviours of categories one and four. Category two has small peaks during and after the crisis. Category three only has a peak up to $2$ after the crisis, closer resembling category four. Note, the graph of category three cyclical leverage contains some extreme outliers in year 1992Q4, which increased the limits of the vertical axis up to $20$. To ease analysis, table \ref{table:CyclicalLeverageFinancCrisis} gives us the actual cyclical values of the average leverage for the crisis periods. We marked changes $>4\%$ with red. Similar to the graph, we can see a spill-over effect of high leverage from large to small banks. Figure \ref{fig:AssetsvsLeverageFinancCrisis} gives us a visual insight into the structural changes that occur regarding asset size and leverage. Each data-point represents one bank. We can see a clear increase in dispersion of leverage in 2009 among the small and medium banks. This aligns with the standard deviation shown in Figure \ref{fig:averageLeverage_Std_Categories}, where the standard deviation also has a spike in 2009. It is important to mention that the previous defined pro-cyclicality of leverage stands in contrast to the observations made in the crisis 2008. We have defined pro-cyclicality with positive co-movement of leverage with assets and not GDP. While the GDP might be falling in the crisis 2008, total assets of commercial banks do not immediately behave the same, see section \ref{sec:AssetLiabs}.  
Another pattern we can observe within the standard deviation is an increase of volatility the smaller the asset size category becomes. Figure \ref{fig:averageLeverage_Std_Categories} show this with the ranges of its vertical axes.   

\begin{figure}[H]
\begin{minipage}{\textwidth}
\centering
\includegraphics[scale=0.3]{graphs/DescriptiveStats/NewLeverage_AverageLeverageCyclical_7613.png}
\caption[1]{Cyclical average leverage by category. Category 3 contains a banks with leverage over 10000 in year 1992Q4, which results in this exorbitant high spike.}
\label{fig:averageLeverage_cyclical_Categories}
\end{minipage}
\end{figure}

\begin{figure}[H]
\begin{minipage}{\textwidth}
\centering
\includegraphics[scale=1]{graphs/Tables/CyclicalLeverage_Crisis2008.png}
\caption[1]{Cyclical Average Leverage}
\label{table:CyclicalLeverageFinancCrisis}
\end{minipage}
\end{figure}

\begin{figure}[H]
\begin{minipage}{\textwidth}
\centering
\includegraphics[scale=0.3]{graphs/DescriptiveStats/NewLeverage_ScatterAssetsLeverageFinancCrisis_7613.png}
\caption[1]{Scatterplot: Assets/Leverage. Banks with leverage ratios beyond 50 are considered as outlier and not included. Each data-point represents one bank.}
\label{fig:AssetsvsLeverageFinancCrisis}
\end{minipage}
\end{figure}


\begin{figure}[H]
\begin{minipage}{\textwidth}
\centering
\includegraphics[scale=0.3]{graphs/NewLeverage_stdByCat_7613.png}
\caption[1]{Standard deviation of leverage by category. Category 3 contains a banks with leverage over 10000 in year 1992Q4, which results in this exorbitant high spike.}
\label{fig:averageLeverage_Std_Categories}
\end{minipage}
\end{figure}

\subsubsection{Leverage distribution}
\label{sec:LeverageDistribution}

\subparagraph{Long-term discussion}

In regards to the distribution of leverage, we have plotted the skewness as well as the kurtosis for all banks in Figure \ref{fig:averageLeverage}. Both variables behave similar. There are periods of strong as well as low variation. Notable periods of high variation are the periods around the two banking crises in 1990 and 2008. Since high positive skewness means the graph is right-skewed with the mean being higher than the median and high kurtosis indicates heavy tails, together they prove the existence of high positive outliers. The periods with low variation in turn indicate periods of normal distributed leverage. Furthermore, the two variables only move in the positive direction (values of zero and above). For the skewness, this can be explained by the fact that banks are kind of "sticky" to the lower boundaries of leverage, with not much variation happening within banks of the left tail of the distribution. But there is much more variation happening between banks located at the right tail of the distribution - banks with leverage above the mode. Essentially, high levered banks show much higher variation in their leverage ratio than low levered, conservative banks. The consistent positive kurtosis in turn tells us that there are never less outlier than a normal distribution.\\
Figure \ref{fig:averageLeverage_skew_Categories} and \ref{fig:averageLeverage_kurt_Categories} give us the distribution information by asset category over time. It is important to note that the overall distribution is mainly driven by small banks, because of their sheer quantity. Thus, the division by categories gives us a clearer view. Again, skewness and kurtosis behave very similarly. For the top 10 and top 10-100 (cat2) banks we have short periods where the skewness moves below zero. We take a look at those periods in the short-term discussion. The rest of the time, both measures are either zero or above for all categories, suggesting that once you have a certain amount of banks, the distribution tends to be right skewed. 
In general, we can deduce that most of the distributional changes in our graphs are driven by two factors:
\begin{enumerate}
\item Already high levered banks increasing their leverage even more (high skewness)
\item Increases of outliers (high kurtosis)
\end{enumerate}
These factors seem to be most present around the crisis in 1990 and the crisis in 2008, hence, the significant graph movements around that time-periods.
The top 10 banks have negative kurtosis in some periods, which means the top 10 banks have less outlier than the normal distribution. Hence, the top 10 banks tend to act together regarding their leverage decisions. 
They also show much less distributional volatility around the S\&L crisis, compared to the other categories. Actually, when moving along our categories, the distributional changes are higher the smaller the banks become. This aligns with the arguments made about the standard deviation in the paragraph before.

\begin{figure}[H]
\begin{minipage}{\textwidth}
\centering
\includegraphics[scale=0.3]{graphs/NewLeverage_KurtByCat_7613.png}
\caption[1]{Kurtosis of leverage by category}
\label{fig:averageLeverage_kurt_Categories}
\end{minipage}
\end{figure}

\begin{figure}[H]
\begin{minipage}{\textwidth}
\centering
\includegraphics[scale=0.3]{graphs/NewLeverage_SkewByCat_7613.png}
\caption[1]{Skewness of leverage by category}
\label{fig:averageLeverage_skew_Categories}
\end{minipage}
\end{figure} 

\subparagraph{Short-term discussion} 

As mentioned, in our short-term analysis, we investigate the reason skewness turns negative in some periods. The negative measurements of the top 10 in the crisis 2008 are particularly interesting, since we associate crises with already high levered banks increasing leverage even more. However, this could indicate some low levered banks within the top 10 became high leveraged as well.
The skewness rises from 2008Q1-2008Q3 and then it takes a dive in 2008Q4 and 2009Q1. As a result, the distribution of leverage is left skewed in 2008Q4 and 2009Q1. This left skewness means that the mean is to the left of the peak. To have a better overview, Figure \ref{fig:BoxplotLeverageDatapointsFinancCrisis} combines a boxplot with the top 10 banks leverage ratios marked as dots for the year and quarter in question. The boxplot as whole and the lower whisker moved significantly up from 2007 to 2008. Both are characteristics of left skewness. Here, we can see an overall increase in leverage among top 10 banks, not only driven by outliers. Despite being small in numbers, the asset share of the top 10 was $60\%$ in year 2013. Therefore, this had a significant impact on the bank industry, documented as a spill-over effect in the sections before.\\
The skewness of the top 10 also became negative in the years around 1980. Similar to conclusions we have drawn about the crisis in 2008, this indicates a general increase in leverage among the top 10 banks.  


\begin{figure}[H]
\begin{minipage}{\textwidth}
\centering
\includegraphics[scale=0.3]{graphs/DescriptiveStats/NewLeverage_ScatterAssetsLeverageFinancCrisis_Cat1&2_7613.png}
\caption[1]{Boxplot: Leverage data points. The first row of plots represents top 10 (cat1) and the second row top 10-100}
\label{fig:BoxplotLeverageDatapointsFinancCrisis}
\end{minipage}
\end{figure}

\iffalse
Figure \ref{fig:medianLeverage_cyclical_Categories} illustrates the cyclical component of the median over all banks and our categories. The behaviour of the typical banks are more stable and show consistent asset cycles. Here, it is much more difficult to identify anomalies. 
\fi








\iffalse
\begin{figure}[hbtp]
\begin{minipage}{\textwidth}
\centering
\caption[1]{Cyclical average leverage by category without outlier \footnote{Datapoints above the $0.999$ quantile and below the $0.001$ are removed}}
\includegraphics[scale=0.3]{graphs/DescriptiveStats/NewLeverage_AverageLeverageCyclical_OutlierFilter_7613.png}
\label{fig:averageLeverage_cyclical_noOutlier_Categories}
\end{minipage}
\end{figure}
\fi












\iffalse
\begin{figure}[hbtp]
\begin{minipage}{\textwidth}
\centering
\caption[1]{Scatterplot: Leverage data points \footnote{}}
\includegraphics[scale=0.3]{graphs/DescriptiveStats/NewLeverage_ScatterAssetsLeverageFinancCrisis_Top10_7613.png}
\label{fig:ScatterplotLeverageDatapointsFinancCrisis}
\end{minipage}
\end{figure}
\fi





\iffalse

\begin{figure}[hbtp]
\centering
\caption{Average/Mean leverage plots}
\includegraphics[scale=0.3]{graphs/NewLeverage_LeverageRatioMean_7613}
\end{figure}


\begin{figure}[hbtp]
\centering
\caption{Median leverage plots}
\includegraphics[scale=0.3]{graphs/NewLeverage_LeverageRatioMedian_7613}
\end{figure}



\begin{figure}[hbtp]
\centering
\caption{Trend plots}
\includegraphics[scale=0.3]{graphs/NewLeverage_LeverageRatioTrends_7613.png}
\end{figure}



\begin{figure}[hbtp]
\centering
\caption{Cyclical plots}
\includegraphics[scale=0.3]{graphs/NewLeverage_LeverageRatioCyclical_7613.png}
\end{figure}



\begin{figure}[hbtp]
\centering
\caption{Correlation cyclical leverage}
\includegraphics[scale=1]{graphs/DescriptiveStats/Correlation_Cyclical_Leverage.png}
\end{figure}


\noindent \textit{Key Observations:}
\begin{itemize}
\item Extreme outliers of leverage in year 1992/93 and 2009 lead to spikes in average leverage.
\item Average-weighted leverage significantly higher from 1976 to 2004. Aligns with higher leverage of top $0.1\%$ and $1\%$ banks. Hence, the overall risk from year 1976-2004, was driven high by the larger banks. 
\item Figure 21: All banks median leverage, seasonal effect every year?
\item Leverage lowest in 2007
\item Overall Leverage did fall over time: Introduction of Basel 1 in 1988 might have lead to continuously decrease in leverage
\item Top $0.1\%$ have much higher volatility, which could just be caused by the low sample size.
\item Top $0.1\%$ and $1\%$ actually become less risky than all banks together from 2010 onwards (caused by additional capital requirements)
\item Figure 25 shows slightly negative correlation between the top $0.1\%$ banks and all banks. 
\item Crisis 2008/9: Top $0.1\%$ banks have a average leverage spike in 2008, while top $10\%$, $50\%$ and all  spike around a year later.  
\end{itemize}


\newpage

\begin{figure}[hbtp]
\centering
\caption{Boxplots (1976-2013)}
\includegraphics[scale=0.3]{graphs/Leverage/LeverageDistribution_LeverageRatioBoxplot_7613.png}
\end{figure}

\noindent \textit{Graph description}: Boxplots of all leverage ratios by banks by year. 
\\

\noindent \textit{Key Observations}:
\begin{itemize}
\item $75\%$ of all banks have a leverage ratio between 10-15.
\end{itemize}




\begin{figure}[hbtp]
\centering
\caption{Leverage Top 10 vs Rest over all years}
\includegraphics[scale=0.4]{graphs/Leverage/LeverageDistribution_LeverageRatioTop10vsRest_7813}
\end{figure}



\newpage

\noindent \textit{Graph description}: Since the top 10 banks share of assets did rise up to 60\% in 2013, it is important to differentiate. The graph shows the average leverage (assets/equity) for every year quarter 4.\\ Blue Line: Top 10 Banks by assets\\
Orange Line: All banks beside the top 10\\

\noindent \textit{Key Observations}:
\begin{itemize}
\item Leverage of top 10 banks tends to be higher
\item Trend of falling leverage is similar
\end{itemize}




\begin{figure}[hbtp]
\centering
\caption{Leverage Top10 vs Rest detailed look into crisis}
\includegraphics[scale=0.4]{graphs/Leverage/LeverageDistribution1_LeverageDetailCrisisTop10vsRest_0510}
\end{figure}


\noindent \textit{Key Observations}:
\begin{itemize}
\item Top10 banks leverage peak in year 2008/3 before the rest banks leverage peak in year 2009/4 (theory of risky assets from big banks to small transfer?) 
\end{itemize}









\begin{figure}[hbtp]
\centering
\caption{Distribution 1980-2013}
\includegraphics[scale=0.3]{graphs/Leverage/LeverageDistribution_LeverageRatio_8013}
\end{figure}

\noindent \textit{Graph description}: Counts are normed to 1. Leverage is transformed with log10. Leverage ratios are always from quarter 4.
\\

\noindent \textit{Key Observations}:
\begin{itemize}
\item Log-normal distribution
\item Large standard deviation in year 2010 with 18.82
\item Less and higher bars in 2012 indicate higher homogeneity in 2013 compared to the years before.
\end{itemize}

\pagebreak


\begin{figure}[hbtp]
\centering
\caption{Distribution in crisis 2003-2011}
\includegraphics[scale=0.3]{graphs/Leverage/LeverageDistribution_LeverageRatio_0311}
\end{figure}

\noindent \textit{Graph description}: Leverage ratios are always from quarter 4 and  logarithmised.

\noindent \textit{Key Observations}:
\begin{itemize}
\item Increasing homogenity over time.
\end{itemize}

\fi

\newpage
\section{Evaluation and Outlook}

In general, this thesis gives a broad overview over the U.S. commercial bank landscape and key important factors that should be considered. We find interesting trends and cycles on an aggregate level as well as for different bank sizes. Many points discussed are open for interpretation and future work should go into more detail about the approached topics. For instance, it would be interesting to find an optimal way to categorize U.S. commercial banks. The literature seems to have found no coherent way of categorization. These categories would obviously be of key importance to regulators. Furthermore, as seen from the large amount of literature about leverage there is a wide variety of possibilities to explore it in greater depths. One might consider the relationship of leverage not just with assets, but GDP and other variables. Moreover, commercial banks are just a part of the financial intermediaries existing economies have. The so called "shadow banking" sector does play a major role in today's financial industry and hold a significant share of total assets. It also was investment banks which had to bear the major impacts of the financial crisis in 2007, not commercial banks. Lastly, another factor we have not considered is that, according to \citet{Kalemli-OzcanBentSorensenSevcanYesiltas2011}, a big fraction of assets, especially for large commercial banks, are off balance sheet items. 

\iffalse
Further points
\begin{itemize}
\item Shadow banking/investment banks not considered,significant part of trading assets still owned by non commercial bank. Investment banks do hold a high share of total assets, which were not considered in this analysis. It were investment banks which had to bear the major impacts of the financial crisis in 2007. 
\item  Another factor we have not considered is that according to \citet{Kalemli-OzcanBentSorensenSevcanYesiltas2011} a big fraction of assets , especially for large commercial banks, are off balance sheet items. Especially, large amounts of financial innovation occurring in our time-frame caused them to rise significantly.
\item Valuations not realistic, book values. Loans are recorded at face value, which makes assets value changes less realistic. 
\item Applying more time-series models
\item Applying of models such as Regression...
\item Significant part of trading assets still owned by non commercial banks
\item Correlation between assets and liabilities: Key part of Asset liability management for banks is maturity transformation. For correlation analysis, we should have differed between the different maturities of assets and liabilities. Correlations between positions of different maturity would have a more causal relationship. In addition, canonical correlation analysis could have been used to consider that balance sheet positions are jointly determined by the other positions. Also, the cyclical variation of shares instead of the cyclical variation of the log of absolutes values could have been used.
\item Cyclical of share could have been analysed instead of absolute values (Some literature work with shares)
\item Total assets represent the indicator which regulators and academics use most
frequently for categorising. It measures the gross nominal volume of a bank’s activities, but
suffers from significant valuation problems, not only for derivatives, and it does
not account for differences in individual bank business models or between
financial systems.
\item Leverage: could have looked at how leverage behaves to other balance sheet accounts such as loans to total assets
\item considered different factors that affect leverage and look at them independently
\item Our choice of categorization could have been different. The asset size ranges they cover differ over the years. This can be seen as an advantage or disadvantage. On the one side they evolve over the years and possibly match changing asset size levels. On the other side, there is a risk of distributional changes among the asset sizes of banks, making our chosen categorization unsuitable.  
\end{itemize}
\fi







% ================================================================================
% 							Literaturverzeichnis
% ================================================================================

\newpage
\printbibliography[
heading=bibintoc,
title={Bibliography}
]

\newpage




% ================================================================================
% 								Appendix
% ================================================================================
\appendix
\section{Appendix}

\begin{figure}[hbtp]
\begin{minipage}{\textwidth}

\centering
\includegraphics[scale=1]{graphs/Tables/autocorrelations/AllAssets}
\caption[1]{Autocorrelation - Asset side }
\label{fig:autocorrelation_asset_side}

\end{minipage}
\end{figure}

\begin{figure}[hbtp]
\begin{minipage}{\textwidth}

\centering
\includegraphics[scale=0.8]{graphs/Tables/autocorrelations/AllLiab}
\caption[1]{Autocorrelation - Liabilities side}
\label{fig:autocorrelation_liab_side}

\end{minipage}
\end{figure}




\begin{figure}[hbtp]
\begin{minipage}{\textwidth}
\centering
\includegraphics[scale=0.75]{graphs/Tables/autocorrelations/AssetsCat1}
\includegraphics[scale=0.75]{graphs/Tables/autocorrelations/AssetsCat2}
\includegraphics[scale=0.75]{graphs/Tables/autocorrelations/AssetsCat3}
\includegraphics[scale=0.75]{graphs/Tables/autocorrelations/AssetsCat4}
\caption[1]{Correlations: Category 1-4. This graph shows the correlation of a banks size category assets with the lagged assets of another banks size category. The category after the "+" is the lagged category. Hence, the first graph shows the correlation between categories 1 aggregate assets and all the different other categories lagged aggregate assets.}
\label{autocorrelationsCat14}
\end{minipage}
\end{figure}




\begin{figure}[hbtp]
\begin{minipage}{\textwidth}
\centering
\includegraphics[scale=1]{graphs/MedianBank_FrequencyTable_8010}
\caption[1]{Banks count by asset size. The left column is the asset interval size and the corresponding row the number of banks per year.}
\end{minipage}
\end{figure}

\iffalse
\begin{figure}[hbtp]
\begin{minipage}{\textwidth}
\centering
\caption[1]{Assets vs Leverage \footnote{Banks with leverage ratios beyond 50 are considered as outlier and not included.}}
\includegraphics[scale=0.1]{graphs/DescriptiveStats/NewLeverage_ScatterAssetsLeverage_7613.png}
\label{fig:AssetsvsLeverage}
\end{minipage}
\end{figure}
\fi

\begin{figure}[H]
\begin{minipage}{\textwidth}
\centering
\includegraphics[scale=0.3]{graphs/DescriptiveStats/NewLeverage_MedianLeverageCyclical_7613.png}
\caption[1]{Cyclical median leverage by category}
\label{fig:medianLeverage_cyclical_Categories}
\end{minipage}
\end{figure}

\begin{figure}[hbtp]
\begin{minipage}{\textwidth}
\centering
\includegraphics[scale=0.3]{graphs/NewLeverage_stdByCat_cyclical_7613.png}
\caption[1]{Cyclical standard deviation of leverage by category. Category 3 contains a banks with leverage over 10000 in year 1992Q4, which results in this exorbitant high spike.}
\label{fig:averageLeverage_cyclicalStd_Categories}
\end{minipage}
\end{figure}

\iffalse

\begin{figure}[hbtp]
\begin{minipage}{\textwidth}
\centering
\caption[1]{Cyclical skewness of leverage by category \footnote{}}
\includegraphics[scale=0.3]{graphs/NewLeverage_KurtByCat_cyclical_7613.png}
\label{fig:averageLeverage_kurt_cyclical_Categories}
\end{minipage}
\end{figure}

\begin{figure}[hbtp]
\begin{minipage}{\textwidth}
\centering
\caption[1]{Cyclical kurtosis of leverage by category \footnote{}}
\includegraphics[scale=0.3]{graphs/NewLeverage_SkewByCat_cyclical_7613.png}
\label{fig:averageLeverage_skew_cyclical_Categories}
\end{minipage}
\end{figure}

\fi


\end{document}
