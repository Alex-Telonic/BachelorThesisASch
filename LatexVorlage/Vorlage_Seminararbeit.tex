\documentclass[12pt, a4paper]{article} % Dokumentenklasse

\usepackage[utf8]{inputenc} % Zeichensatz und Schrift
\usepackage[T1]{fontenc}
\usepackage{lmodern} % nice alternative is {mathpazo}
\usepackage[english]{babel}

\usepackage[left=2.5cm, right=2.5cm, top=2.5cm, bottom=2.50cm]{geometry} % Seitenformat

\usepackage{amsmath} % Mathematikzeichen
\usepackage{amsfonts}
\usepackage{amssymb}
\usepackage{mathtools}

\usepackage{booktabs} % Tabellen
\usepackage{array}
\usepackage{dcolumn}
\usepackage{tabularx}
\usepackage{threeparttable}
\usepackage{multirow}

\usepackage{float}

\usepackage{graphicx} % Grafiken
\usepackage{subfigure}

\usepackage{threeparttable} %footnotes for captions

\usepackage{lipsum} % Paket für Random-Text (nur für Anschauungszwecke benötigt, kann gelöscht werden)

\usepackage{setspace} % Zeilenabstand
\onehalfspacing

\usepackage{placeins}

%% Literaturverzeichnis und Zitation % %

\usepackage[%
citestyle=authoryear-comp,%
bibstyle=JME,%
maxbibnames=99,% %maximum number of names printed in bibliography before truncation with ``et al.'' is used
maxcitenames=2,
datezeros=false,% no leading 0 if dates are printed
date=long,%
isbn=false,% show no ISBNs
natbib=true,% enable natbib-compatibility
url=false,% show no urls
backend=biber % use Biber here
]{biblatex}

% vergrößert Abstand zwischen den Quellen im Lit.verzeichnis

\setlength\bibitemsep{1.5\itemsep}

% setzt "et al." bei deutscher Sprache

\DefineBibliographyStrings{ngerman}{%
  andothers = {et\addabbrvspace al\adddot}
}


% vermeidet Witwen und Waisen

\clubpenalty10000 
\widowpenalty10000
\displaywidowpenalty=10000 



% Hier .bib-Datei einfügen

 \bibliography{bspbib.bib}




\usepackage[font=scriptsize, labelfont=bf]{caption}



%---------------------------------------------------------------------------------------
%
%								HIER BEGINNT DAS DOKUMENT
%
%---------------------------------------------------------------------------------------


\begin{document}

% TITELSEITE

\begin{titlepage}
\begin{center}
\includegraphics[scale=0.65]{KITlogo}\\
%\Large
%Karlsruher Institut für Technologie\\
\large
Fakultät für Wirtschaftswissenschaften\\
Institut für Volkswirtschaftslehre (ECON)
\vfill{
%\\\vspace{2cm}
\Large
Bachelor Thesis in Macroeconomics\\\vspace{0.5cm}
,,Study on U.S. commercial banks``\\\vspace{0.5cm}
Winter semester 2019/20% \vspace{2cm}
}
\vfill{
\LARGE
% -------------------------------------------------------------------------------------
% Ihr Thema
Not yet defined
% -------------------------------------------------------------------------------------
\\ 
\vspace{0.5cm}
\normalsize
% -------------------------------------------------------------------------------------
% Nummer Ihres Themas
% -------------------------------------------------------------------------------------
}
\end{center}

\vfill{
\normalsize
% -------------------------------------------------------------------------------------
% Ihre pers\"{o}nliche Daten
\noindent Alexander Schlechter \\
Matr.-Nr. 2054108 \\
alexander.schlechter@student.kit.edu}
% -------------------------------------------------------------------------------------
\end{titlepage}
\newpage

% VERZEICHNISSE

\pagenumbering{Roman}
\tableofcontents
% \listoffigures
% \listoftables
\newpage
\pagenumbering{arabic}


% ================================================================================
% 									Hauptteil
% ================================================================================


\section{Introduction}

This article is an explorative journey through the historical balance sheet filings of U.S. commercial banks. Its objective is to shed light on the financial development of arguably the most important backbone of the U.S. economy - commercial banking. Not least, the severe financial crisis in 2008, which originated from the banking industry, proves the importance of regulating commercial banking. However, only with a deep empirical understanding of the behaviour of commercial banks, one can design regulations that are ultimately effective. We use different perspectives and a variety of approaches to unveil interesting stylized facts about balance sheet trends (long-term) and cycles (short-term) among banks over time. We study the banks in both dimensions - time and cross-section - to enrich those facts. The analysis takes place under careful consideration of contextual information such as crises and regulatory efforts, within our focused time-frame years 1976-2013. The growth and distribution of assets, the relationship between different balance sheet accounts and leverage are some areas of interest this article will elaborate on. We gather interesting insights such as that small banks business cycles are independent to that of large banks.  
This article should be seen as a complement and a way of clarification to the wide variety of existing literature exploring similar themes. Adrian and Shin 2011 for instance investigate the pro-cyclicality of leverage. We used their approaches as a basis to apply them on our data and compared the results. We find that indeed commercial banks do have pro-cyclical leverage, but it differs between bank sizes. Similar to DeYoung and Yom 2008, we also do correlation analysis between different balance sheet accounts, but we do not go in such great depths.
We start of by outlining the data and methods used. Then, we give a more general overview of commercial banks and elaborate on each balance sheet position. A section about the development of distribution of assets follows. We then continue by analysing banks by different asset sizes. Lastly, we examine the important economic indicator leverage for commercial banks.

\iffalse
   ,  We outline trends and business cycles. 

Findings:


Analysis under close consideration of economic business cycles.  

Different perspectives: 
\begin{itemize}
\item banking sector as whole (Aggregated)
\item Banks categories
\end{itemize}

Analysis timeseries and cross section

Long term trend analysis vs short term business cycles analysed

Some questions approached: 
\begin{itemize}
\item How did the balance sheets of commercial banks evolve over time?
\item To what extent are balance sheet positions pro-cyclical, with regards to crisis and trough definitions by the NBER ?
\item Are there relationships between different balance sheet positions on an aggregate level?
\item How did the commercial bank landscape change over time in regards to asset size?
\item How did the balance sheets of commercial banks of different sizes evolve over time?
\item To what extent is leverage pro-cyclical ?
\item Are there differences in leverage behaviour between different banks categorized by asset size?
\end{itemize}

Explain structure of thesis...

\fi
\section{Main part}

\subsection{Data}
\label{sec:data}

The analysis in this thesis is build upon a dataset of balance sheets originally provided by the FFIEC. Also named call reports, the FFIEC collects this balance sheet information quarterly from every FDIC insured institution. \cite{DrechslerSchnabel2017} used these reports and formed a consistent time-series from year 1976Q1 to 2013Q4, accounting for variable and other changes over the years. They only included commercial banks (banks with Charter Type 200).
The accounts reported in those call-reports were recorded at fair value. 
To graph these time-series we create a horizontal axis with a tick for every quarter. We also add a year label for every first quarter. This axis is consistently used throughout the analysis. Bank filings with negative equity are removed from the dataset, since they indicate a bankrupt bank. To prevent skewing the data, the two big investment banks Goldman Sachs and Morgan Stanley becoming commercial banks in the proceedings of the financial crisis 2008 are removed. When looking at leverage, we aggregate all commercial banks to their belonging bank holding companies. For our use-case it is not necessary deflate the data.

In the proceedings of our analysis, we took into account recession definitions provided by the National Bureau of Economic Research. They define a recession not in terms of two consecutive quarters of decline in real GDP, but a significant decline in economic activity spread across the economy, lasting more than a few months, normally visible in real GDP, real income, employment, industrial production, and wholesale-retail sales (\cite{NBERBusinessCycles}). In addition, we differentiated between so called "banking" (originated in the banking sector) vs "market" (originated from outside banking sector) crisis as in \cite{BergerBouwman2013}. We assume here that banking crisis could be reflected stronger in the data.\\
Crisis:
\begin{itemize}
\item 1980Q1-1980Q3 early 1980s recession (market crisis)
\item 1981Q3-1982Q4 early 1980s recession (market crisis)
\item 1990Q3-1991Q2 credit crunch (banking crisis)
\item 2001Q2-2001Q4 dotcom bubble (market crisis)
\item 2007Q4-2009Q3 financial crisis (banking crisis)
\end{itemize}
Those dates will be marked in our graph as gray zones.\\
For further context awareness, it is important to mention following other structural events that affected the us commercial banks landscape considerably:

\begin{itemize}
\item Gramm-Leach-Bliley Act in 1999 - This act repealed part of the Glass-Steagal Act of 1933, removing barriers that prevented banks from offering traditional commercial banks services and investment bank services or insurance company services at the same time.  
\item Reigle-Neil in 1994 - This law removed several obstacles to banks opening branches in other states and provided a uniform set of rules regarding banking in each state.
\item FDIC Improvement Act (FDICIA) passed in 1991, gave the FIDC the responsibility to rescue banks with least-costly method. Aimed to relativize the evolving moral hazard. 
\item Basel 1 in 1988, Basel 2 in 2004, Basel 3 in 2010 - Capital and liquidity regulations to improve banks-sector stability.
\item Over our data time-frame the banking sector experienced a wide-spread adoption of financial innovations. The main ones being interest rate derivatives, asset securitization and adjustable rate mortgages. We will refer back to this in our analysis, when we see these reflected in our data. 
\end{itemize}



\subsection{Methods}

We use a number of methods to aid analysis of banking data over time and in the cross-section.
For most methods we transform the data with the natural logarithm to focus on relative changes. In graphs we will indicate this transformation with proper labels. Furthermore, we apply the recognized Hodrick-Prescott filter with the recommended parameter of 1600 for quarterly time-series to de-trend out data. Seasonal effects will not be removed. The resulting cyclical graphs show absolute business cycle variations of the underlying variable. For correlations and autocorrelations, we use the linear Pearsons' correlation coefficient. To determine significance we compute the 2-tailed p-value. Significance is then determined according to following levels:
\begin{enumerate}
\item ***: <0.01
\item **: <0.05
\item *: <0.1
\end{enumerate}

\iffalse
\begin{itemize}
\item No inflation adjustment: We are only interested in the actual priced value of the banks assets and not the quantity of the assets. Meaning for instance the banks could hold ten assets in 2000 with a value of 100\$ and ten assets in 2013 with a value of 150\$ caused by a rise the overall price level. Although there was not welfare increase as the quantity did not increase, the value the banks hold still increased.
\item Tranformations: Log, to account for relative changes and enable comparisons over time.
\item Timeseries analysis:
\item Detrending with HP Filter (Parameter: 1600)
\item (Deseasonalize with X11 procedure)
\item Correlation/Autocorrelation
\item 2-tailed p-value on correlation with significance levels: ***: <0.01
					**: <0.05
					*: <0.1
\end{itemize}
\fi


\subsection{U.S. commercial banks - Overview}
\label{sec:AssetLiabs}

This section will guide us through the distribution of financial components held by the us commercial banking sector as a whole. We will see what types and amounts of financial instruments banks are holding and how these positions evolved over time. 

\subsubsection{Stylized balance sheet}


Table \ref{tab:balancesheet} shows a common perspective of a us commercial banks' balance sheet.

\iffalse
\begin{itemize}
 \item Cash
 \item Fed funds sold and securities purchased under agreements to resell
 \item Securities
  	\begin{itemize}
 	\item Treasury
 	\item Mortage backed security
 	\item Other
 	\end{itemize}
 \item Loans net\footnote{Loans and leases net of unearned income and allowance for loan and lease losses}
 \item Trading assets
 	\begin{itemize}
 	\item Interest rate derivatives
 	\item Other fixed income
 	\item Other trading
 	\end{itemize}
 \item Other assets \footnote{composed of derivatives "not for trading" and other items}
 \end{itemize}
 
 \fi
 
\begin{figure}[H]
\begin{minipage}{\textwidth}
\centering
\caption[1]{Stylized balance sheet of us commercial bank \footnote{Every position beside the trading assets are hold "not for trading purposes". Meaning for instance the securities position and loans position are not held for trading.}}
 \begin{tabular}{|l|l|}
 \hline
 \textbf{Assets} & \textbf{Liabilities}  \\ \hline \hline
 Cash & Equity \\ \hline
 \begin{tabular}{@{}l}Fed funds sold and securities purchased \\ under agreements to resell (fedfundsrepoassets)\end{tabular} & \begin{tabular}{@{}l} Fed funds bought and securities sold \\ under agreements to repurchase \end{tabular} \\ \hline
 Securities: & Deposits:  \\ 
 \begin{tabular}{l} - Treasury \\ - Mortgage-backed Security (MBS) \\ - Other \end{tabular} & \begin{tabular}{l} - short \\ - other \end{tabular} \\ \cline{1-1}
 Loans net\footnote{Loans and leases net of unearned income and allowance for loan and lease losses} & \\ \hline
  Trading assets: & Trading liabilities  \\ 
\multicolumn{2}{|c|}{\begin{tabular}{l} - net interest rate derivatives  \\  - net other fixed income \\ - net other trading \end{tabular}} \\ \hline
 Other assets\footnote{composed of derivatives "not for trading" and other items} & Other liabilities \\ \hline
 \end{tabular} \\ 
 \label{tab:balancesheet}
 \end{minipage}
\end{figure}


We have simplified the balance sheet of a typical U.S. commercial bank similar to \cite{DrechslerSchnabel2017}. Cash consist out of noninterest-bearing balances with currency and coin included and of interest-bearing balances.
Federal funds sold and securities purchased under agreements to resell are both ways of lending excess cash to fellow commercial banks in return for interest. Fed funds bought and securities sold under agreements to repurchase in turn is the opposite. Securities can be divided into Held-to-maturity and Available-for-sale. These categories then include a large amount of different types of securities, with Treasury and MBS being the largest. Loans are netted by unearned income and allowance for loan and lease losses to gather their existent value. Trading assets or  are securities hold with the intention to sell them as profit. They are mostly only hold for short-term. Trading asset can be in any type of form such as a derivative, MBS or loan. Trading liabilities tend to be in the form of short positions or derivatives. Deposits can be divided into transaction and non-transaction deposits. Time and savings deposits make up non-transaction deposits, while the major part of transaction deposits are demand deposits.
 
\subsubsection{Total assets}
 Figure \ref{fig:assets} gives a general overview how the aggregated total assets held by all banks per year and quarter evolved over time. The value of assets rose from below 2 trillion to above 13 trillions dollars. In comparison, the GDP of U.S. in 2013 was 16.78 trillions. The first graph already indicates a period of flatness in years 1990-1993 and an asset spike in 2008/Q2. The fourth graph within our Figure also emphasizes these two periods with significantly low growth.  To get a better insight into what is occurring in specific periods, we will perform time-series analysis by decomposing the logged data into a trend and cyclical component. The second graph within figure \ref{fig:assets} compares the logged with the absolute total assets. Both are drawn within their own vertical axis. While the growth of absolute assets is more exponential over time, the linear trend of the logged assets, suggests a constant relative growth rate. The third graph of Figure \ref{fig:assets} shows the de-trended total assets, which can also be interpreted as its cyclical movements. The gray areas within this graph indicate crises, as defined in section \ref{sec:data}. The alignment of crises with our commercial banks total assets cyclical is limited. We can see that the impact of the early 1980s recession did not lead to more volatility than other normal periods. The dotcom bubble in 2001 lead to a downward variation of us commercial banks assets away from the trend. In regards to the financial crisis in 2008 we see a huge positive variation with a rapid drop back to the trend. Assuming we have set the most fitting filtering parameters for the HP-Filter, it is interesting to see that the assets did not fall significantly below the trend. The loss was rather caused by an overheated market falling back to normal. However, we need to consider the fact that there might be differences in cyclical behaviour between different banks sizes. Section \ref{sec:banksByAssetSize} addresses this point. Finally, the last graph of Figure \ref{fig:assets} shows the banks default rate per year. The graph aligns with the growth graph just above. In periods with a lot defaults we have a low growth rate. Periods that mark high default rates are the loans and saving crisis in 1990 and the financial crisis in 2008. It is interesting to mention that the loans and savings crisis caused significant higher default rates than the crisis in 2008. This might be related to the fact that in 1990 the number of small banks was considerably higher and small banks were more affected in this crisis. In 1990, $74\%$ of the banks that defaulted were small, while in 2010 the percentage of small banks was only $35\%$. We elaborate on the change in banking landscape in section \ref{sec:ToBigToFail}.
 
  
\begin{figure}[hbtp]
\begin{minipage}{\textwidth}
\centering
\caption[1]{Asset side \footnote{Graph 5 shows an estimation of how many banks have defaulted every year. For instance in year 1989 over $0.6\%$ have defaulted. It is based on the negative equity recorded by banks. Hence, it is not exact and some banks might continue to exist in case of mergers or bailouts. Also, sometimes banks are double counted, if a negative equity does not immediately result in bankruptcy.}}
\includegraphics[scale=0.3]{graphs/DescriptiveStats/OtherAnalysis_clean_AssetDistribution_7613}
\label{fig:assets}
\end{minipage}
\end{figure}

\subsubsection{Cyclical analysis of balance sheet accounts}
Diving into more detailed analysis of the balance sheet positions, Figure \ref{fig:positions} gives us the detrended development of each individual position for both the asset side(left column) and liabilities side(right column) of a balance sheet. As a complement Figure \ref{fig:share_both} shows the share of each position in relation to the total assets. 
The cash position shows a clear spike in 2008, but beside then, the movements show no clear sign of irregularity. With the background of the 2008 crisis it makes sense that banks liquidated some of their assets to increase cash. The rise in cash comes along with a significant fall in securities. Figure \ref{fig:share_both} further supports our thesis that securities are sold to raise cash. Here we can see that while the share of securities fell until 2008, the share of cash rose.
The liability "other borrowed money" also gets to its highest point in 2008, indicating anomalies in a bank financing in crisis.
Trading assets follow the same behaviour as total assets in crisis 2008, but its variation in crisis periods do not significantly differ to other periods. We know from \ref{fig:share_both} that the share of trading assets continuously rose over time. In addition, we can observe a spike of trading assets in the period from 1992-1996.
Equity has its lowest downward variation in 2004. We will see more of equities behaviour in the leverage section, when we analyse it in regards to total assets.
Another interesting observation is that some positions show larger fluctuations in more recent times. The volatility of fedfundsrepoasset increased from 1996 onwards and of foreign deposits from 1992 onwards. 
We also see a contradictory relationship between loans and securities. When securities fall, loans rise and vice versa. Figure \ref{fig:corr_assets} confirms this relationship with a negative correlation coefficient of $r=-0.73$ and high significance according to the p-value. The scatter plot in Figure \ref{fig:scatterplots} illustrates this negative relationship. The two asset categories could be seen as substitutes to each other. With a substantial part of securities being mortage backed securities, this relationship does not come as a surprise. 
There is a small positive relationship between fedfundsrepoassets and trading assets. This could indicate that banks lending out excess federal funds or purchasing repurchase agreements are in such a healthy position to be able to increase trading assets as well. The scatterplot of this relationship in Figure \ref{fig:scatterplots} confirms a possible positive linear relationship. A similar positive relationship can be seen between fedfundsrepoassets and loans. However, the scatterplot in Figure \ref{fig:scatterplots} does not support a clear relationship.
Quite surprising is the slightly negative correlation between domestic deposits and foreign deposits of $r=-0.34$.
In addition, there is a positive correlation of $r=0.37$ between equity and trading assets, indicating that increases in equity leads to increases in trading.\footnote{It is important to note that the mere assumption of a correlation between the two sides of a balance sheet contradicts the Modigliani-Miller-Theorem. The Theorem states the independence of assets by the financing capital structure}
Another observation is the strong positive relationship between loans and foreign deposits of $r=0.59$. (Foreign deposits are deposits made in foreign offices. It is not clear why there is such strong relationship here.)

\ref{Balance sheet composition of aggregate U.S. commercial banks}
Further elaborating on \ref{fig:share_both}, we can see that on the asset side, loans are the main target of investment for commercial banks. Throughout the 37-year time-frame the share of loans always stayed between $50-60\%$. The banks started with a share at $~55\%$ until it rose to just above $60\%$ from 1985 onwards. The credit crunch crisis in 1991 caused a fall of the share back to $~55\%$. This fall continued until 1995. From then on, the share of loans rose back to $~60\%$ until 2008, where it started to fell again. It fell to an all-time low in 2013 with a share of just above $~50\%$. With the confirmed negative correlation between securities and loans, this came along with a rise in securities. 
The development of the cash share is also interesting. Cash continuously fell from a share of just below $20\%$ to a share of below $~5\%$. Here the crisis 2008 also marked a turning point with share rising back to above $10\%$ again. 
On the liability side, as one would expect, we have deposits as a dominating source of funding for commercial banks. The share started in 1976 with $~70\%$ and fell until 2008 to an all time low of just above $~50\%$. From there it went back to roughly $~65\%$. This decrease in deposits, especially until 2008, must obviously come along with the increases of other types of finance. There is a significant increase of other borrowed money, peaking in 2008. Other borrowed money consists of Federal Home Loan Bank advances (FHLB) and other borrowings not clearly defined. After the crisis in 2008 there was a rapid decreases of other borrowings. FHLB advances are mainly used in funding low mortgages for low income households, which explains the alignment with the housing crisis in 2008.  Lastly Figure \ref{fig:share_both} shows a general increase in the share of equity commercial banks hold from just above $~5\%$ to above $~10\%$. 
 

\iffalse
- general more negative correlation as expected. Stealing shares
- foreign deposits and other borrowed money positive correlation. Makes sense, because other borrowed money is mainly foreign as well.
- positive relationship between equity and trading assets. More equity, more risk?
- positive relationship between trading assets and fedfundsrepoliab. Repo financing used to buy trading assets
- positive relationship between fedfundsrepoasset and trading liabilties 
\fi

\begin{figure}[hbtp]
\begin{minipage}{\textwidth}

\centering
\caption[1]{Detrended asset positions(left column) \hspace{2cm} Detrended liability positions(right column) \footnote{Data is aggregated in the cross section over all banks, transformed with natural logarithm and detrended with HP-Filter. See details in the data section. Trading assets and liabilities have missing data in the beginning of the time period.} }
\includegraphics[scale=0.3]{graphs/DescriptiveStats/OtherAnalysis_clean_PositionsCyclical_7613.png}
\label{fig:positions}

\end{minipage}
\end{figure}


\begin{figure}[hbtp]
\begin{minipage}{\textwidth}

\centering
\caption[1]{Correlation assets\footnote{Pearsons correlation coefficient based on the detrended data used in Figure \ref{fig:positions}.} }
\includegraphics[scale=0.8]{graphs/Tables/asset_cycle_corr_sig}
\label{fig:corr_assets}

\end{minipage}
\end{figure}

\begin{figure}[hbtp]
\begin{minipage}{\textwidth}

\centering
\caption[1]{Correlation liabilities\footnote{Pearsons correlation coefficient based on the detrended data used in Figure \ref{fig:positions}.} }
\includegraphics[scale=0.8]{graphs/Tables/liab_cycle_corr_sig}
\label{fig:corr_liab}

\end{minipage}
\end{figure}



\begin{figure}[hbtp]
\begin{minipage}{\textwidth}

\centering
\caption[1]{Correlation assets with liabilities\footnote{Pearsons correlation coefficient based on the detrended data used in Figure \ref{fig:positions}.} }
\includegraphics[scale=0.8]{graphs/Tables/comb_cycle_corr_sig}
\label{fig:corr_comb}

\end{minipage}
\end{figure}


\begin{figure}[hbtp]
\begin{minipage}{\textwidth}

\centering
\caption[1]{Scatterplot for selected positions\footnote{Linear regression based on the detrended data used in Figure \ref{fig:positions}.} }
\includegraphics[scale=0.5]{graphs/DescriptiveStats/OtherAnalysis_clean_scatterplots_7613}
\label{fig:scatterplots}

\end{minipage}
\end{figure}


\begin{figure}[hbtp]
\begin{minipage}{\textwidth}

\centering
\caption[1]{Share of balance sheet positions \footnote{The second/fourth graph is a focus of the first/third, just without loans/deposits position.} }
\includegraphics[scale=0.3]{graphs/DescriptiveStats/OtherAnalysis_clean_sharePositions_7613.png}
\label{fig:share_both}

\end{minipage}
\end{figure}



\begin{figure}[hbtp]
\begin{minipage}{\textwidth}

\centering
\caption[1]{Autocorrelation - Asset side \footnote{} }
\includegraphics[scale=1]{graphs/Tables/autocorrelations/AllAssets}
\label{fig:share_both}

\end{minipage}
\end{figure}

\begin{figure}[hbtp]
\begin{minipage}{\textwidth}

\centering
\caption[1]{Autocorrelation - Liabilities side \footnote{} }
\includegraphics[scale=0.8]{graphs/Tables/autocorrelations/AllLiab}
\label{fig:share_both}

\end{minipage}
\end{figure}


\iffalse
\begin{figure}[hbtp]
\centering
\caption{Share of asset positions - unstacked}
\includegraphics[scale=0.3]{graphs/DescriptiveStats/OtherAnalysis_ShareofAssetsPlot_7613.png}
\label{fig:sharepositions}
\end{figure}


\noindent \textit{Graph description}: Figure 1 shows the aggregates of the main variables from the asset side of the balance sheet over time. Figure 2 shows the share of each aggregated balance sheet position of all commercial banks over time. Figure 3 plots the share of each balance sheet position unstacked.\\


\noindent \textit{Key Observations:}
\begin{itemize}
\item loans make up the largest share of assets
\item share of trading assets have risen as well as interest rate derivatives
\item loans and trading assets have risen more than securities in timeframe year 2000-2009
\item Share of trading assets peaked in 2008 while securities fell.
\item There is a noticeable anomaly in year 2002. Significant amounts of repo lending is transferred into other assets. Other assets are derivatives not available for sale. 
\item drop in assets in 2002 and 2008
\item share in cash has fallen until 2008 and then increased again
\item Share of cash continously fell until 2008, and then it increased significantly again
\end{itemize}
\fi
\newpage

\iffalse

\begin{figure}[hbtp]
\centering
\caption{Banks default}
\includegraphics[scale=0.3]{graphs/DescriptiveStats/NewLeverage_BanksDefaultPercentage_7613.png}
\end{figure}

\noindent \textit{Graph description}: The graph shows an estimation of how many banks have defaulted at a certain time (year,quarter). It is based on the negative equity recorded by banks. Hence, it is not exact and some banks might continue to exist in case of mergers or bailouts. Also sometimes banks are double counted, if a negative equity does not immediately result in bankruptcy. \\
\noindent \textit{Key Observations:}
\begin{itemize}
\item main defaults in years 1986-1991 and 2009-2011
\item long stable period from 1991-2008
\item In 1990 there were many more smaller banks. Smaller banks might have a higher likelihood to fail. In 1990:  $~74\%$ small banks, 2010: $~35\%$ small banks 
\end{itemize}

\textbf{To-Do: Bank Defaults within bank categories}

\fi

\iffalse


\textbf{Not finished, more coming}

\begin{figure}[hbtp]
\centering
\caption{Loans}
\includegraphics[scale=0.3]{graphs/DescriptiveStats/OtherAnalysis_LoanDistribution_7613}
\end{figure}

\noindent \textit{Graph description}: It shows the share of loan types of total loans over time.\\

\noindent \textit{Key Observations:}
\begin{itemize}
\item real estate loans has largest share

\end{itemize}

\begin{figure}[hbtp]
\centering
\caption{Loans by repricing maturity}
\includegraphics[scale=0.3]{graphs/DescriptiveStats/OtherAnalysis_ShareofMaturityLoans_9713}
\end{figure}


\begin{figure}[hbtp]
\centering
\caption{Residential Loans by repricing maturity}
\includegraphics[scale=0.3]{graphs/DescriptiveStats/OtherAnalysis_ShareofMaturityResLoans_9713}
\end{figure}

\fi

\newpage

\subsection{Too Big to Fail - Distribution of total assets among banks}
\label{sec:ToBigToFail}

This section tries to empirically illustrate a problematic commonly referred to as \textit{"Too Big to Fail}". 
Banks are considered as "too big to fail", when they have a large size and are interconnected with other banks in a way that its individual risk impacts the systemic risk of a whole economy. The term first came into play with the failure and bailout of Continental Illinois National Bank and Trust Company in 1984. From that point onwards, it developed into a world-wide phenomenon with its severe consequences unveiled in the financial crisis of 2008.
And indeed, over the last few centuries the number of banks on the US landscape fell significantly from 14419 banks in 1976 to 6035 banks in year 2013. While the mere reduction would not impose such a problem, the distribution of total assets developed more and more unequal. In 1976, the top $0.1\%$ a total of 14 banks held $~32.4\%$ of all assets. In comparison, in 2013 the top $0.1\%$ a total of 6 banks held $50\%$ of all assets. Table \ref{table:assetsByPercentiles} and Figure \ref{fig:assetsByPercentiles} show these numbers by looking at the assets distribution by banks percentiles. In addition, the unequal distribution of assets can also been seen in Figure \ref{fig:lorenz} The curved lines show the Lorenz curve per year as referenced in the legend. The more curved the lines become, the more unequal is the distribution. The horizontal line represents perfect equality. Although in 1980 unequal distribution was high already, it increased even more. In year 2013, the top $5\%$ held almost $90\%$ of all assets. Figure \ref{fig:gini} shows us the gini coefficient over time. Its range is from zero to one. A value of one means one bank owns everything, while a value of zero indicates perfect equality ($10\%$ of banks own $10\%$ of assets, $50\%$ of banks own $50\%$ of assets and so on...). The higher the value, the higher the inequality in asset distribution. The trend supports our observation of rising inequality. An interesting observation here is the impact of crisis on the asset distribution. Crisis tend to reduce the inequality and act as redistribution. Assuming that assets values fall in times of crisis, the impact of crisis must be higher on larger banks. We will look into how different banks size categories are impacted differently in section \ref{sec:banksByAssetSize}.
The exact causes of the structural changes in the US banks landscape are not clear. However, geographic deregulation and other regulation reforms such as the repeal of the Glass-Steagall act in 1999 did support the increasing inequality. In addition, larger banks are more likely to be bailed out. This puts them in an easier position to finance themselves. The perverse consequence is a moral hazard. A bank that with high likelihood to be bailed out takes on to much risk (Farhi \& Tirole, 2012) . The severe consequences of this are clear since the financial crisis in 2008. Authorities responded to this issue by setting additional capital requirements on larger banks. 

\begin{figure}[hbtp]
\centering
\caption{Count of banks by percentiles}
\includegraphics[scale=0.8]{graphs/DescriptiveStats/BankCounts}
\label{table:assetsByPercentiles}
\end{figure}


\begin{figure}[hbtp]
\centering
\caption{Aggregate assets by percentiles}
\includegraphics[scale=0.3]{graphs/DescriptiveStats/OtherAnalysis_ShareofAssetsByBanksPercentiles_7613.png}
\label{fig:assetsByPercentiles}
\end{figure}


\begin{figure}[hbtp]
\begin{minipage}{\textwidth}
\centering
\caption[1]{Lorenz Curve\footnote{Always Quarter 1} }
\includegraphics[scale=0.3]{graphs/DescriptiveStats/OtherAnalysis_clean_LorenzCurve_7613.png}
\label{fig:lorenz}
\end{minipage}
\end{figure}


\begin{figure}[hbtp]
\begin{minipage}{\textwidth}
\centering
\caption[1]{Gini coefficient\footnote{} }
\includegraphics[scale=0.3]{graphs/DescriptiveStats/OtherAnalysis_clean_gini_7613.png}
\label{fig:gini}
\end{minipage}
\end{figure}




\subsection{Banks by total assets}
\label{sec:banksByAssetSize}

\subsubsection{General}
In this section we allocate banks into different categories ranked by asset size to find differences in behaviours and impacts of business cycles on each category over time. We have four categories following the convention of the Federal Reserve Bulletin\footnote{https://www.federalreserve.gov/pubs/bulletin/2000/0600lead.pdf}:

\begin{itemize}
 \item 10 largest banks
 \item large banks (those ranked
  11 through 100)
 \item medium-sized banks (those ranked 101 through
 1,000)
 \item small banks (those ranked 1,001 and higher)
 \end{itemize} 

To get an overview of what asset sizes each category covers figure \ref{fig:cat_boxplot} contains boxplots for each category and year. Within all categories we can see a consistent rise in asset sizes, with all banks per category benefiting.
The ten largest banks start of with every banks asset size below than a quarter of a trillion assets in 1976. In 2013, the median asset size was $~0.32$ trillion with banks going up to an asset size of just under two trillion. Note, we have not combined commercial banks with their matching bank holding company. Bank Holding Companies have asset sizes beyond two trillion. For the top 10 banks, we can see a clear rise in heterogeneity over time regarding their asset sizes. The Interquartile Range (IQR) get to its largest size until the end of the time-frame.
Large banks began with an asset size way below $0.25*10^11$ year 1976 and worked their way up with asset sizes up to $1.75*10^11$ dollar in year 2013. The heterogeneity of large banks regarding asset size also increased over time. 
Medium banks ranges between $~0.25-8$ billion dollar assets per bank and small banks between $~0.25-5$ hundred million dollars over our time-frame. As with the aggregate assets analysed below, the top two categories benefit more from the asset size increases. In comparison to the asset increases within the top 10, the typical small bank did not show any significant gains over time. The fact that the chosen categories do not show so many outliers strengthens our choice of categories. Only the small banks category has a decent amount of outliers with an asset size below the median. 


\subsubsection{Trend and cyclical behaviour of total assets by category}
Figure \ref{fig:assetsbycat} shows us the development of aggregate assets by the defined banks categories over time. 
There a key points in time for each category that marked changes in their asset growth. From the start of our chosen timeframe $(1976)$ until $~1985$ all the categories showed similar growth behaviour. Then, in year $1984$, growth of the top $10$ assets started to slow. Shortly after that, year $1985$ marked a starting point of flat, low growth for the banks ranked $1000-\infty$. These small banks did not recover from this low growth until the end of our chosen timeframe. An obvious reason for this could be the fact that the  total number of banks also fell. Table \ref{table:assetsByPercentiles} shows the year 1984 marked a starting point for a continuous fall in the number of banks. Category two and three, covering the banks ranked from $11-1000$, showed similar behaviour until $~1992$. From this point in time, the banks ranked $101-1000$ entered a period of low and negative growth, while the banks ranked from $11-100$ entered, together with the top 10 banks, a period of high growth. In the 1990s, a lot of regulation reforms occurred, aiding the growth of larger banks. These reforms are mentioned in section \ref{sec:data} and could have been key drivers for the growth of larger banks in the 1990s.
The growth of banks ranked from $11-100$ then also fell back to low growth in year $2001$. The top 10 banks assets, however, kept growing until the financial crisis in year 2008. 
\iffalse
From the crisis onwards, the growth behaviour became similar again.  
The clear trend of high growth by the top 10 banks and top $11-100$ are confirming observations made in the section \ref{sec:ToBigToFail}. While the top $11-100$ show a general trend, there is a period of flat almost falling growth for the top 10 banks. The period in question is between $1982-1992$, around $10$ years long. Actually, there are two clear points in time which marked significant changes in the development of the assets per category. From the start of our chosen timeframe $(1976)$ until $~1985$ all the categories showed similar growth behaviour. Just before $1985$ growth of top $10$ assets started to slow and $1985$ marked a period of flat low growth for the banks ranked $1000-\infty$. There could be several reasons for this. We need to take into account that the total amount of banks also fell. Table \ref{table:assetsByPercentiles} shows the year 1984 marked a starting point for a falling number in banks. This falling number of banks could have impacted the bank categories assets growth strongly. 
The second point in time is in year $1992$, assets of the top $101-1000$ stopped growing, while the top $11-100$ and top $10$ started to grow at a high rate again. In the 1990s a lot of regulation reforms occurred, aiding the growth of larger banks. These reforms are mentioned in section \ref{sec:data} and could have been key drivers for the growth of the top 10 in the 1990s. Interestingly, in 2013 they all came back to a low growth rate near zero.
\fi
We are also trying to see how closely the banks in the different categories resemble each other. Following section will look at how similar the business cycles of the banks are by looking at de-trended assets movements. 
In Figure \ref{fig:catassetscycle} we can see de-trended assets values by each category over time. All bank categories were more or less affected by the crisis in 2008. For each category we see a spike, followed by a fall in assets. The top 10 banks were affected the most. However, they also have higher volatility overall. The figure also shows the standard deviation of aggregate assets by category. The larger the category the larger the standard deviation and business cycles.

\subsubsection{Correlation between asset categories}
In addition, table \ref{table:cat_assets_corr} shows the linear correlation between assets development over time for each category. As one might expect, all categories are positive correlated with the category just below themselves. However, there are significant differences when going beyond that. Category $1$ (Top 10 banks) has a negative correlation with category $3$ (Top $101-1000$) of $-0.27$ and no correlation with category $4$ (Top 1000-rest). Although a negative correlation of $-0.27$ is not strong, this difference in business cycle timings would probably not be expected. 
This would mean that while the top 10 banks might go through a period of decreasing assets, the Top $101-1000$ might go through a period of increasing assets. However, a closer look at the graph indicates that the main drive for this negative correlation could be the period from 1996 to 1999. Indeed, excluding this time-period from the correlation computation reveals an $r=0.05$. This correlation in turn indicates no connection at all between the business cycle timings. Similar to the relationship between top 10 and small banks (cat4). This complete lack of cyclical relationship between large banks and small banks underlines their independence of balance sheet decisions in regards to the other category. 
An example of that can be seen in Figure \ref{fig:catassetscycle} in year 1996. 
We are also considering the autocorrelation to take into account different timings. Significant asset changes of the top 10 might not have a immediate effect on the other categories in the same period, but perhaps one quarter later. We go up to ten quarters back to see possible impacts. The associated tables can be found in the Appendix (Figure \ref{autocorrelationsCat14}). An interesting observation can be found for the correlation between category 1 and category 2 (lag 1) one period later. The correlation did rise from 0.4 to 0.43 with an one quarter lag. Indicating that large banks (cat 2) react slightly delayed to the decisions of the top 10 banks. The rest autocorrelations show no sign of anomalies. 

\subsubsection{Balance sheet composition by category}

To get an understanding how the balance sheets differ between the categories. Figure \ref{fig:catAssetsShare}, \ref{fig:catLiabShare} show the share of each position, as outlined in \ref{sec:AssetLiabs}, for the assets as well as the liabilities. Loans continue to be the highest share on the asset side for all categories. Interestingly, all other categories beside the top 10 banks, show an increasing trend for share of loans. But the top 10 banks share of loans fell over our time-frame.  We can also see only the top 10 are the only ones engaging in proper trading with a share of trading assets starting to rise significant in year 1994.
For the liabilities, deposits are a main source of funding for all categories. However, the share of deposits varies between the categories. Larger banks tend to have a lower deposit share than medium and small banks. The share for the top 10 is between $~60\%-80\%$ and for the top 11-100 at $~60\%$ most of the time. But for medium banks the share is consistently at $80\%$ and for small banks almost $~90\%$. Hence, other forms of finance are relatively low for smaller banks. We can see the pattern that the larger the bank is, the more alternative ways of financing beside deposits are facilitated.

\begin{figure}[hbtp]
\begin{minipage}{\textwidth}
\centering
\caption[1]{Boxplots for each category \footnote{The scientific "leX" notation on the vertical axis indicate the $10^x$. All numbers are in thousands. Coloured boxes cover the mid $50\%$ of asset sizes - IQR:25th Percentile to 75th Percentile. For the top 10, all individual datapoints are marked as dots. For the rest, only outliers are marked as dots. Outliers are marked above 1.5 times IQR.} }
\includegraphics[scale=0.3]{graphs/DescriptiveStats/BanksByAssetCategory_boxplots_eachCategory_7613.png}
\label{fig:cat_boxplot}
\end{minipage}
\end{figure}


\begin{figure}[hbtp]
\begin{minipage}{\textwidth}
\centering
\caption[1]{Total assets by bank category \footnote{} }
\includegraphics[scale=0.3]{graphs/DescriptiveStats/BanksByAssetCategory_assetsbycat_7613.png}
\label{fig:assetsbycat}
\end{minipage}
\end{figure}



\begin{figure}[hbtp]
\begin{minipage}{\textwidth}
\centering
\caption[1]{De-trended assets by bank category \footnote{} }
\includegraphics[scale=0.5]{graphs/DescriptiveStats/BanksByAssetCategory_catAssetsCycle_7613.png}
\label{fig:catassetscycle}
\end{minipage}
\end{figure}


\begin{figure}[hbtp]
\begin{minipage}{\textwidth}
\centering
\caption[1]{Correlation between assets of each category \footnote{Pearsons Correlation Coefficient} }
\includegraphics[scale=1]{graphs/Tables/cat_asset_cycle_corr}s
\label{table:cat_assets_corr}
\end{minipage}
\end{figure}


\begin{figure}[hbtp]
\begin{minipage}{\textwidth}
\centering
\caption[1]{Share of total assets for each balance sheet account \footnote{} }
\includegraphics[scale=0.3]{graphs/DescriptiveStats/BanksByAssetCategory_cat_assets_share_7613.png}
\label{fig:catAssetsShare}
\end{minipage}
\end{figure}


\begin{figure}[hbtp]
\begin{minipage}{\textwidth}
\centering
\caption[1]{Share of total liabilities for each balance sheet account \footnote{} }
\includegraphics[scale=0.3]{graphs/DescriptiveStats/BanksByAssetCategory_cat_liabs_share_7613.png}
\label{fig:catLiabShare}
\end{minipage}
\end{figure}

\newpage


\subsection{Leverage}

\subsubsection{General}

In this section we are going to take a look at the leverage of commercial banks. Leverage is a well known and often used concept for monitoring risk and health of financial institutions. While there are a few definitions of leverage, given the dataset we are working with, focus will be on accounting leverage: Total assets divided by total equity capital\footnote{Tier 1 capital, as defined in Basel III.}. Leverage is used by banks to improve their return on equity. As long as the interest on external capital does not exceed total capital ratio, raising external capital, thus increasing leverage, is beneficial for a bank. With this incentive in mind, it might not come as a surprise that when shareholder ask for a high return, increases in leverage follow. As a result the buffer to cover losses in case investments turn bad is then reduced. Therefore, increases in leverage can be seen as increases in risk.
We took into account that policy makers set capital requirements on banks on their highest organizational level and aggregated all commercial banks to their belonging bank holding company. 
We are also removing all banks with negative equity from the dataset as they can be considered bankrupt.\footnote{Banks with negative equity, do not report financial information in the following periods. Only in the rare case of bailouts, they survive.} We only want to assess leverage behaviour of still operating banks. For more information about bankrupted banks in the dataset see section \ref{sec:AssetLiabs}. 

When looking at balance sheet leverage, it is important to realize its dynamics in regards to business cycle movements.
Lets assume, we have a negative business cycle and asset values are falling together with the bank experiencing losses. This reduces banks equity. As balance sheet leverage can be written as $(equity + liabilities)/equity$, this would result in increased leverage, assuming liabilities do not change. Hence, when banks do not actively adjust their balance sheet towards business cycle changes, leverage behaves countercyclical. However, \cite{AdrianShin2010} showed that commercial banks tend to actively manage their balance sheet by trying to keep their leverage constant. Figure \ref{fig:ScatterCyclicalLeverageAssets}, showing the growth of average leverage for all commercial banks in regards to aggregate assets growth, supports this theory. Many observations show no changes in leverage, when assets change. However, \cite{AdrianShin2011} use different data and come to the conclusion that leverage is pro-cyclical. Instead of computing the average leverage, they computed the aggregate leverage of all commercial banks. Essentially computing leverage of the the commercial banking market as a whole. Instead of following this approach, we also consider the relationship of average leverage and assets by bank size category. We know from previous sections, that different bank sizes do differ in the behaviour. Figure \ref{fig:ScatterCyclicalLeverageAssetsbyCat} gives us scatterplots between asset growth and leverage growth by category. Indeed, positive relationships (pro-cyclicality) can be identified within all bank size categories. 
Now that we have established this positive relationship, it is important to recall what this tell us. We know that with passive behaviour by banks, leverage should be counter-cyclical. The identified pro-cyclicality now, not only indicates that banks actively adjust their balance sheet, but they are increasing leverage in good times and decreasing leverage in bad times. Banks are taking on additional debt to not just balance the normal negative relationship, but lever their assets even further.


\begin{figure}[H]
\begin{minipage}{\textwidth}
\centering
\caption[1]{Scatterplot: Cylical Assets vs Cyclical Leverage (All commercial banks) \footnote{We compute the leverage for each bank individually and then take the average. With that average, we then compute cyclical growth by applying the HP-Filter. We aggregate all assets per year and quarter and then compute cyclical growth.}}
\includegraphics[scale=0.7]{graphs/DescriptiveStats/LeverageCylicality_ScatterAverageLeveragevsAggregateAssets_7613.png}
\label{fig:ScatterCyclicalLeverageAssets}
\end{minipage}
\end{figure}

\begin{figure}[H]
\begin{minipage}{\textwidth}
\centering
\caption[1]{Scatterplot: Cylical Assets vs Cyclical Leverage by Category \footnote{We compute the leverage for each bank individually and then take the average. With that average, we then compute cyclical growth by applying the HP-Filter.}}
\includegraphics[scale=0.5]{graphs/DescriptiveStats/BanksbyAssetsCategory_ScatterAverageLeveragevsAggregateAssetsCat_7613.png}
\label{fig:ScatterCyclicalLeverageAssetsbyCat}
\end{minipage}
\end{figure}

In the following discussion, we will approach the time-series analysis of leverage among commercial banks from a long- and short-term perspective. The long-term discussion analyses trends and cyclical changes  over the whole time-frame, while short-term discussion focuses on the cyclical behaviour in specific periods such as crises.

\subsubsection{Leverage development}

\subparagraph{Long-term discussion} 

Figure \ref{fig:averageLeverage} shows the mean, median and weighted-average leverage for each point in time. We can see the clear impact of high leverage banks on the average. Especially, in the periods around 1990 and 2008 where bankruptcy levels are high, there are major leverage increases in Figure \ref{fig:averageLeverage} on the mean leverage. Looking at the median gives us a clearer indication how leverage among healthy banks look. Hence, depending of what type of measure we choose (average, median...), we arrive at different observations. The only consistent information conveyed by all measures is a falling trend in leverage from year 1976 to 2013. The median, representing the typical bank, started with a leverage of $~12.5$ in 1976 and fell continuously over the years to $~10$ in 2013. The mean also fell from $~12.5$ to $~10$. It had some short-term fluctuations around the crisis in 1990 and 2008, which we will elaborate on later. The weighted-average leverage, which takes into account the asset size, started of with a significant higher leverage level of $~18$, but then also fell to a leverage of $~10$ in 2013. The idea behind the weighted leverage is that larger banks with more assets have a stronger impact on the overall systemic risk than smaller banks. The significant measurement differences between the common average and weighted leverage, marks the importance of differentiating between asset sizes in leverage analysis. Also, as seen in Figure \ref{fig:assetsByPercentiles} the small banks (banks ranked from 1000-Rest) dominate the bank landscape in quantity. As a result, the overall leverage average and small banks leverage average (cat 4) are almost identical. 
The first graph in Figure \ref{fig:averageLeverage_Categories} shows the average leverage for each defined bank size category. Here we can also see an overall falling trend in leverage along all categories. This can be attributed to regulatory efforts such as Basel $1,2$ and $3$. In addition, the graph shows an interesting pattern until year 1993 - the larger the bank is the more leverage it takes. However, after year 1993, the pattern seems to disappear. In 2013, the pattern even reversed - the larger the bank the lower the leverage. These observations are closely linked to information gathered in section \ref{sec:ToBigToFail}. If the top 10 banks would have kept their higher leverage, their significant rise in total asset share from year 1993 and onwards would have resulted in major leverage increases for the whole bank sector. Hence, regulators adjusted their regulations to target systemically important banks with even stronger capital requirements (G-SIB Framework). The top 10 banks, our category 1, are affected by these additional capital requirements.

\subparagraph{Short-term discussion} 

For short term analysis, we are considering the cyclical component and standard deviation of leverage. The standard deviation already indicates the already mentioned two critical periods - the crisis in 1990 and 2008. Looking at the first graph in Figure \ref{fig:averageLeverage_cyclical_Categories}, the cyclical graph marks those same periods. However, similar to the standard deviation, the spike in average leverage for all banks occured right after the NBER crisis definition. We know that small banks drive the average leverage with their quantity. Hence, their cyclical components - graph 1 and graph 5 - in figure \ref{fig:averageLeverage_cyclical_Categories} are almost identical. In comparison to the small banks, the cyclical leverage of the top 10 banks actually spikes in the crisis in 2007-8, indicating top 10 banks counter-cyclical behaviour. The mid-categories two and three show behaviour right between the two extreme behaviours of categories one and four. Category two has small peaks during and after the crisis. Category three only has a peak up to $2$ after the crisis, closer resembling category four. Note, the graph of category three cyclical leverage contains some extreme outliers in year 1992Q4, which increased the limits of the vertical axis up to $20$.
To ease analysis, table \ref{table:CyclicalLeverageFinancCrisis} gives us the actual cyclical values of the average leverage for the crisis periods. We marked changes $>0.04$ with red color. Similar to the graph, we can see a spill-over effect of high leverage from large to small banks. Figure \ref{fig:AssetsvsLeverageFinancCrisis} gives us a visual insight into the structural changes that occurred regarding asset size and leverage. Each data-point represents one bank. We can see a clear increase in dispersion of leverage in year 2009 among the small and medium banks. This aligns with the standard deviation shown in Figure \ref{fig:averageLeverage_Std_Categories}, where the standard deviation also has a spike in 2009. An interesting pattern within the standard deviation is the increasing volatility the smaller the asset size category becomes. Hence, within smaller banks we see much higher differences in leverage. 



\subsubsection{Leverage distribution}

\subparagraph{Long-term discussion}

In regards to the distribution of leverage, we have plotted the skewness as well as the kurtosis for all banks together in figure \ref{fig:averageLeverage}. Both variables behave similar. There are periods of strong as well as low variation. Notable periods of high variation are the S\& L crisis around 1990 and the crisis in 2008. Since high positive skewness means the graph is right-skewed with the mean being higher than the median and high kurtosis indicates heavy tails, together they prove the existence of high positive outliers. The periods with low variation in turn indicate periods of normal distribution leverage. Furthermore, the two variables only move in the positive direction (values of zero and above). For the skewness, this can be explained by the fact that banks are kind of stiff to the lower boundaries of leverage, with not much variation happening within banks of the left tail of the distribution. But there is much more variation happening between banks located at the right tail of the distribution - banks with leverage above the mode. The consistent positive kurtosis in turn tells us that we have never less outlier than a normal distribution. 
Figure \ref{fig:averageLeverage_skew_Categories} and \ref{fig:averageLeverage_kurt_Categories} give us the distribution information by asset category over time. It is important to note that the overall distribution was mainly driven by smalls banks because of their sheer quantity. Thus, the division by categories gives us a clearer view. Again, skewness and kurtosis are behaving very similar. For the top 10 and top 10-100 (cat2) banks, we have short periods where the skewness moves below zero. We take a look at those periods in the short-term discussion. The rest of the time, both measures are either zero or above for all categories, suggesting that once you have a certain amount of banks, the distribution tends to be right skewed. 
In general, we can deduce that most of the distributional changes in our graphs are driven by two factors:
\begin{enumerate}
\item Already high levered banks increasing their leverage even more (high skewness)
\item Increases of outliers (high kurtosis)
\end{enumerate}
These factors seem to be most present during the S\& L crisis around 1990 and the crisis in 2008. Hence, the significant graph movements around that time-periods.
The top 10 banks have negative kurtosis in some periods, which just means that their leverage ratios became really similar in these periods. They also show much less distributional volatility around the S\& L crisis, compared to the rest categories.   Actually, when moving along our categories, the distributional changes are higher the smaller the banks become. This aligns with the arguments made about the standard deviation in the sections before. 

\subparagraph{Short-term discussion} 

As mentioned, in our short-term analysis we investigate the reason skewness turns negative in some periods. The negative measurements of the top 10 in the crisis 2008 are particular interesting, since we associate crises with already high levered banks increasing leverage even more. However, this could indicate some low levered banks within the top 10, became high leveraged as well.
The skewness rises from 2008Q1-2008Q3 and then it takes a dive in 2008Q4 and 2009Q1. As a result, the distribution of leverage is left skewed in 2008Q4 and 2009Q1. 
This left skewness also means that the mean is to the left of the peak. To have a better overview, Figure \ref{fig:BoxplotLeverageDatapointsFinancCrisis} combines a boxplot with the top 10 banks leverage marked as a dots for the year and quarter in question. The boxplot moved significantly up from 2007 to 2008. In addition, the lower whisker increased as well. Both of these observations are characteristics of left skewness. We see here an overall increase in leverage among top 10 banks, which are not only driven by outliers. This has a significant impact on the bank industry as despite being small in numbers, the asset share of the top 10 was $60\%$ in year 2013. 


\iffalse
Figure \ref{fig:medianLeverage_cyclical_Categories} illustrates the cyclical component of the median over all banks and our categories. The behaviour of the typical banks are more stable and show consistent business cycles. Here, it is much more difficult to identify anomalies. 
\fi







\begin{figure}[hbtp]
\begin{minipage}{\textwidth}
\centering
\caption[1]{Median and Average leverage for all banks \footnote{The weighted-average leverage ratio is calculated by taking into account the asset size for each bank every point in time. Every leverage ratio for each individual bank is only accounted in the weighted-average by its share of assets compared to the total assets of all banks at that point of time.}}
\includegraphics[scale=0.3]{graphs/NewLeverage_LeverageRatioAllBanks_7613}
\label{fig:averageLeverage}
\end{minipage}
\end{figure}

\iffalse
\begin{figure}[hbtp]
\begin{minipage}{\textwidth}
\centering
\caption[1]{Median and Average leverage for all banks \footnote{Banks with leverage ratios beyond 100 are considered as outlier and removed.}}
\includegraphics[scale=0.3]{graphs/NewLeverage_LeverageRatioAllBanks_noOutlier_7613}
\label{fig:averageLeverage_noOutlier}
\end{minipage}
\end{figure}
\fi

\begin{figure}[hbtp]
\begin{minipage}{\textwidth}
\centering
\caption[1]{Average leverage by category \footnote{}}
\includegraphics[scale=0.3]{graphs/DescriptiveStats/NewLeverage_LeverageCategories_7613}
\label{fig:averageLeverage_Categories}
\end{minipage}
\end{figure}

\begin{figure}[hbtp]
\begin{minipage}{\textwidth}
\centering
\caption[1]{Cyclical average leverage by category \footnote{Category 3 contains a banks with leverage over 10000 in year 1992Q4, which results in this exorbitant high spike.}}
\includegraphics[scale=0.3]{graphs/DescriptiveStats/NewLeverage_AverageLeverageCyclical_7613.png}
\label{fig:averageLeverage_cyclical_Categories}
\end{minipage}
\end{figure}

\begin{figure}[hbtp]
\begin{minipage}{\textwidth}
\centering
\caption[1]{Cyclical average leverage by category without outlier \footnote{Datapoints above the $0.999$ quantile and below the $0.001$ are removed}}
\includegraphics[scale=0.3]{graphs/DescriptiveStats/NewLeverage_AverageLeverageCyclical_OutlierFilter_7613.png}
\label{fig:averageLeverage_cyclical_noOutlier_Categories}
\end{minipage}
\end{figure}


\begin{figure}[hbtp]
\begin{minipage}{\textwidth}
\centering
\caption[1]{Cyclical median leverage by category \footnote{}}
\includegraphics[scale=0.3]{graphs/DescriptiveStats/NewLeverage_MedianLeverageCyclical_7613.png}
\label{fig:medianLeverage_cyclical_Categories}
\end{minipage}
\end{figure}

\begin{figure}[hbtp]
\begin{minipage}{\textwidth}
\centering
\caption[1]{Standard deviation of average leverage by category \footnote{Category 3 contains a banks with leverage over 10000 in year 1992Q4, which results in this exorbitant high spike.}}
\includegraphics[scale=0.3]{graphs/NewLeverage_stdByCat_7613.png}
\label{fig:averageLeverage_Std_Categories}
\end{minipage}
\end{figure}


\begin{figure}[hbtp]
\begin{minipage}{\textwidth}
\centering
\caption[1]{Kurtosis of leverage by category \footnote{}}
\includegraphics[scale=0.3]{graphs/NewLeverage_KurtByCat_7613.png}
\label{fig:averageLeverage_kurt_Categories}
\end{minipage}
\end{figure}

\begin{figure}[hbtp]
\begin{minipage}{\textwidth}
\centering
\caption[1]{Skewness of leverage by category \footnote{}}
\includegraphics[scale=0.3]{graphs/NewLeverage_SkewByCat_7613.png}
\label{fig:averageLeverage_skew_Categories}
\end{minipage}
\end{figure}

\begin{figure}[hbtp]
\begin{minipage}{\textwidth}
\centering
\caption[1]{Scatterplot: Assets/Leverage \footnote{Banks with leverage ratios beyond 50 are considered as outlier and not included. Each data-point represents one bank. }}
\includegraphics[scale=0.3]{graphs/DescriptiveStats/NewLeverage_ScatterAssetsLeverageFinancCrisis_7613.png}
\label{fig:AssetsvsLeverageFinancCrisis}
\end{minipage}
\end{figure}

\begin{figure}[hbtp]
\begin{minipage}{\textwidth}
\centering
\caption[1]{Boxplot: Leverage data points \footnote{The first row of plots represents top 10 (cat1) and the second row top 10-100}}
\includegraphics[scale=0.3]{graphs/DescriptiveStats/NewLeverage_ScatterAssetsLeverageFinancCrisis_Cat1&2_7613.png}
\label{fig:BoxplotLeverageDatapointsFinancCrisis}
\end{minipage}
\end{figure}

\begin{figure}[hbtp]
\begin{minipage}{\textwidth}
\centering
\caption[1]{Cyclical Average Leverage \footnote{}}
\includegraphics[scale=1]{graphs/Tables/CyclicalLeverage_Crisis2008.png}
\label{table:CyclicalLeverageFinancCrisis}
\end{minipage}
\end{figure}

\iffalse
\begin{figure}[hbtp]
\begin{minipage}{\textwidth}
\centering
\caption[1]{Scatterplot: Leverage data points \footnote{}}
\includegraphics[scale=0.3]{graphs/DescriptiveStats/NewLeverage_ScatterAssetsLeverageFinancCrisis_Top10_7613.png}
\label{fig:ScatterplotLeverageDatapointsFinancCrisis}
\end{minipage}
\end{figure}
\fi





\iffalse

\begin{figure}[hbtp]
\centering
\caption{Average/Mean leverage plots}
\includegraphics[scale=0.3]{graphs/NewLeverage_LeverageRatioMean_7613}
\end{figure}


\begin{figure}[hbtp]
\centering
\caption{Median leverage plots}
\includegraphics[scale=0.3]{graphs/NewLeverage_LeverageRatioMedian_7613}
\end{figure}



\begin{figure}[hbtp]
\centering
\caption{Trend plots}
\includegraphics[scale=0.3]{graphs/NewLeverage_LeverageRatioTrends_7613.png}
\end{figure}



\begin{figure}[hbtp]
\centering
\caption{Cyclical plots}
\includegraphics[scale=0.3]{graphs/NewLeverage_LeverageRatioCyclical_7613.png}
\end{figure}



\begin{figure}[hbtp]
\centering
\caption{Correlation cyclical leverage}
\includegraphics[scale=1]{graphs/DescriptiveStats/Correlation_Cyclical_Leverage.png}
\end{figure}


\noindent \textit{Key Observations:}
\begin{itemize}
\item Extreme outliers of leverage in year 1992/93 and 2009 lead to spikes in average leverage.
\item Average-weighted leverage significantly higher from 1976 to 2004. Aligns with higher leverage of top $0.1\%$ and $1\%$ banks. Hence, the overall risk from year 1976-2004, was driven high by the larger banks. 
\item Figure 21: All banks median leverage, seasonal effect every year?
\item Leverage lowest in 2007
\item Overall Leverage did fall over time: Introduction of Basel 1 in 1988 might have lead to continuously decrease in leverage
\item Top $0.1\%$ have much higher volatility, which could just be caused by the low sample size.
\item Top $0.1\%$ and $1\%$ actually become less risky than all banks together from 2010 onwards (caused by additional capital requirements)
\item Figure 25 shows slightly negative correlation between the top $0.1\%$ banks and all banks. 
\item Crisis 2008/9: Top $0.1\%$ banks have a average leverage spike in 2008, while top $10\%$, $50\%$ and all  spike around a year later.  
\end{itemize}


\newpage

\begin{figure}[hbtp]
\centering
\caption{Boxplots (1976-2013)}
\includegraphics[scale=0.3]{graphs/Leverage/LeverageDistribution_LeverageRatioBoxplot_7613.png}
\end{figure}

\noindent \textit{Graph description}: Boxplots of all leverage ratios by banks by year. 
\\

\noindent \textit{Key Observations}:
\begin{itemize}
\item $75\%$ of all banks have a leverage ratio between 10-15.
\end{itemize}




\begin{figure}[hbtp]
\centering
\caption{Leverage Top 10 vs Rest over all years}
\includegraphics[scale=0.4]{graphs/Leverage/LeverageDistribution_LeverageRatioTop10vsRest_7813}
\end{figure}



\newpage

\noindent \textit{Graph description}: Since the top 10 banks share of assets did rise up to 60\% in 2013, it is important to differentiate. The graph shows the average leverage (assets/equity) for every year quarter 4.\\ Blue Line: Top 10 Banks by assets\\
Orange Line: All banks beside the top 10\\

\noindent \textit{Key Observations}:
\begin{itemize}
\item Leverage of top 10 banks tends to be higher
\item Trend of falling leverage is similar
\end{itemize}




\begin{figure}[hbtp]
\centering
\caption{Leverage Top10 vs Rest detailed look into crisis}
\includegraphics[scale=0.4]{graphs/Leverage/LeverageDistribution1_LeverageDetailCrisisTop10vsRest_0510}
\end{figure}


\noindent \textit{Key Observations}:
\begin{itemize}
\item Top10 banks leverage peak in year 2008/3 before the rest banks leverage peak in year 2009/4 (theory of risky assets from big banks to small transfer?) 
\end{itemize}









\begin{figure}[hbtp]
\centering
\caption{Distribution 1980-2013}
\includegraphics[scale=0.3]{graphs/Leverage/LeverageDistribution_LeverageRatio_8013}
\end{figure}

\noindent \textit{Graph description}: Counts are normed to 1. Leverage is transformed with log10. Leverage ratios are always from quarter 4.
\\

\noindent \textit{Key Observations}:
\begin{itemize}
\item Log-normal distribution
\item Large standard deviation in year 2010 with 18.82
\item Less and higher bars in 2012 indicate higher homogeneity in 2013 compared to the years before.
\end{itemize}

\pagebreak


\begin{figure}[hbtp]
\centering
\caption{Distribution in crisis 2003-2011}
\includegraphics[scale=0.3]{graphs/Leverage/LeverageDistribution_LeverageRatio_0311}
\end{figure}

\noindent \textit{Graph description}: Leverage ratios are always from quarter 4 and  logarithmised.

\noindent \textit{Key Observations}:
\begin{itemize}
\item Increasing homogenity over time.
\end{itemize}

\fi

\newpage
\section{Evaluation/Outlook}

In general, this paper gives a broad overview over the us commercial bank landscape and key important factors that should be considered. The next steps would be to choose a specific focus/section and go into more detail. For instance, it would be interesting to find an optimal way to categorize us commercial banks. The literature seems to have found no coherent way of categorization. These categories would obviously be of key importance to regulators.


Further points
\begin{itemize}
\item Shadow banking/investment banks not considered,significant part of trading assets still owned by non commercial bank. Investment banks do hold a high share of total assets, which were not considered in this analysis. It were investment banks which had to bear the major impacts of the financial crisis in 2007. 
\item  Another factor we have not considered is that according to \cite{Sebnem Kalemli-Ozcan, Bent Sorensen,Sevcan Yesiltas 2011} a big fraction of assets , especially for large commercial banks, are off balance sheet items. Especially, large amounts of financial innovation occurring in our time-frame caused them to rise significantly.
\item Valuations not realistic, book values. Loans are recorded at face value, which makes assets value changes less realistic. 
\item Applying more time-series models
\item Applying of models such as Regression...
\item Significant part of trading assets still owned by non commercial banks
\item Correlation between assets and liabilities: Key part of Asset liability management for banks is maturity transformation. For correlation analysis, we should have differed between the different maturities of assets and liabilities. Correlations between positions of different maturity would have a more causal relationship. In addition, canonical correlation analysis could have been used to consider that balance sheet positions are jointly determined by the other positions. Also, the cyclical variation of shares instead of the cyclical variation of the log of absolutes values could have been used.
\item Cyclical of share could have been analysed instead of absolute values (Some literature work with shares)
\item Total assets represent the indicator which regulators and academics use most
frequently for categorising. It measures the gross nominal volume of a bank’s activities, but
suffers from significant valuation problems, not only for derivatives, and it does
not account for differences in individual bank business models or between
financial systems.
\item Leverage: could have looked at how leverage behaves to other balance sheet accounts such as loans to total assets
\item considered different factors that affect leverage and look at them independently
\item Our choice of categorization could have been different. The asset size ranges they cover differ over the years. This can be seen as an advantage or disadvantage. On the one side they evolve over the years and possibly match changing asset size levels. On the other side, there is a risk of distributional changes among the asset sizes of banks, making our chosen categorization unsuitable.  
\end{itemize}
 



\section{Conclusion}

Along the way of our analysis it were often the outliers that drive the measurements. This aligns with the interdependent banks system of today, where just one bankrupt banks can lead to significant spillover effects. Hence, we took those outliers into careful consideration and did not consistently filter them out. 


% ================================================================================
% 							Literaturverzeichnis
% ================================================================================

\newpage
\printbibliography[
heading=bibintoc,
title={Bibliography}
]

 




% ================================================================================
% 								Appendix
% ================================================================================
\appendix
\section{Appendix}

\begin{figure}[hbtp]
\begin{minipage}{\textwidth}
\centering
\caption[1]{Correlations: Category 1-4 \footnote{This graph shows the correlation of a banks size category assets with the lagged assets of another banks size category. The category after the "+" is the lagged category. Hence, the first graph shows the correlation between categories 1 aggregate assets and all the different other categories lagged aggregate assets.}}
\includegraphics[scale=0.75]{graphs/Tables/autocorrelations/AssetsCat1}
\includegraphics[scale=0.75]{graphs/Tables/autocorrelations/AssetsCat2}
\includegraphics[scale=0.75]{graphs/Tables/autocorrelations/AssetsCat3}
\includegraphics[scale=0.75]{graphs/Tables/autocorrelations/AssetsCat4}
\label{autocorrelationsCat14}
\end{minipage}
\end{figure}




\begin{figure}[hbtp]
\begin{minipage}{\textwidth}
\centering
\caption[1]{Banks count by asset size \footnote{The left column is the asset interval size and the corresponding row the number of banks per year.}}
\includegraphics[scale=1]{graphs/MedianBank_FrequencyTable_8010}
\end{minipage}
\end{figure}

\iffalse
\begin{figure}[hbtp]
\begin{minipage}{\textwidth}
\centering
\caption[1]{Assets vs Leverage \footnote{Banks with leverage ratios beyond 50 are considered as outlier and not included.}}
\includegraphics[scale=0.1]{graphs/DescriptiveStats/NewLeverage_ScatterAssetsLeverage_7613.png}
\label{fig:AssetsvsLeverage}
\end{minipage}
\end{figure}
\fi

\begin{figure}[hbtp]
\begin{minipage}{\textwidth}
\centering
\caption[1]{Cyclical standard deviation of average leverage by category \footnote{Category 3 contains a banks with leverage over 10000 in year 1992Q4, which results in this exorbitant high spike.}}
\includegraphics[scale=0.3]{graphs/NewLeverage_stdByCat_cyclical_7613.png}
\label{fig:averageLeverage_cyclicalStd_Categories}
\end{minipage}
\end{figure}


\begin{figure}[hbtp]
\begin{minipage}{\textwidth}
\centering
\caption[1]{Cyclical skewness of leverage by category \footnote{}}
\includegraphics[scale=0.3]{graphs/NewLeverage_KurtByCat_cyclical_7613.png}
\label{fig:averageLeverage_kurt_cyclical_Categories}
\end{minipage}
\end{figure}

\begin{figure}[hbtp]
\begin{minipage}{\textwidth}
\centering
\caption[1]{Cyclical kurtosis of leverage by category \footnote{}}
\includegraphics[scale=0.3]{graphs/NewLeverage_SkewByCat_cyclical_7613.png}
\label{fig:averageLeverage_skew_cyclical_Categories}
\end{minipage}
\end{figure}


\end{document}