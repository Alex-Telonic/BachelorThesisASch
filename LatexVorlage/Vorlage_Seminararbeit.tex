\documentclass[12pt, a4paper]{article} % Dokumentenklasse

\usepackage[utf8]{inputenc} % Zeichensatz und Schrift
\usepackage[T1]{fontenc}
\usepackage{lmodern} % nice alternative is {mathpazo}
\usepackage[english]{babel}

\usepackage[left=2.5cm, right=2.5cm, top=2.5cm, bottom=2.50cm]{geometry} % Seitenformat

\usepackage{amsmath} % Mathematikzeichen
\usepackage{amsfonts}
\usepackage{amssymb}
\usepackage{mathtools}

\usepackage{booktabs} % Tabellen
\usepackage{array}
\usepackage{dcolumn}
\usepackage{tabularx}
\usepackage{threeparttable}

\usepackage{graphicx} % Grafiken
\usepackage{subfigure}



\usepackage{lipsum} % Paket für Random-Text (nur für Anschauungszwecke benötigt, kann gelöscht werden)

\usepackage{setspace} % Zeilenabstand
\onehalfspacing

\usepackage{placeins}

%% Literaturverzeichnis und Zitation % %

\usepackage[%
citestyle=authoryear-comp,%
bibstyle=JME,%
maxbibnames=99,% %maximum number of names printed in bibliography before truncation with ``et al.'' is used
maxcitenames=2,
datezeros=false,% no leading 0 if dates are printed
date=long,%
isbn=false,% show no ISBNs
natbib=true,% enable natbib-compatibility
url=false,% show no urls
backend=biber % use Biber here
]{biblatex}

% vergrößert Abstand zwischen den Quellen im Lit.verzeichnis

\setlength\bibitemsep{1.5\itemsep}

% setzt "et al." bei deutscher Sprache

\DefineBibliographyStrings{ngerman}{%
  andothers = {et\addabbrvspace al\adddot}
}


% vermeidet Witwen und Waisen

\clubpenalty10000 
\widowpenalty10000
\displaywidowpenalty=10000 



% Hier .bib-Datei einfügen

\bibliography{bspbib}

\usepackage[font=scriptsize, labelfont=bf]{caption}

%---------------------------------------------------------------------------------------
%
%								HIER BEGINNT DAS DOKUMENT
%
%---------------------------------------------------------------------------------------


\begin{document}

% TITELSEITE

\begin{titlepage}
\begin{center}
\includegraphics[scale=0.65]{KITlogo}\\
%\Large
%Karlsruher Institut für Technologie\\
\large
Fakultät für Wirtschaftswissenschaften\\
Institut für Volkswirtschaftslehre (ECON)
\vfill{
%\\\vspace{2cm}
\Large
Bachelor Thesis in Macroeconomics\\\vspace{0.5cm}
,,US commercial banks``\\\vspace{0.5cm}
Winter semester 2019/20% \vspace{2cm}
}
\vfill{
\LARGE
% -------------------------------------------------------------------------------------
% Ihr Thema
Not yet defined
% -------------------------------------------------------------------------------------
\\ 
\vspace{0.5cm}
\normalsize
% -------------------------------------------------------------------------------------
% Nummer Ihres Themas
(Topic 1)
% -------------------------------------------------------------------------------------
}
\end{center}

\vfill{
\normalsize
% -------------------------------------------------------------------------------------
% Ihre pers\"{o}nliche Daten
\noindent Alexander Schlechter \\
Matr.-Nr. 2054108 \\
alexander.schlechter@student.kit.edu}
% -------------------------------------------------------------------------------------
\end{titlepage}
\newpage

% VERZEICHNISSE

\pagenumbering{Roman}
\tableofcontents
% \listoffigures
% \listoftables
\newpage
\pagenumbering{arabic}


% ================================================================================
% 									Hauptteil
% ================================================================================


\section{Introduction}


\section{Main part}

\subsection{General look at us banks}

\subsubsection{Assets}

The following section will show some insights about the asset side of the whole us commercial banking sector. The asset side is composed of: ... 

\begin{figure}[hbtp]
\centering
\caption{Asset side}
\includegraphics[scale=0.3]{graphs/DescriptiveStats/OtherAnalysis_AssetDistribution_7613.png}
\end{figure}

\newpage


\begin{figure}[hbtp]
\centering
\caption{Share of asset positions - unstacked}
\includegraphics[scale=0.3]{graphs/DescriptiveStats/OtherAnalysis_ShareofAssetsPlot_7613.png}
\end{figure}


\noindent \textit{Graph description}: Figure 1 shows the aggregates of the main variables from the asset side of the balance sheet over time. Figure 2 shows the share of each aggregated balance sheet position of all commercial banks over time. Figure 3 plots the share of each balance sheet position unstacked.\\


\noindent \textit{Key Observations:}
\begin{itemize}
\item loans make up the largest share of assets
\item share of trading assets have risen as well as interest rate derivatives
\item loans and trading assets have risen more than securities in timeframe year 2000-2009
\item Share of trading assets peaked in 2008 while securities fell.
\item There is a noticeable anomaly in year 2002. Significant amounts of repo lending is transferred into other assets. Other assets are derivatives not available for sale. 
\item drop in assets in 2002 and 2008
\item share in cash has fallen until 2008 and then increased again
\item Share of cash continously fell until 2008, and then it increased significantly again
\end{itemize}

\newpage

\begin{figure}[hbtp]
\centering
\caption{Correlation of balance sheet positions (assets side)}
\includegraphics[scale=0.5]{graphs/DescriptiveStats/OtherAnalysis_AssetsCorrelationHeatMap_7613.png}
\end{figure}

\noindent \textit{Graph description}: Figure 3 shows the Pearson Correlation Coefficient between different balance sheet positions\\

\textit{Key Observations:}
\begin{itemize}
\item Cash has the weakest correlation with the rest positions
\item No significant stand outs
\end{itemize}

\newpage

\subsubsection{Liabilities}

\if(false)
\begin{figure}[hbtp]
\centering
\caption{Liabilities side}
\includegraphics[scale=0.3]{graphs/DescriptiveStats/OtherAnalysis_LiabilitiesDistribution_7613}
\end{figure}


\begin{figure}[hbtp]
\centering
\caption{Share of liabilities positions}
\includegraphics[scale=0.3]{graphs/DescriptiveStats/OtherAnalysis_ShareofLiabilities_7613}
\end{figure}

\newpage

\begin{figure}[hbtp]
\centering
\caption{Share of liabilities positions}
\includegraphics[scale=0.3]{graphs/DescriptiveStats/OtherAnalysis_ShareofLiabilitiesPlot_7613}
\end{figure}
\fi

\begin{figure}[hbtp]
\centering
\caption{Share of liabilities positions}
\includegraphics[scale=0.3]{graphs/DescriptiveStats/OtherAnalysis_ShareofLiabilitiesAll3_7613.png}
\end{figure}


\noindent \textit{Graph description}: The graph shows the aggregates of the main variables from the liabilities side of the balance sheet over time.\\

\noindent \textit{Key Observations:}
\begin{itemize}
\item Deposits as main source of funding
\item Irregularities in year 2002: repos drop while other liabilities rise
\item Irregularity in year 1983 might be caused by measuring/reporting differences
\item In 2008 share of deposits at lowest point. Although the aggregated assets peaked at that time. 
\item Deposits continuously decreased from 1990 onwards. Other financing such as "Other borrowed money" and "trading liabilities" rose
\item repos share decreased significantly until end of 2013
\end{itemize}



\begin{figure}[hbtp]
\centering
\caption{Correlation of balance sheet positions (liabilities side)}
\includegraphics[scale=0.5]{graphs/DescriptiveStats/OtherAnalysis_LiabilitiesCorrelationHeatMap_7613}
\end{figure}

\newpage

\subsection{Growth}


\begin{figure}[hbtp]
\centering
\caption{Growth of assets}
\includegraphics[scale=0.3]{graphs/DescriptiveStats/OtherAnalysis_GrowthAssetsYear_7613}
\end{figure}

\noindent \textit{Graph description}: The graph shows the annual growth rate of aggregated assets of all commercial banks. Two investment banks, who did become commercial banks in 2009, are excluded. \\

\noindent \textit{Key Observations:}
\begin{itemize}
\item Three negative growth rates in year 1991, 2001, 2010
\end{itemize}

\begin{figure}[hbtp]
\centering
\caption{Growth of top 1 percent banks assets}
\includegraphics[scale=0.3]{graphs/DescriptiveStats/OtherAnalysis_GrowthAssetsYear1_7613.png}
\end{figure}

\noindent \textit{Graph description}: The graph shows the annual growth rate of aggregated assets of top top $1\%$ commercial banks. Two investment banks, who did become commercial banks in 2009, are excluded. \\

\noindent \textit{Key Observations:}
\begin{itemize}
\item More negative growth rates in 1990 and 2010
\item No negative growth in 2001

\end{itemize}

\begin{figure}[hbtp]
\centering
\caption{Growth of all banks vs top 1 percent}
\includegraphics[scale=0.3]{graphs/DescriptiveStats/OtherAnalysis_GrowthTrends_7613}
\end{figure}

\noindent \textit{Graph description}: Annual growth rate with trend for all banks and top $1\%$. The second graph shows the cycle part from the time series filter. \\

\noindent \textit{Key Observations:}
\begin{itemize}
\item Top $1\%$ growth rates are more volatile
\item Pearson Correlation between all banks vs top $1\%$, Significance: (0.6371607133788253, 1.696703469447756e-05)
\item 1986, 2001, 2006 almost 0,02 difference toward trend
\end{itemize}

\newpage


\begin{figure}[hbtp]
\centering
\caption{Banks default}
\includegraphics[scale=0.3]{graphs/DescriptiveStats/NewLeverage_BanksDefaultPercentage_7613.png}
\end{figure}

\noindent \textit{Graph description}: The graph shows an estimation of how many banks have defaulted at a certain time (year,quarter). It is based on the negative equity recorded by banks. Hence, it is not exact and some banks might continue to exist in case of mergers or bailouts. Also sometimes banks are double counted, if a negative equity does not immediately result in bankruptcy. \\
\noindent \textit{Key Observations:}
\begin{itemize}
\item main defaults in years 1986-1991 and 2009-2011
\item long stable period from 1991-2008
\item In 1990 there were many more smaller banks. Smaller banks might have a higher likelihood to fail. In 1990:  $~74\%$ small banks, 2010: $~35\%$ small banks 
\end{itemize}


\subsection{Loans}

\begin{figure}[hbtp]
\centering
\caption{Loans}
\includegraphics[scale=0.3]{graphs/DescriptiveStats/OtherAnalysis_LoanDistribution_7613}
\end{figure}

\noindent \textit{Graph description}: It shows the share of loan types of total loans over time.\\

\noindent \textit{Key Observations:}
\begin{itemize}
\item real estate loans has largest share

\end{itemize}

\begin{figure}[hbtp]
\centering
\caption{Loans by repricing maturity}
\includegraphics[scale=0.3]{graphs/DescriptiveStats/OtherAnalysis_ShareofMaturityLoans_9713}
\end{figure}


\begin{figure}[hbtp]
\centering
\caption{Residential Loans by repricing maturity}
\includegraphics[scale=0.3]{graphs/DescriptiveStats/OtherAnalysis_ShareofMaturityResLoans_9713}
\end{figure}

\newpage

\subsection{Distribution of asset sizes among banks - To Big to Fail}

\begin{figure}[hbtp]
\centering
\caption{Count of banks by percentiles}
\includegraphics[scale=0.8]{graphs/DescriptiveStats/BankCounts}
\end{figure}

\newpage

\begin{figure}[hbtp]
\centering
\caption{Aggregate assets by percentiles}
\includegraphics[scale=0.3]{graphs/DescriptiveStats/OtherAnalysis_ShareofAssetsByBanksPercentiles_7613.png}
\end{figure}

\iffalse
\begin{figure}[hbtp]
\centering
\caption{Top 10 banks assets vs rest}
\includegraphics[scale=0.3]{graphs/DescriptiveStats/LeverageDistribution_AssetsTop10vsRest_7813}
\end{figure}

\begin{figure}[hbtp]
\centering
\caption{Rise of top 10 banks asset share}
\includegraphics[scale=0.3]{graphs/DescriptiveStats/LeverageDistribution_shareTop10_7813}
\end{figure}
\fi


\subsection{Median banks by asset size}

In Figure 15, we have in the left column the asset interval size and in the corresponding row the number of banks per year.\\

\begin{figure}[hbtp]
\centering
\caption{Banks count by asset size}
\includegraphics[scale=1]{graphs/MedianBank_FrequencyTable_8010}
\end{figure}


\textbf{Typical small/medium/large bank}

Banks are assigned three different buckets (small/medium/large) depending on asset size.\\

Small bank: $0<$ assets $<= 10^5$

Medium bank: $10^5<$ assets $<= 10^6$

Large bank: $10^6<$ assets 

\begin{figure}[hbtp]
\centering
\caption{Asset size by bank}
\includegraphics[scale=0.3]{graphs/DescriptiveStats/MedianBank_SmallvsMiddlevsBigMedianBankAssets_7613.png}
\end{figure}

\begin{figure}[hbtp]
\centering
\caption{Medium vs large bank by asset size}
\includegraphics[scale=0.3]{graphs/DescriptiveStats/MedianBank_MiddlevsBigMedianBankAssets_7613.png}
\end{figure}


\begin{figure}[hbtp]
\centering
\caption{Small bank: liability side}
\includegraphics[scale=0.3]{graphs/DescriptiveStats/MedianBank_ShareofLiabilitiesSmallBank_7613.png}
\end{figure}


\begin{figure}[hbtp]
\centering
\caption{Medium bank: liability side}
\includegraphics[scale=0.3]{graphs/DescriptiveStats/MedianBank_ShareofLiabilitiesMediumBank_7613.png}
\end{figure}


\begin{figure}[hbtp]
\centering
\caption{Large bank: liability side}
\includegraphics[scale=0.3]{graphs/DescriptiveStats/MedianBank_ShareofLiabilitiesBigBank_7613.png}
\end{figure}



\newpage

\subsection{Leverage}

Throughout the analysis the definition of accounting leverage (assets/equity) is used. Equity is calculated by total assets minus total liabilities. In addition, for risk analysis banks belonging a bank holding company were aggregated. Hence, the dataset which was used contained bhcs and independent banks.

\begin{figure}[hbtp]
\centering
\caption{Median and Average leverage for all banks}
\includegraphics[scale=0.3]{graphs/NewLeverage_LeverageRatioAllBanks_7613}
\end{figure}

\newpage

\begin{figure}[hbtp]
\centering
\caption{Average/Mean leverage plots}
\includegraphics[scale=0.3]{graphs/NewLeverage_LeverageRatioMean_7613}
\end{figure}


\begin{figure}[hbtp]
\centering
\caption{Median leverage plots}
\includegraphics[scale=0.3]{graphs/NewLeverage_LeverageRatioMedian_7613}
\end{figure}

\newpage

\begin{figure}[hbtp]
\centering
\caption{Trend plots}
\includegraphics[scale=0.3]{graphs/NewLeverage_LeverageRatioTrends_7613.png}
\end{figure}


\begin{figure}[hbtp]
\centering
\caption{Cyclical plots}
\includegraphics[scale=0.3]{graphs/NewLeverage_LeverageRatioCyclical_7613.png}
\end{figure}

\newpage

\begin{figure}[hbtp]
\centering
\caption{Correlation cyclical leverage}
\includegraphics[scale=1]{graphs/DescriptiveStats/Correlation_Cyclical_Leverage.png}
\end{figure}

\noindent \textit{Graph description}: Figures 21-25 focus on leverage ratios for every year and quarter over all banks. Banks with equity or assets below zero are excluded.


\noindent \textit{Key Observations:}
\begin{itemize}
\item Extreme outliers of leverage in year 1992/93 and 2009 lead to spikes in average leverage.
\item Figure 21: All banks median leverage, seasonal effect every year?
\item Leverage lowest in 2007
\item Overall Leverage did fall over time: Introduction of Basel 1 in 1988 might have lead to continuously decrease in leverage
\item Top $0.1\%$ have much higher volatility, which could just be caused by the low sample size.
\item Top $0.1\%$ and $1\%$ actually become less risky than all banks together from 2010 onwards
\item Figure 25 shows slightly negative correlation between the top $0.1\%$ banks and all banks. 
\end{itemize}

\newpage

\begin{figure}[hbtp]
\centering
\caption{Boxplots (1976-2013)}
\includegraphics[scale=0.3]{graphs/Leverage/LeverageDistribution_LeverageRatioBoxplot_7613.png}
\end{figure}

\noindent \textit{Graph description}: Boxplots of all leverage ratios by banks by year. 
\\

\noindent \textit{Key Observations}:
\begin{itemize}
\item $75\%$ of all banks have a leverage ratio between 10-15.
\end{itemize}



\iffalse
\begin{figure}[hbtp]
\centering
\caption{Leverage Top 10 vs Rest over all years}
\includegraphics[scale=0.4]{graphs/Leverage/LeverageDistribution_LeverageRatioTop10vsRest_7813}
\end{figure}


\newpage

\noindent \textit{Graph description}: Since the top 10 banks share of assets did rise up to 60\% in 2013, it is important to differentiate. The graph shows the average leverage (assets/equity) for every year quarter 4.\\ Blue Line: Top 10 Banks by assets\\
Orange Line: All banks beside the top 10\\

\noindent \textit{Key Observations}:
\begin{itemize}
\item Leverage of top 10 banks tends to be higher
\item Trend of falling leverage is similar
\end{itemize}




\begin{figure}[hbtp]
\centering
\caption{Leverage Top10 vs Rest detailed look into crisis}
\includegraphics[scale=0.4]{graphs/Leverage/LeverageDistribution1_LeverageDetailCrisisTop10vsRest_0510}
\end{figure}


\noindent \textit{Key Observations}:
\begin{itemize}
\item Top10 banks leverage peak in year 2008/3 before the rest banks leverage peak in year 2009/4 (theory of risky assets from big banks to small transfer?) 
\end{itemize}
\fi
\pagebreak

\textbf{A look into the distribution of leverage}\\



\begin{figure}[hbtp]
\centering
\caption{Distribution 1980-2013}
\includegraphics[scale=0.3]{graphs/Leverage/LeverageDistribution_LeverageRatio_8013}
\end{figure}

\noindent \textit{Graph description}: Counts are normed to 1. Leverage are transformed with log10. Leverage ratios are always from quarter 4.
\\

\noindent \textit{Key Observations}:
\begin{itemize}
\item Log-normal distribution
\item Large standard deviation in year 2010 with 18.82
\item Less and higher bars in 2012 indicate higher homogeneity in 2013 compared to the years before.
\end{itemize}

\pagebreak

\iffalse
\begin{figure}[hbtp]
\centering
\caption{Distribution in crisis 2003-2011}
\includegraphics[scale=0.3]{graphs/Leverage/LeverageDistribution_LeverageRatio_0311}
\end{figure}

\noindent \textit{Graph description}: Counts are normed to 1. Only leverage ratios between 0-20 are accounted for. The others are seen as outliers. As you can see ~50$\%$ of banks are within a range of 8-12. Leverage ratios are always from quarter 4.

\noindent \textit{Key Observations}:
\begin{itemize}
\item Increasing homogenity over time.
\end{itemize}
\fi

\section{Conclusion}




% ================================================================================
% 							Literaturverzeichnis
% ================================================================================

\newpage
\printbibliography[
heading=bibintoc,
title={Bibliography}
]


% ================================================================================
% 								Appendix
% ================================================================================

%\appendix
%\section{Anhang}




\end{document}