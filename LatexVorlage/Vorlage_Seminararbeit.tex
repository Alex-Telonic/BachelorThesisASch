\documentclass[12pt, a4paper]{article} % Dokumentenklasse

\usepackage[utf8]{inputenc} % Zeichensatz und Schrift
\usepackage[T1]{fontenc}
\usepackage{lmodern} % nice alternative is {mathpazo}
\usepackage[english]{babel}

\usepackage[left=2.5cm, right=2.5cm, top=2.5cm, bottom=2.50cm]{geometry} % Seitenformat

\usepackage{amsmath} % Mathematikzeichen
\usepackage{amsfonts}
\usepackage{amssymb}
\usepackage{mathtools}

\usepackage{booktabs} % Tabellen
\usepackage{array}
\usepackage{dcolumn}
\usepackage{tabularx}
\usepackage{threeparttable}

\usepackage{graphicx} % Grafiken
\usepackage{subfigure}

\usepackage{caption} % Paket für Tabellen- und Bildbeschriftungen
\usepackage{lipsum} % Paket für Random-Text (nur für Anschauungszwecke benötigt, kann gelöscht werden)

\usepackage{setspace} % Zeilenabstand
\onehalfspacing

\usepackage{placeins}

%% Literaturverzeichnis und Zitation % %

\usepackage[%
citestyle=authoryear-comp,%
bibstyle=JME,%
maxbibnames=99,% %maximum number of names printed in bibliography before truncation with ``et al.'' is used
maxcitenames=2,
datezeros=false,% no leading 0 if dates are printed
date=long,%
isbn=false,% show no ISBNs
natbib=true,% enable natbib-compatibility
url=false,% show no urls
backend=biber % use Biber here
]{biblatex}

% vergrößert Abstand zwischen den Quellen im Lit.verzeichnis

\setlength\bibitemsep{1.5\itemsep}

% setzt "et al." bei deutscher Sprache

\DefineBibliographyStrings{ngerman}{%
  andothers = {et\addabbrvspace al\adddot}
}


% vermeidet Witwen und Waisen

\clubpenalty10000 
\widowpenalty10000
\displaywidowpenalty=10000 



% Hier .bib-Datei einfügen

\bibliography{bspbib}

%---------------------------------------------------------------------------------------
%
%								HIER BEGINNT DAS DOKUMENT
%
%---------------------------------------------------------------------------------------


\begin{document}

% TITELSEITE

\begin{titlepage}
\begin{center}
\includegraphics[scale=0.65]{KITlogo}\\
%\Large
%Karlsruher Institut für Technologie\\
\large
Fakultät für Wirtschaftswissenschaften\\
Institut für Volkswirtschaftslehre (ECON)
\vfill{
%\\\vspace{2cm}
\Large
Bachelor Thesis in Macroeconomics\\\vspace{0.5cm}
,,US commercial banks``\\\vspace{0.5cm}
Winter semester 2019/20% \vspace{2cm}
}
\vfill{
\LARGE
% -------------------------------------------------------------------------------------
% Ihr Thema
Not yet defined
% -------------------------------------------------------------------------------------
\\ 
\vspace{0.5cm}
\normalsize
% -------------------------------------------------------------------------------------
% Nummer Ihres Themas
(Topic 1)
% -------------------------------------------------------------------------------------
}
\end{center}

\vfill{
\normalsize
% -------------------------------------------------------------------------------------
% Ihre pers\"{o}nliche Daten
\noindent Alexander Schlechter \\
Matr.-Nr. 2054108 \\
alexander.schlechter@student.kit.edu}
% -------------------------------------------------------------------------------------
\end{titlepage}
\newpage

% VERZEICHNISSE

\pagenumbering{Roman}
\tableofcontents
% \listoffigures
% \listoftables
\newpage
\pagenumbering{arabic}


% ================================================================================
% 									Hauptteil
% ================================================================================


\section{Introduction}


\section{Main part}

\subsection{General look at us banks}



\begin{figure}[hbtp]
\centering
\caption{Asset side}
\includegraphics[scale=0.3]{graphs/DescriptiveStats/OtherAnalysis_AssetDistribution_7613.png}
\end{figure}


\noindent \textit{Graph description}: The graph shows the aggregates of the main variables from the asset side of the balance sheet over time.\\

\noindent \textit{Key Observations:}
\begin{itemize}
\item loans make up the largest share of assets
\item share of trading assets have risen
\item loans and trading assets have risen more than securities
\item drop in assets in 2002 and 2008
\end{itemize}

\newpage

\begin{figure}[hbtp]
\centering
\caption{Share of asset positions}
\includegraphics[scale=0.3]{graphs/DescriptiveStats/OtherAnalysis_ShareofAssets_7613.png}
\end{figure}


\begin{figure}[hbtp]
\centering
\caption{Liabilities side}
\includegraphics[scale=0.3]{graphs/DescriptiveStats/OtherAnalysis_LiabilitiesDistribution_7613}
\end{figure}

\noindent \textit{Graph description}: The graph shows the aggregates of the main variables from the liabilities side of the balance sheet over time.\\

\noindent \textit{Key Observations:}
\begin{itemize}
\item deposits as main source of funding
\end{itemize}


\begin{figure}[hbtp]
\centering
\caption{Share of liabilities positions}
\includegraphics[scale=0.3]{graphs/DescriptiveStats/OtherAnalysis_ShareofLiabilities_7613}
\end{figure}

\begin{figure}[hbtp]
\centering
\caption{Growth of assets}
\includegraphics[scale=0.3]{graphs/DescriptiveStats/OtherAnalysis_GrowthAssetsYear_7613}
\end{figure}


\begin{figure}[hbtp]
\centering
\caption{Loans}
\includegraphics[scale=0.3]{graphs/DescriptiveStats/OtherAnalysis_LoanDistribution_7613}
\end{figure}

\noindent \textit{Graph description}: It shows the share of loan types of total loans over time.\\

\noindent \textit{Key Observations:}
\begin{itemize}
\item real estate loans has largest share

\end{itemize}

\begin{figure}[hbtp]
\centering
\caption{Loans by maturity}
\includegraphics[scale=0.3]{graphs/DescriptiveStats/OtherAnalysis_ShareofMaturityLoans_9713}
\end{figure}


\begin{figure}[hbtp]
\centering
\caption{Residential Loans by maturity}
\includegraphics[scale=0.3]{graphs/DescriptiveStats/OtherAnalysis_ShareofMaturityResLoans_9713}
\end{figure}


\begin{figure}[hbtp]
\centering
\caption{Banks by percentiles}
\includegraphics[scale=0.3]{graphs/DescriptiveStats/OtherAnalysis_BanksPercentiles_7613}
\end{figure}

\begin{figure}[hbtp]
\centering
\caption{Top10 asset share}
\includegraphics[scale=0.3]{graphs/DescriptiveStats/LeverageDistribution_AssetsTop10vsRest_7813}
\end{figure}


\begin{figure}[hbtp]
\centering
\caption{Rise of Top10 asset share}
\includegraphics[scale=0.3]{graphs/DescriptiveStats/LeverageDistribution_shareTop10_7813}
\end{figure}

\textbf{Typical bank}

\begin{figure}[hbtp]
\centering
\caption{Median bank financing over time}
\includegraphics[scale=0.3]{graphs/DescriptiveStats/OtherAnalysis_MedianBankEveryYear_7613}
\end{figure}
\noindent \textit{Graph description}: Each pie chart shows how the liabilities side of a typical bank looks like. 

\newpage

\subsection{Looking into leverage}

Throughout the analysis the definition of accounting leverage (assets/equity) is used. Equity is calculated by total assets minus total liabilities. In addition, for risk analysis banks belonging a bank holding company were aggregated. Hence, the dataset which was used contained bhcs and independent banks.

\begin{figure}[hbtp]
\centering
\caption{Average Leverage over all years}
\includegraphics[scale=0.4]{graphs/Leverage/LeverageRatio0_AssetEquityRatio_7813}
\end{figure}

\noindent \textit{Graph description}: The graph shows the average leverage (assets/equity) for every year over all banks. Banks with equity or assets below zero are excluded.


\noindent \textit{Key Observations:}
\begin{itemize}
\item Overall Leverage did fall over time
\item Spike in leverage in year 2008/2009
\item Leverage lowest in 2007
\item Small spike in year 1999
\item Introduction of Basel 1 in 1988 might have lead to continuously decrease in leverage
\end{itemize}


\begin{figure}[hbtp]
\centering
\caption{Leverage Top 10 vs Rest over all years}
\includegraphics[scale=0.4]{graphs/Leverage/LeverageDistribution_LeverageRatioTop10vsRest_7813}
\end{figure}


\newpage

\noindent \textit{Graph description}: Since the top 10 banks share of assets did rise up to 60\% in 2013, it is important to differentiate. The graph shows the average leverage (assets/equity) for every year quarter 4.\\ Blue Line: Top 10 Banks by assets\\
Orange Line: All banks beside the top 10\\

\noindent \textit{Key Observations}:
\begin{itemize}
\item Leverage of top 10 banks tends to be higher
\item Trend of falling leverage is similar
\end{itemize}




\begin{figure}[hbtp]
\centering
\caption{Leverage Top10 vs Rest detailed look into crisis}
\includegraphics[scale=0.4]{graphs/Leverage/LeverageDistribution1_LeverageDetailCrisisTop10vsRest_0510}
\end{figure}

\noindent \textit{Key Observations}:
\begin{itemize}
\item Top10 banks leverage peak in year 2008/3 before the rest banks leverage peak in year 2009/4 (theory of risky assets from big banks to small transfer?) 
\end{itemize}

\pagebreak

\textbf{A look into the distribution of leverage}\\



\begin{figure}[hbtp]
\centering
\caption{Distribution 1980-2013}
\includegraphics[scale=0.3]{graphs/Leverage/LeverageDistribution_LeverageRatio_8013}
\end{figure}

\noindent \textit{Graph description}: Counts are normed to 1. Only leverage ratios between 0-20 are accounted for. The others are seen as outliers. Leverage ratios are always from quarter 4.
\\

\noindent \textit{Key Observations}:
\begin{itemize}
\item large standard deviation in year 2010 with 18.82
\item less and higher bars in 2013 indicate higher homogeneity in 2013 compared to the years before.
\end{itemize}

\pagebreak


\begin{figure}[hbtp]
\centering
\caption{Distribution in crisis 2003-2011}
\includegraphics[scale=0.3]{graphs/Leverage/LeverageDistribution_LeverageRatio_0311}
\end{figure}

\noindent \textit{Graph description}: Counts are normed to 1. Only leverage ratios between 0-20 are accounted for. The others are seen as outliers. Leverage ratios are always from quarter 4.

\noindent \textit{Key Observations}:
\begin{itemize}
\item Increasing homogenity over time.
\item extremely high standard deviation in year 2009 with 168.22
\end{itemize}


\section{Conclusion}




% ================================================================================
% 							Literaturverzeichnis
% ================================================================================

\newpage
\printbibliography[
heading=bibintoc,
title={Bibliography}
]


% ================================================================================
% 								Appendix
% ================================================================================

%\appendix
%\section{Anhang}




\end{document}